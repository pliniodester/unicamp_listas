\documentclass[answers, 12pt]{exam}

\usepackage{graphicx,epsfig,psfrag,rotate,xcolor,url}
\usepackage{amsmath,amsthm,amsfonts,amssymb,mathrsfs,latexsym,multicol}
\usepackage{setspace,enumerate,enumitem,ifthen,subfig}
\usepackage{hyperref}
\usepackage{siunitx,tcolorbox}
\usepackage[portuges]{babel}
\usepackage[utf8]{inputenc}
\usepackage[algo2e,english,onelanguage,algoruled]{algorithm2e}
\usepackage[margin=0.7in]{geometry}
\usepackage{calrsfs,empheq}
\renewcommand{\rmdefault}{pplx}
\usepackage{cancel}
\renewcommand{\CancelColor}{\color{red}}
\usepackage{eulervm}
\DeclareMathAlphabet{\mathcal}{OMS}{zplm}{m}{n}
\def\defeq{\mathrel{\mathop:}=}

\usepackage{tikz, circuitikz, pgfplots}
\pgfplotsset{compat=1.15}

\newcommand{\euler}{\mathrm{e}}
\newcommand\diff{\mathrm{d}}
\newcommand{\Log}{\mathop{Log}}

\renewcommand{\thequestion}{\arabic{section}.\arabic{question}}
\renewcommand{\solutiontitle}{\noindent\textbf{Solução:}\enspace}
\renewcommand{\d}{\mathrm{d}}
\renewcommand{\Re}{\mathsf{Re}}
\renewcommand{\Im}{\mathsf{Im}}

\newtheorem{theorem}{Teorema}

\def\C{{\mathbb C}}
\def\N{{\mathbb N}}
\def\R{{\mathbb R}}
\def\Z{{\mathbb Z}}
\def\Q{{\mathbb Q}}
\def\E{{\mathbb E}}
\def\X{{\mathbb X}}
\def\ind{\mathds{1}}
\def\cal{\mathcal}
\def\T{\top}

\footer{}{\thepage}{}

\title{	%EA044 - Planejamento e Análise\\ de Sistemas de Produção - 2S 2018\\
		% {\large \textit{Docente}: Matheus Souza}\\[2mm]
        EE400 -- \textit{Métodos da Engenharia Elétrica} \\[-0mm]
        {\Large Docente: Matheus Souza} \\[+1mm]
        {\Large Solução para problemas selecionados}\\[-0mm]
        -- Lista de Exercícios 9 --
}
\author{Plínio S. Dester\\ (\url{pliniodester@hotmail.com})}

\begin{document}

%% Content goes here
\maketitle

Em caso de dúvidas, sugestões ou correções, não hesite em mandar um e-mail.

% Lista 09: Exercícios 5,6,10,11,14,15,19,20,22,24 + Problemas 3,5;

\setcounter{section}{8}
\section{Exercícios}

\begin{questions}

% 9.4  % % % % % % % % % % % % % % % % % % % % %
\setcounter{question}{3}
\question{
Mostre que as séries $\sum_{n = 1}^\infty \frac{1}{ni}$ e $\sum_{n = 1}^\infty \frac{1}{n+i}$ divergem.
}
%
\begin{solution}
Para o primeiro somatório
    \[
        \left| \sum_{n=1}^\infty \frac{1}{ni} \right| =  \frac{1}{|i|} \left| \sum_{n=1}^\infty \frac{1}{n} \right| = \left| \sum_{n=1}^\infty \frac{1}{n} \right|,
    \]
    que diverge, pois sabemos que a soma harmônica $\sum_n 1/n$ diverge.
    
    Para o segundo somatório, vamos usar o fato que se a parte real da soma diverge, então a soma diverge. Assim,
    \[
        \Re\left\{\sum_{n = 1}^\infty \frac{1}{n+i}\right\} = \sum_{n = 1}^\infty \frac{n}{n^2+1} \ge \sum_{n = 1}^\infty \frac{n}{n^2+n} = \sum_{n = 2}^\infty \frac{1}{n},
    \]
    que diverge. Como a parte real do somatório é maior do que algo que diverge, então o somatório original diverge.
\end{solution}

% 9.5 % % % % % % % % % % % % % % % % % % % % %
% \setcounter{question}{2}
\question{
Determine o raio de convergência e a soma das seguintes séries:
    \begin{multicols}{3}
    \begin{enumerate}[label=(\alph*)]
      \item $\displaystyle \sum_{n = 0}^\infty z^n$;
      \item $\displaystyle \sum_{n = 0}^\infty n z^n$;
      \item $\displaystyle \sum_{n = 0}^\infty n^2 z^n$.
   \end{enumerate}
    \end{multicols}
}
%
\begin{solution}
    Sabemos que o raio de convergência $R$ de uma série de potências $\sum_n a_n z^n$ é dado pelo seguinte limite, se ele existir,
    \[R = \lim_{n\to\infty} \frac{a_{n}}{a_{n+1}}.\]
    Ademais, sabemos que se $|z|<1$ podemos derivar a soma geométrica em relação à $z$,
    \[\sum_{n=0}^\infty z^n = \frac{1}{1-z} 
        ~\Rightarrow~ \sum_{n=0}^\infty n\,z^{n-1} = \frac{1}{(1-z)^2} 
        ~\Rightarrow~ \sum_{n=0}^\infty n(n-1)\,z^{n-2} = \frac{2}{(1-z)^3}.\]
    Assim,
    \begin{enumerate}[label=(\alph*)]
      \item $\displaystyle R = \lim_{n\to\infty} \frac{1}{1} = 1$,\\ $\displaystyle \sum_{n = 0}^\infty z^n = \frac{1}{1-z}, \quad |z|<1$;
      \item $\displaystyle R = \lim_{n\to\infty} \frac{n}{n+1} = 1$,\\ $\displaystyle \sum_{n = 0}^\infty n z^n = z \sum_{n = 0}^\infty n z^{n-1} = \frac{z}{(1-z)^2}, \quad |z|<1$;
      \item $\displaystyle R = \lim_{n\to\infty} \frac{n^2}{(n+1)^2} = 1$,\\ $\displaystyle \sum_{n = 0}^\infty n^2 z^n = z^2\sum_{n = 0}^\infty n(n-1) z^{n-2} + z \sum_{n = 0}^\infty n z^{n-1}  =  \frac{2z^2}{(1-z)^3} + \frac{z}{(1-z)^2} = \frac{z(1+z)}{(1-z)^3}, ~ |z|<1$.
   \end{enumerate}
\end{solution}

% 9.6 % % % % % % % % % % % % % % % % % % % % %
% \setcounter{question}{8}
\question{
Calcule a soma $S$ da série
\[
S = \sum_{n = 0}^\infty \left(\frac{1 + i}{3}\right)^n.
\]
}
%
\begin{solution}
    Note que $|(1+i)/3| = \sqrt{2}/3 < 1$. Portanto, podemos usar a fórmula da soma da progressão geométrica
    \[
        \sum_{n = 0}^\infty \left(\frac{1 + i}{3}\right)^n = \frac{1}{1 - \frac{1 + i}{3}} = \frac{3}{2-i} = \frac{3}{5}(2+i).
    \]
\end{solution}

% 9.7 % % % % % % % % % % % % % % % % % % % % %
% \setcounter{question}{6}
\question{
Considere $z = re^{i\theta}$, com $r \in (0,1)$. Use a soma da série $\sum z^n$ provada no item anterior para mostrar que
\[
\sum_{n = 0}^\infty r^n \cos(n\theta) = \frac{1 - r\cos\theta}{1 - 2r\cos\theta + r^2} \quad \text{e} \quad \sum_{n = 0}^\infty r^n \sin(n\theta) = \frac{r\sin\theta}{1 - 2r\cos\theta + r^2}.
\]
}
%
\begin{solution}
    Como $|re^{i\theta}| = |r| < 1$, então
    \begin{align*}
        \sum_{n=0}^\infty r^n e^{i n\theta} = \frac{1}{1-re^{i\theta}} = \frac{1-r e^{-i\theta}}{1 - r(e^{i\theta}+e^{-i\theta}) + r^2}.
    \end{align*}
    Usando a fórmula de Euler $e^{i\theta} = \cos\theta + i\sin\theta$, temos que
    \begin{align*}
        \sum_{n=0}^\infty r^n(\cos(n\theta) + i\sin(n\theta))
            = \frac{1-r\cos\theta + i \sin\theta}{1 - 2r\cos\theta + r^2}.
    \end{align*}
    Comparando as partes reais e imaginárias da identidade acima termina a prova.
\end{solution}


% 9.9  % % % % % % % % % % % % % % % % % % % % %
\setcounter{question}{8}
\question{
Obtenha a representação em série de Maclaurin para as funções $\cosh$ e $\sinh$. Use este resultado para mostrar que
\[
z\cosh(z^2) = \sum_{n = 0}^\infty \frac{z^{4n+1}}{(2n)!} \quad \forall z \in \C.
\]
}
%
\begin{solution}
    \begin{align*}
        \cosh(z) 
            &= \frac{e^z+e^{-z}}{2} = \frac{1}{2} \sum_{n=0}^\infty \frac{1+(-1)^n}{n!}\,z^n = \sum_{n=0}^\infty \frac{z^{2n}}{(2n)!},\quad z\in\C,\\
        \sinh(z) 
            &= \frac{e^z-e^{-z}}{2} = \frac{1}{2} \sum_{n=0}^\infty \frac{1-(-1)^n}{n!}\,z^n = \sum_{n=0}^\infty \frac{z^{2n+1}}{(2n+1)!},\quad z\in\C.
    \end{align*}
    Logo,
    \begin{align*}
        z\cosh z^2 = z \sum_{n=0}^\infty\frac{(z^2)^{2n}}{(2n)!}
            = \sum_{n=0}^\infty\frac{z^{4n+1}}{(2n)!}, \quad z\in\C.
    \end{align*}
\end{solution}

% 9.10 % % % % % % % % % % % % % % % % % % % % %
% \setcounter{question}{7}
\question{
Mostre que a série de Taylor da exponencial centrada em $z = 1$ é dada por
\[
e^z = e \sum_{n = 0}^\infty \frac{(z-1)^n}{n!}, \quad \forall z \in \C.
\]
}
%
\begin{solution}
Sabemos da expansão de Maclaurin da exponencial que
    \begin{align*}
        e^{z-1} = \sum_{n=0}^\infty \frac{(z-1)^n}{n!}, \quad z\in\C.
    \end{align*}
    Multiplicando por $e$ em ambos lados da equação, temos
    \begin{align*}
        e^z = e \sum_{n=0}^\infty \frac{(z-1)^n}{n!}, \quad z\in\C.
    \end{align*}
\end{solution}

% 9.11 % % % % % % % % % % % % % % % % % % % % %
% \setcounter{question}{7}
\question{
Encontre uma representação em série de Maclaurin para a função $f$ dada por
\[
f(z) = \frac{z}{z^4 + 9}
\]
}
\begin{solution}
    \begin{align*}
        f(z) = \frac{z}{z^4+9} = \frac{z}{9}\, \frac{1}{1-(-z^4/3^2)}
            = \frac{z}{9} \sum_{n=0}^{\infty} \frac{(-1)^n z^{4n}}{3^{2n}}
            = \sum_{n=0}^{\infty} (-1)^n \frac{z^{4n+1}}{3^{2(n+1)}},
            \quad |z| < \sqrt{3}.
    \end{align*}
    Para encontrar a região de convergência, basta lembrar que antes de expandir usando a série geométrica, devemos lembrar que $|-z^4/3^2|<1$.
\end{solution}

% 9.12 % % % % % % % % % % % % % % % % % % % % %
% \setcounter{question}{7}
\question{
Demonstre a representação em série de Taylor
\[
\frac{1}{1 - z} = \sum_{n = 0}^\infty \frac{(z - i)^n}{(1 - i)^{n+1}}, \quad \forall z \; : \; |z - i| < \sqrt{2}.
\]
}
\begin{solution}
    Seja $\omega = z-i$,
    \begin{align*}
        \frac{1}{1 - z} = \frac{1}{1-i-\omega}
            = \frac{1}{1-i}\,\frac{1}{1 - \frac{\omega}{1-i}}
            = \sum_{n=0}^\infty \frac{\omega^n}{(1-i)^{n+1}}, \quad \left|\frac{\omega}{1-i}\right| < 1 \Leftrightarrow |\omega| < \sqrt{2}.
    \end{align*}
    Dessa forma,
    \begin{align*}
        \frac{1}{1-z} = \sum_{n=0}^\infty \frac{(z-i)^n}{(1-i)^{n+1}}, \quad |z-i| < \sqrt{2}.
    \end{align*}
\end{solution}

% 9.14 % % % % % % % % % % % % % % % % % % % % %
\setcounter{question}{13}
\question{
Expanda a função $f$ dada por
\[
f(z) = \frac{1}{(2z - 1)(z - 2)^2}
\]
em série de Laurent em torno dos pontos $z = 1/2$ e $z = 2$.
}
%
\begin{solution}
    Seja $\omega = z-1/2$,
    \begin{align*}
        \frac{1}{(2z - 1)(z - 2)^2} &= \frac{2}{\omega(3-2\omega)^2}
                = \frac{2}{9\omega} \frac{1}{(1-\frac{2}{3}\omega)^2} \\
            &= \frac{2}{9\omega} \sum_{n=1}^\infty n \left(\frac{2}{3}\omega\right)^{n-1}
                = \sum_{n=1}^\infty n \frac{2^{n}}{3^{n+1}}\omega^{n-2}, \quad |\tfrac{2}{3}\omega| < 1 \\
            &= \sum_{n=1}^\infty n \frac{2^{n+1}}{3^{n+2}}(z-1/2)^{n-2},
                \quad |z-1/2| < 3/2.
    \end{align*}
    
    Seja $\omega = z-2$,
    \begin{align*}
        \frac{1}{(2z - 1)(z - 2)^2} &= \frac{1}{\omega^2(3+2\omega)}
                = \frac{1}{3\omega^2}\,\frac{1}{1-(-\tfrac{2}{3}\omega)} \\
            &= \frac{1}{3\omega^2} \sum_{n=0}^\infty \left(\frac{2}{3}\omega\right)^{n}
                = \sum_{n=0}^\infty \frac{2^{n}}{3^{n+1}}\omega^{n-2}, \quad |\tfrac{2}{3}\omega| < 1 \\
            &= \sum_{n=0}^\infty \frac{2^{n}}{3^{n+1}}(z-2)^{n-2},
                \quad |z-2| < 3/2.
    \end{align*}
\end{solution}

% 9.15 % % % % % % % % % % % % % % % % % % % % %
% \setcounter{question}{7}
\question{
Expanda a função $f$ dada por
\[
f(z) = \frac{1}{4z - z^2}
\]
em série de Laurent centrada na origem. Indique a região de convergência da série.
}
%
\begin{solution}
    \begin{align*}
        f(z) = \frac{1}{4z - z^2} = \frac{1}{4z}\,\frac{1}{1 - z/4}
            = \frac{1}{4z} \sum_{n=0}^\infty \left(\frac{z}{4}\right)^n
            = \sum_{n=0}^\infty \frac{z^{n-1}}{4^{n+1}}, \quad |z| < 4.
    \end{align*}
    Alternativamente,
    \begin{align*}
        f(z) = \frac{1}{4z - z^2} = \frac{-1}{z^2}\,\frac{1}{1 - 4/z}
            = \frac{-1}{z^2} \sum_{n=0}^\infty \left(\frac{4}{z}\right)^n
            = \sum_{n=0}^\infty \frac{-4^n}{z^{n+2}}, \quad |z| > 4.
    \end{align*}
\end{solution}

% 9.16 % % % % % % % % % % % % % % % % % % % % %
% \setcounter{question}{15}
\question{
Encontre uma expansão da função real $f$ dada por
\[
f(x) = \frac{\cosh x}{\sinh x} - \frac{1}{x}
\]
para $x$ pequeno.
}
%
\begin{solution}
    \begin{align*}
        f(x) + \frac{1}{x} &= \frac{1+x^2/2!+\cdots}{x+x^3/3!+\cdots}\\
            &= \frac{1}{x}\,\frac{1+x^2/2+\cdots}{1+x^2/6+\cdots}\\
            &= \frac{1}{x} \left( (1+x^2/2 + \cdots) (1 - x^2/6 + \cdots) \right)\\
            &= \frac{1}{x} (1 + x^2/3 + \cdots).
    \end{align*}
    Portanto, para $x\approx 0$ temos que
    \[f(x) \approx \frac{x}{3}.\]

    A expansão completa é difícil de provar e é igual a
    \[
        f(x) = \sum_{n=1}^\infty \frac{2^{2n} B_{2n}}{(2n)!}\,z^{2n-1},
    \]
    onde $B_n$ é o $n$-ésimo número de Bernoulli.
\end{solution}

% 9.18 % % % % % % % % % % % % % % % % % % % % %
\setcounter{question}{17}
\question{
Encontre uma representação em série de Laurent, centrada em $z_0 = 0$, para a função $f$ dada por
\[
f(z) = z^2 \sin\left(\frac{1}{z^2}\right).
\]
Indique a região de convergência da série.
}
%
\begin{solution}
    Usando que
    \[\sin(z) = \sum_{n=0}^{\infty}\frac{(-1)^n z^{2n+1}}{(2n+1)!}, \quad z\in\C\]
    temos que
    \[z^2\sin(1/z^2) = z^2 \sum_{n=0}^{\infty}\frac{(-1)^n (1/z^2)^{2n+1}}{(2n+1)!}
        = \sum_{n=0}^{\infty} \frac{(-1)^n}{z^{4n} (2n+1)!}, \quad |z|>0.\]
\end{solution}

% 9.19 % % % % % % % % % % % % % % % % % % % % %
% \setcounter{question}{17}
\question{
Encontre uma expansão em série de potências para a função $f$ dada por
\[
f(z) = \frac{e^z}{(z+1)^2}
\]
em torno do ponto singular $z_0 = -1$. Indique a região de convergência da série.
}
%
\begin{solution}
Usando a expansão da função exponencial, temos que
    \[e^{z+1} = \sum_{n=0}^{\infty} \frac{(z+1)^n}{n!}, \quad z\in\C.\]
Portanto,
    \[f(z) = \frac{e^{z+1}}{e(z+1)^2} = \frac{1}{e} \sum_{n=0}^{\infty} \frac{(z+1)^{n-2}}{n!}, \quad |z+1| > 0.\]
\end{solution}

% 9.20 % % % % % % % % % % % % % % % % % % % % %
% \setcounter{question}{17}
\question{
Determine uma representação em série de potências para a função $f$ dada por
\[
f(z) = \frac{1}{1+z}
\]
em potências negativas de $z$ e que seja válida quando $1 < |z| < \infty$.
}
%
\begin{solution}
    \begin{align*}
        f(z) &= \frac{1}{z}\,\frac{1}{1+1/z}
                = \frac{1}{z} \sum_{n=0}^\infty \frac{(-1)^n}{z^n}, \quad |1/z|<1,\\
            &= \sum_{n=0}^\infty \frac{(-1)^n}{z^{n+1}}, \quad |z|>1.
    \end{align*}
\end{solution}

% 9.21 % % % % % % % % % % % % % % % % % % % % %
% \setcounter{question}{17}
\question{
Expanda a função $f$ dada por
\[
f(z) = \frac{1}{z^2(1 - z)}
\]
em duas séries de potências de $z$ e especifique as respectivas regiões de convergência.
}
%
\begin{solution}
Uma possível expansão é
    \begin{align*}
        f(z) = \frac{1}{z^2}\,\frac{1}{1 - z} = \frac{1}{z^2} \sum_{n=0}^{\infty} z^n = \sum_{n=0}^{\infty} z^{n-2}, \quad 0<|z|<1.
    \end{align*}
A outra é
    \begin{align*}
        f(z) = \frac{-1}{z^3}\,\frac{1}{1 - 1/z} = \frac{-1}{z^3} \sum_{n=0}^{\infty} \frac{1}{z^n} = \sum_{n=0}^{\infty} \frac{-1}{z^{n+3}}, \quad |z|>1.
    \end{align*}
\end{solution}

% 9.23 % % % % % % % % % % % % % % % % % % % % %
\setcounter{question}{22}
\question{
Mostre que, se $0 < |z-1| < 2$, então
\[
    \frac{z}{(z-1)(z-3)} = -3\sum_{n = 0}^\infty \frac{(z-1)^n}{2^{n+2}} - \frac{1}{2} \frac{1}{z-1}.
\]
}
%
\begin{solution}
    Seja $\omega = z-1$, então
    \begin{align*}
        \frac{z}{(z-1)(z-3)} &= \frac{1+\omega}{\omega (\omega-2)}
            = \left(1 + \frac{1}{\omega}\right) \frac{-1/2}{1-\omega/2}
            = \frac{-1}{2} \left(1 + \frac{1}{\omega}\right) \sum_{n=0}^\infty \left(\frac{\omega}{2}\right)^n, \quad 0<|\omega|<2,\\
        &= \frac{-1}{2} \left(\sum_{n=0}^\infty \left(\frac{\omega}{2}\right)^n + \sum_{n=0}^\infty \frac{\omega^{n-1}}{2^n} \right) = \frac{-1}{2} \left(\frac{1}{\omega} + \sum_{n=0}^\infty \left(\frac{1}{2^n} + \frac{1}{2^{n-1}} \right)\omega^n \right) \\
        &= \frac{-1}{2\omega} - \frac{1}{2} \sum_{n=0}^\infty 3 \frac{\omega^n}{2^{n+1}} = \frac{-1}{2(z-1)} - 3 \sum_{n=0}^\infty \frac{(z-1)^n}{2^{n+2}}, \quad 0<|z-1|<2.
    \end{align*}
\end{solution}

% 9.24 % % % % % % % % % % % % % % % % % % % % %
% \setcounter{question}{22}
\question{
Mostre que a expansão
\[
\frac{e^z}{z(z^2+1)} = \frac{1}{z} + 1 - \frac{1}{2}z - \frac{5}{6}z^2 + \cdots
\]
é válida para todo $z$ tal que $0 < |z| < 1$.
}
%
\begin{solution}
    Sabemos que
    \begin{align*}
        \frac{1}{1+z^2} &= \sum_{n=0}^\infty (-1)^n z^{2n}, \quad |z| < 1,\\
        e^z &= \sum_{n=0}^\infty \frac{z^n}{n!}, \quad z\in\C.
    \end{align*}
    Logo,
    \begin{align*}
        \frac{e^z}{z(z^2+1)} &= \frac{1}{z} \left(1 - z^2 + z^4 - \cdots\right)     \left(1 + z + z^2/2 + z^3/6 + \cdots\right) \\
        &= \frac{1}{z} + 1 - \frac{z}{2} - \frac{5}{6} z^2 + \cdots, \quad 0<|z|<1.
    \end{align*}
\end{solution}

\end{questions}


\setcounter{section}{8}
\section{Problemas}

\begin{questions}

% 9.3  % % % % % % % % % % % % % % % % % % % % %
\setcounter{question}{2}
\question{
{\bf (Números de Euler)} Os {\em números de Euler} são os números $E_n$, $n = 0,1,2,\cdots$, na série de Maclaurin
 \[
 \frac{1}{\cosh z} = \sum_{n = 0}^\infty \frac{E_n}{n!} z^n, \quad \forall z \, : \, |z| < \pi/2.
 \]
 Verifique o raio de convergência desta série. Verifique também que $E_{2n+1} = 0$, para todo $n \in \N$. Finalmente, determine os quatro primeiros números de Euler não-nulos.
}
%
\begin{solution}
    O raio de convergência é dado pela distância entre a origem e o pólo mais próximo, que é em $\pm i\pi/2$, pois $\cosh i\pi/2 = 0$.
    
    Como a função é par, então todos os coeficientes ímpares são nulos, ou seja, $E_{2n+1} = 0$.
    
    Vamos expandir a série do $\cosh$ no denominador e em seguida expandir usando a série geométrica.
    \begin{align*}
        \frac{1}{\cosh z} &= \frac{1}{1 + (z^2/2! + z^4/4! + z^6/6! + \cdots)}\\
            &= 1 - (z^2/2! + z^4/4! + z^6/6!) + (z^2/2! + z^4/4! + z^6/6!)^2 - (z^2/2! + z^4/4! + z^6/6!)^3 + \cdots\\
            &= 1 - \frac{1}{2!}z^2 + \frac{5}{4!}z^4 - \frac{61}{6!}z^6 + \cdots, \quad |z| < \pi/2.
    \end{align*}
    Comparando os coeficientes temos que $E_0 = 1$, $E_2 = -1$, $E_4 = 5$ e $E_6 = -61$.
\end{solution}

% 9.4  % % % % % % % % % % % % % % % % % % % % %
% \setcounter{question}{1}
\question{
{\bf (Transformada ${\cal Z}$)} Suponha que a série
 \[
 \sum_{n = -\infty}^\infty x[n] z^{-n}
 \]
 converge para uma função analítica $X$ em um anel $r_1 < |z| < r_2$. Esta soma, $X(z)$, é chamada de transformada ${\cal Z}$ do sinal $x$. Use a expressão do termo geral da série de Laurent para mostrar que, se a região de convergência da série contiver o círculo unitário $|z| = 1$, então a {\em transformada ${\cal Z}$ inversa} pode ser escrita como
 \[
 x[n] = \frac{1}{2\pi} \int_{-\pi}^\pi X(e^{i\theta})e^{in\theta} \d \theta, \quad n \in \Z.
 \]
}
\begin{solution}
    Seja $C$ o círculo unitário ao redor da origem. Sabemos que
    \begin{align*}
        \oint_C z^{k-1} X(z) \d z = \sum_{n = -\infty}^\infty x[n] \oint_C z^{k-n-1} \d z = 2\pi i \, x[k],
    \end{align*}
    pois a função é analítica em $|z|=1$.
    Logo, seja $z = e^{i\theta}$, $\theta\in(-\pi,\pi)$, temos que
    \begin{align*}
        x[n] = \frac{1}{2\pi i} \int_{-\pi}^{\pi} X(e^{i\theta}) e^{(n-1)i\theta} i e^{i\theta} \d\theta
            = \frac{1}{2\pi} \int_{-\pi}^{\pi} X(e^{i\theta}) e^{ni\theta}\d\theta.
    \end{align*}
\end{solution}

% 9.5  % % % % % % % % % % % % % % % % % % % % %
% \setcounter{question}{10}
\question{
{\bf (Um Sapo Preguiçoso e Assimétrico)} Um sapo pula um metro de $z = 0$ para $z = 1$ em seu primeiro pulo, $1/2$ metro no seu segundo pulo, $1/4$ de metro em seu terceiro pulo e assim sucessivamente; a cada salto, dada a sua condição, o sapo ainda gira de um ângulo $\alpha$ para a esquerda com relação ao salto precedente. Mostre que o sapo sempre irá parar, depois de muito tempo, sobre o círculo $|z - 4/3| = 2/3$, independentemente da escolha de $\alpha$.
}
%
\begin{solution}
    A posição do sapo é dada pelo somatório
    \begin{align}
        \sum_{n=0}^\infty \left(\frac{e^{i\alpha}}{2}\right)^n
            = \frac{1}{1 - e^{i\alpha}/2},
    \end{align}
    pois $|e^{i\alpha}/2| = 1/2 < 1$.
    
    Agora vamos calcular o valor de
    \begin{align*}
        \left| \frac{1}{1 - e^{i\alpha}/2} - \frac{4}{3} \right|^2
            &= 4 \left| \frac{1}{2 - e^{i\alpha}} \frac{2 - e^{-i\alpha}}{2 - e^{-i\alpha}} - \frac{2}{3} \right|^2 \\
            &= 4 \left| \frac{2 - e^{-i\alpha}}{4 - 4 \cos\alpha + 1} - \frac{2}{3} \right|^2 \\
            &= 4 \left| \frac{3(2 - \cos\alpha + i \sin\alpha) - 2 (5-4\cos\alpha) }{3(5 - 4 \cos\alpha)}\right|^2 \\
            &= \frac{4}{9} \left| \frac{5 \cos\alpha - 4 + 3i\sin\alpha}{5 - 4 \cos\alpha} \right|^2 \\
            &= \left( \frac{2}{3} \right)^2 \frac{(5 \cos\alpha - 4)^2 + 9\sin^2\alpha}{(5 - 4 \cos\alpha)^2} \\
            &= \left( \frac{2}{3} \right)^2 \frac{(16+9)\cos^2\alpha -40\cos\alpha + 16 + 9\sin^2\alpha}{ 25 - 40\cos\alpha + 16\cos^2\alpha} \\
            &= \left( \frac{2}{3} \right)^2.
    \end{align*}
    Portanto,
    \[ \left| \sum_{n=0}^\infty \left(\frac{e^{i\alpha}}{2}\right)^n - \frac{4}{3} \right| = \frac{2}{3} \quad \forall \alpha\in\R.\]
\end{solution}

\end{questions}


% \setcounter{section}{4}
% \section{Exercícios Teóricos}
% \begin{questions}

% 5.27 % % % % % % % % % % % % % % % % % % % % %
\setcounter{question}{26}
\question{
Se $X$ é uniformemente distribuída em $(a, b)$, qual variável aleatória que varia linearmente com $X$ é uniformemente distribuída em $(0, 1)$?
}
\begin{solution}
Seja $Y = (X-a)/(b-a)$, então
\begin{align*}
	F_Y(y) &= P(Y\le y) = P((X-a)/(b-a) \le y)\\
    	&= P(X \le (b-a)\,y+a) = F_X((b-a)\,y+a).
\end{align*}
Derivando os dois lados da equação em relação à $y$ obtemos que
\begin{align*}
	f_Y(y) =  (b-a)f_X((b-a)\,y+a) =
    \begin{cases}
    	1, &\text{se }y\in(0,1);\\
        0, &\text{caso contrário.}
    \end{cases}
\end{align*}
Logo, $Y$ é uniformemente distribuída em $(0,1)$.\\[1mm]
\textit{Observação:} outra opção é fazer $Y = (b-X)/(b-a)$.
\end{solution}

% 5.29 % % % % % % % % % % % % % % % % % % % % %
\setcounter{question}{28}
\question{
Seja $X$ uma variável aleatória contínua
com função distribuição cumulativa $F$.
Defina a variável aleatória $Y$ como $Y = F(X)$.
Mostre que $Y$ é uniformemente distribuída em $(0, 1)$.
}
\begin{solution}
	Por simplicidade, vamos supor que $F: \mathbb{R} \to [0,1]$ seja estritamente crescente. Logo, $F$ é inversível e para $y \in (0,1)$,
	\begin{align*}
		F_Y(y) = P(Y\le y) = P(F(X)\le y) = P(X \le F^{-1}(y)) = F(F^{-1}(y)) = y.
	\end{align*}
    Dessa forma, quando $y \in \mathbb{R}$,
    \begin{align*}
    	F_Y(y) =
        \begin{cases}
    	0, &\text{se }y \le 0;\\
        y, &\text{se }0 < y < 1;\\
        1, &\text{se }y \ge 1;\\
    	\end{cases}
    \end{align*}
    o que caracteriza uma distribuição uniforme em $(0,1)$.\\[1mm]
    \textit{Observação:} Isso também acontece quando $F$ não é inversível.
\end{solution}

% 5.30 % % % % % % % % % % % % % % % % % % % % %
%\setcounter{question}{28}
\question{
Suponha que $X$ tenha função densidade
de probabilidade $f_X$. Determine a função
densidade de probabilidade da variável
aleatória $Y$ definida como $Y = aX + b$.
}
\begin{solution}
	Seja $a>0$,
	\begin{align*}
		F_Y(y) = P(Y\le y) = P(aX+b\le y) = P(X\le (y-b)/a) = F_X((y-b)/a).
	\end{align*}
    Derivando ambos lados da equação em relação à $y$ leva à
    \begin{align*}
    	f_Y(y) = \frac{f_X((y-b)/a)}{a}.
    \end{align*}
    Por outro lado, se $a<0$, então
    \begin{align*}
    	F_Y(y) &= P(X\ge (y-b)/a) = 1-P(X<(y-b)/a) = 1-F_X([(y-b)/a]^-)\\
        	&= 1-F_X((y-b)/a) \quad\text{(a variável aleatória é contínua).}
    \end{align*}
    Novamente, derivando ambos lados da equação em relação à $y$ leva à
    \begin{align*}
    	f_Y(y) = \frac{f_X((y-b)/a)}{-a}.
    \end{align*}
    Portanto, quando $a\neq 0$,
    \begin{align*}
    	f_Y(y) = \frac{f_X((y-b)/a)}{|a|}.
    \end{align*}
\end{solution}

\end{questions}
%\newpage

\end{document}