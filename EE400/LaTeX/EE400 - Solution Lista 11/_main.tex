\documentclass[answers, 12pt]{exam}

\usepackage{graphicx,epsfig,psfrag,rotate,xcolor,url}
\usepackage{amsmath,amsthm,amsfonts,amssymb,mathrsfs,latexsym,multicol}
\usepackage{setspace,enumerate,enumitem,ifthen,subfig}
\usepackage{hyperref}
\usepackage{siunitx,tcolorbox}
\usepackage[portuges]{babel}
\usepackage[utf8]{inputenc}
\usepackage[algo2e,english,onelanguage,algoruled]{algorithm2e}
\usepackage[margin=0.7in]{geometry}
\usepackage{calrsfs,empheq}
\renewcommand{\rmdefault}{pplx}
\usepackage{cancel}
\renewcommand{\CancelColor}{\color{red}}
\usepackage{eulervm}
\DeclareMathAlphabet{\mathcal}{OMS}{zplm}{m}{n}
\def\defeq{\mathrel{\mathop:}=}

\usepackage{tikz, circuitikz, pgfplots}
\pgfplotsset{compat=1.15}

\newcommand{\euler}{\mathrm{e}}
\newcommand\diff{\mathrm{d}}
\newcommand{\Log}{\mathop{Log}}
\newcommand{\Arg}{\mathop{Arg}}
\newcommand{\Res}{\mathop{Res}\limits}

\renewcommand{\thequestion}{\arabic{section}.\arabic{question}}
\renewcommand{\solutiontitle}{\noindent\textbf{Solução:}\enspace}
\renewcommand{\d}{\mathrm{d}}
\renewcommand{\Re}{\mathsf{Re}}
\renewcommand{\Im}{\mathsf{Im}}

\newtheorem{theorem}{Teorema}

\def\C{{\mathbb C}}
\def\N{{\mathbb N}}
\def\R{{\mathbb R}}
\def\Z{{\mathbb Z}}
\def\Q{{\mathbb Q}}
\def\E{{\mathbb E}}
\def\X{{\mathbb X}}
\def\ind{\mathds{1}}
\def\cal{\mathcal}
\def\T{\top}

\footer{}{\thepage}{}

\title{	%EA044 - Planejamento e Análise\\ de Sistemas de Produção - 2S 2018\\
		% {\large \textit{Docente}: Matheus Souza}\\[2mm]
        EE400 -- \textit{Métodos da Engenharia Elétrica} \\[-0mm]
        {\Large Docente: Matheus Souza} \\[+1mm]
        {\Large Solução para problemas selecionados}\\[-0mm]
        -- Lista de Exercícios 11 --
}
\author{Plínio S. Dester\\ (\url{pliniodester@hotmail.com})}

\begin{document}

%% Content goes here
\maketitle

Em caso de dúvidas, sugestões ou correções, não hesite em mandar um e-mail.

% Lista 10: Exercícios 1,2,3,5,6,7,12;

\setcounter{section}{10}
\section{Exercícios}

\begin{questions}

% 11.4 % % % % % % % % % % % % % % % % % % % % %
\setcounter{question}{3}
\question{
Determine a imagem do retângulo $a \leq \Re\,z \leq b$ e $c \leq \Im\,z \leq d$ sob a transformação $f(z) = e^z$.
}
%
\begin{solution}
    Vamos supor que $c,d\in(-\pi,\pi]$.
    Seja $z = x+iy$, $x,y\in\R$. Temos que a transformação é dada por $w = e^x e^{iy}$, onde a componente $x$ está relacionada ao módulo ($|w|=e^x$) e a componente $y$ ao ângulo ($\Arg(w) = y$) no plano complexo. Dessa forma, o retângulo se transforma em um setor de anel, ou seja, na região $$\{w\in\C \mid e^a \le |w| \le e^b,~ c \le \Arg(w) \le d \}.$$
\end{solution}

% 11.6 % % % % % % % % % % % % % % % % % % % % %
\setcounter{question}{5}
\question{
Determine os pontos fixos da aplicação dada por $f(z) = \displaystyle \frac{2z - 5}{z + 4}$.
}
%
\begin{solution}
    As soluções da equação $f(z)=z$ são $-1\pm2i$.
\end{solution}

% 11.8 % % % % % % % % % % % % % % % % % % % % %
\setcounter{question}{7}
\question{
Construa uma aplicação conforme que leve o semiplano $\Re\, z \leq 0$ no disco $|z| \leq 1$.
}
%
\begin{solution}
    Usando o método ensinado em aula para construir aplicações bilineares, podemos escolher $z_0 = -1$ e $z_0' = 1$. Dessa forma, a aplicação desejada é dada por
    \[
        w = \frac{z-z_0}{z-z_0'} = \frac{z+1}{z-1}.
    \]
\end{solution}


% 11.9  % % % % % % % % % % % % % % % % % % % % %
% \setcounter{question}{8}
\question{
Construa uma bijeção analítica entre o semi-plano $x + y \geq 0$ e o disco $|z - i| \leq 1$.
}
%
\begin{solution}
    Usando o método ensinado em aula para construir aplicações bilineares, podemos escolher $z_0 = 1+i$ e $z_0' = -1-i$. Dessa forma, a aplicação que leva no círculo unitário centrado na origem é dada por
    \[
        w' = \frac{z-z_0}{z-z_0'} = \frac{z-(1+i)}{z+(1+i)}.
    \]
    Como queremos um círculo unitário centrado em $i$, basta fazer $w=w'+i$, ou seja,
    \[
        w = \frac{z-(1+i))}{z+(1+i)}+i = \frac{(1+i)z-2}{z+(1+i)}.
    \]
\end{solution}

\end{questions}


% \setcounter{section}{9}
% \section{Problemas}
% 
\begin{questions}

% 9.3  % % % % % % % % % % % % % % % % % % % % %
\setcounter{question}{2}
\question{
{\bf (Números de Euler)} Os {\em números de Euler} são os números $E_n$, $n = 0,1,2,\cdots$, na série de Maclaurin
 \[
 \frac{1}{\cosh z} = \sum_{n = 0}^\infty \frac{E_n}{n!} z^n, \quad \forall z \, : \, |z| < \pi/2.
 \]
 Verifique o raio de convergência desta série. Verifique também que $E_{2n+1} = 0$, para todo $n \in \N$. Finalmente, determine os quatro primeiros números de Euler não-nulos.
}
%
\begin{solution}
    O raio de convergência é dado pela distância entre a origem e o pólo mais próximo, que é em $\pm i\pi/2$, pois $\cosh i\pi/2 = 0$.
    
    Como a função é par, então todos os coeficientes ímpares são nulos, ou seja, $E_{2n+1} = 0$.
    
    Vamos expandir a série do $\cosh$ no denominador e em seguida expandir usando a série geométrica.
    \begin{align*}
        \frac{1}{\cosh z} &= \frac{1}{1 + (z^2/2! + z^4/4! + z^6/6! + \cdots)}\\
            &= 1 - (z^2/2! + z^4/4! + z^6/6!) + (z^2/2! + z^4/4! + z^6/6!)^2 - (z^2/2! + z^4/4! + z^6/6!)^3 + \cdots\\
            &= 1 - \frac{1}{2!}z^2 + \frac{5}{4!}z^4 - \frac{61}{6!}z^6 + \cdots, \quad |z| < \pi/2.
    \end{align*}
    Comparando os coeficientes temos que $E_0 = 1$, $E_2 = -1$, $E_4 = 5$ e $E_6 = -61$.
\end{solution}

% 9.4  % % % % % % % % % % % % % % % % % % % % %
% \setcounter{question}{1}
\question{
{\bf (Transformada ${\cal Z}$)} Suponha que a série
 \[
 \sum_{n = -\infty}^\infty x[n] z^{-n}
 \]
 converge para uma função analítica $X$ em um anel $r_1 < |z| < r_2$. Esta soma, $X(z)$, é chamada de transformada ${\cal Z}$ do sinal $x$. Use a expressão do termo geral da série de Laurent para mostrar que, se a região de convergência da série contiver o círculo unitário $|z| = 1$, então a {\em transformada ${\cal Z}$ inversa} pode ser escrita como
 \[
 x[n] = \frac{1}{2\pi} \int_{-\pi}^\pi X(e^{i\theta})e^{in\theta} \d \theta, \quad n \in \Z.
 \]
}
\begin{solution}
    Seja $C$ o círculo unitário ao redor da origem. Sabemos que
    \begin{align*}
        \oint_C z^{k-1} X(z) \d z = \sum_{n = -\infty}^\infty x[n] \oint_C z^{k-n-1} \d z = 2\pi i \, x[k],
    \end{align*}
    pois a função é analítica em $|z|=1$.
    Logo, seja $z = e^{i\theta}$, $\theta\in(-\pi,\pi)$, temos que
    \begin{align*}
        x[n] = \frac{1}{2\pi i} \int_{-\pi}^{\pi} X(e^{i\theta}) e^{(n-1)i\theta} i e^{i\theta} \d\theta
            = \frac{1}{2\pi} \int_{-\pi}^{\pi} X(e^{i\theta}) e^{ni\theta}\d\theta.
    \end{align*}
\end{solution}

% 9.5  % % % % % % % % % % % % % % % % % % % % %
% \setcounter{question}{10}
\question{
{\bf (Um Sapo Preguiçoso e Assimétrico)} Um sapo pula um metro de $z = 0$ para $z = 1$ em seu primeiro pulo, $1/2$ metro no seu segundo pulo, $1/4$ de metro em seu terceiro pulo e assim sucessivamente; a cada salto, dada a sua condição, o sapo ainda gira de um ângulo $\alpha$ para a esquerda com relação ao salto precedente. Mostre que o sapo sempre irá parar, depois de muito tempo, sobre o círculo $|z - 4/3| = 2/3$, independentemente da escolha de $\alpha$.
}
%
\begin{solution}
    A posição do sapo é dada pelo somatório
    \begin{align}
        \sum_{n=0}^\infty \left(\frac{e^{i\alpha}}{2}\right)^n
            = \frac{1}{1 - e^{i\alpha}/2},
    \end{align}
    pois $|e^{i\alpha}/2| = 1/2 < 1$.
    
    Agora vamos calcular o valor de
    \begin{align*}
        \left| \frac{1}{1 - e^{i\alpha}/2} - \frac{4}{3} \right|^2
            &= 4 \left| \frac{1}{2 - e^{i\alpha}} \frac{2 - e^{-i\alpha}}{2 - e^{-i\alpha}} - \frac{2}{3} \right|^2 \\
            &= 4 \left| \frac{2 - e^{-i\alpha}}{4 - 4 \cos\alpha + 1} - \frac{2}{3} \right|^2 \\
            &= 4 \left| \frac{3(2 - \cos\alpha + i \sin\alpha) - 2 (5-4\cos\alpha) }{3(5 - 4 \cos\alpha)}\right|^2 \\
            &= \frac{4}{9} \left| \frac{5 \cos\alpha - 4 + 3i\sin\alpha}{5 - 4 \cos\alpha} \right|^2 \\
            &= \left( \frac{2}{3} \right)^2 \frac{(5 \cos\alpha - 4)^2 + 9\sin^2\alpha}{(5 - 4 \cos\alpha)^2} \\
            &= \left( \frac{2}{3} \right)^2 \frac{(16+9)\cos^2\alpha -40\cos\alpha + 16 + 9\sin^2\alpha}{ 25 - 40\cos\alpha + 16\cos^2\alpha} \\
            &= \left( \frac{2}{3} \right)^2.
    \end{align*}
    Portanto,
    \[ \left| \sum_{n=0}^\infty \left(\frac{e^{i\alpha}}{2}\right)^n - \frac{4}{3} \right| = \frac{2}{3} \quad \forall \alpha\in\R.\]
\end{solution}

\end{questions}


% \setcounter{section}{4}
% \section{Exercícios Teóricos}
% \begin{questions}

% 5.27 % % % % % % % % % % % % % % % % % % % % %
\setcounter{question}{26}
\question{
Se $X$ é uniformemente distribuída em $(a, b)$, qual variável aleatória que varia linearmente com $X$ é uniformemente distribuída em $(0, 1)$?
}
\begin{solution}
Seja $Y = (X-a)/(b-a)$, então
\begin{align*}
	F_Y(y) &= P(Y\le y) = P((X-a)/(b-a) \le y)\\
    	&= P(X \le (b-a)\,y+a) = F_X((b-a)\,y+a).
\end{align*}
Derivando os dois lados da equação em relação à $y$ obtemos que
\begin{align*}
	f_Y(y) =  (b-a)f_X((b-a)\,y+a) =
    \begin{cases}
    	1, &\text{se }y\in(0,1);\\
        0, &\text{caso contrário.}
    \end{cases}
\end{align*}
Logo, $Y$ é uniformemente distribuída em $(0,1)$.\\[1mm]
\textit{Observação:} outra opção é fazer $Y = (b-X)/(b-a)$.
\end{solution}

% 5.29 % % % % % % % % % % % % % % % % % % % % %
\setcounter{question}{28}
\question{
Seja $X$ uma variável aleatória contínua
com função distribuição cumulativa $F$.
Defina a variável aleatória $Y$ como $Y = F(X)$.
Mostre que $Y$ é uniformemente distribuída em $(0, 1)$.
}
\begin{solution}
	Por simplicidade, vamos supor que $F: \mathbb{R} \to [0,1]$ seja estritamente crescente. Logo, $F$ é inversível e para $y \in (0,1)$,
	\begin{align*}
		F_Y(y) = P(Y\le y) = P(F(X)\le y) = P(X \le F^{-1}(y)) = F(F^{-1}(y)) = y.
	\end{align*}
    Dessa forma, quando $y \in \mathbb{R}$,
    \begin{align*}
    	F_Y(y) =
        \begin{cases}
    	0, &\text{se }y \le 0;\\
        y, &\text{se }0 < y < 1;\\
        1, &\text{se }y \ge 1;\\
    	\end{cases}
    \end{align*}
    o que caracteriza uma distribuição uniforme em $(0,1)$.\\[1mm]
    \textit{Observação:} Isso também acontece quando $F$ não é inversível.
\end{solution}

% 5.30 % % % % % % % % % % % % % % % % % % % % %
%\setcounter{question}{28}
\question{
Suponha que $X$ tenha função densidade
de probabilidade $f_X$. Determine a função
densidade de probabilidade da variável
aleatória $Y$ definida como $Y = aX + b$.
}
\begin{solution}
	Seja $a>0$,
	\begin{align*}
		F_Y(y) = P(Y\le y) = P(aX+b\le y) = P(X\le (y-b)/a) = F_X((y-b)/a).
	\end{align*}
    Derivando ambos lados da equação em relação à $y$ leva à
    \begin{align*}
    	f_Y(y) = \frac{f_X((y-b)/a)}{a}.
    \end{align*}
    Por outro lado, se $a<0$, então
    \begin{align*}
    	F_Y(y) &= P(X\ge (y-b)/a) = 1-P(X<(y-b)/a) = 1-F_X([(y-b)/a]^-)\\
        	&= 1-F_X((y-b)/a) \quad\text{(a variável aleatória é contínua).}
    \end{align*}
    Novamente, derivando ambos lados da equação em relação à $y$ leva à
    \begin{align*}
    	f_Y(y) = \frac{f_X((y-b)/a)}{-a}.
    \end{align*}
    Portanto, quando $a\neq 0$,
    \begin{align*}
    	f_Y(y) = \frac{f_X((y-b)/a)}{|a|}.
    \end{align*}
\end{solution}

\end{questions}
%\newpage

\end{document}