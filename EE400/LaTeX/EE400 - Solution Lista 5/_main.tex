\documentclass[answers, 12pt]{exam}

\usepackage{graphicx,epsfig,psfrag,rotate,xcolor,url}
\usepackage{amsmath,amsthm,amsfonts,amssymb,mathrsfs,latexsym,multicol}
\usepackage{setspace,enumerate,enumitem,ifthen,subfig}
\usepackage{hyperref}
\usepackage{siunitx,tcolorbox}
\usepackage[portuges]{babel}
\usepackage[utf8]{inputenc}
\usepackage[algo2e,english,onelanguage,algoruled]{algorithm2e}
\usepackage[margin=0.7in]{geometry}
\usepackage{calrsfs,empheq}
\renewcommand{\rmdefault}{pplx}
\usepackage{cancel}
\renewcommand{\CancelColor}{\color{red}}
\usepackage{eulervm}
\DeclareMathAlphabet{\mathcal}{OMS}{zplm}{m}{n}
\def\defeq{\mathrel{\mathop:}=}

\usepackage{tikz, circuitikz, pgfplots}
\pgfplotsset{compat=1.15}

\newcommand{\euler}{\mathrm{e}}
\newcommand\diff{\mathrm{d}}

\renewcommand{\thequestion}{\arabic{section}.\arabic{question}}
\renewcommand{\solutiontitle}{\noindent\textbf{Solução:}\enspace}
\renewcommand{\d}{\mathrm{d}}
\renewcommand{\Re}{\mathsf{Re}}
\renewcommand{\Im}{\mathsf{Im}}

\newtheorem{theorem}{Teorema}

\def\C{{\mathbb C}}
\def\N{{\mathbb N}}
\def\R{{\mathbb R}}
\def\Z{{\mathbb Z}}
\def\Q{{\mathbb Q}}
\def\E{{\mathbb E}}
\def\X{{\mathbb X}}
\def\ind{\mathds{1}}
\def\cal{\mathcal}
\def\T{\top}

\footer{}{\thepage}{}

\title{	%EA044 - Planejamento e Análise\\ de Sistemas de Produção - 2S 2018\\
		% {\large \textit{Docente}: Matheus Souza}\\[2mm]
        EE400 -- \textit{Métodos da Engenharia Elétrica} \\[-0mm]
        {\Large Docente: Matheus Souza} \\[+1mm]
        {\Large Solução para problemas selecionados}\\[-0mm]
        -- Lista de Exercícios 5 --
}
\author{Plínio S. Dester\\ (\url{pliniodester@hotmail.com})}

\begin{document}

%% Content goes here
\maketitle

Em caso de dúvidas, sugestões ou correções, não hesite em mandar um e-mail.

% Lista 05: Exercícios 1,2,3,5,6,8,9,11,12,13,14,15,16,17,19,20,21,22,23 + Problemas 1,2,3,4;

\setcounter{section}{4}
\section{Exercícios}

\begin{questions}

% 5.1  % % % % % % % % % % % % % % % % % % % % %
% \setcounter{question}{1}
\question{
Reduza os seguintes números complexos à forma $x + iy$.
\begin{multicols}{3}
\begin{enumerate}[label=(\alph*)]
  \item $z_1 = (2-3i)(1+5i)$;
  \item $z_2 = (1+2i)(2 + i)$;
  \item $z_3 = (4 - 3i)(5 - i)(1 + i)$;
  \item $z_4 = (3+2i)^2$;
  \item $z_5 = (1 + i)^3$;
  \item $\displaystyle  z_6 = \frac{1+2i}{3-i}$;
  \item $\displaystyle  z_7 = \frac{1 + i}{(1 - i)^2}$.
\end{enumerate}
\end{multicols}
}
\begin{solution}
    \begin{multicols}{2}
    \begin{enumerate}[label=(\alph*)]
      \item $z_1 = 17+7i$;
      \item $z_2 = 5i$;
      \item $z_3 = 36-2i$;
      \item $z_4 = 5+12i$;
      \item $z_5 = -2+2i$;
      \item $z_6 = \dfrac{(1+2i)(3+i)}{(3-i)(3+i)} = \dfrac{1}{10}(1+7i)$;
      \item $z_7 = \dfrac{(1+i)^3}{(1+1)^2} = \dfrac{1}{2}(-1+i)$.
    \end{enumerate}
    \end{multicols}
\end{solution}

% 5.2 % % % % % % % % % % % % % % % % % % % % %
% \setcounter{question}{2}
\question{
Determine o valor de $i^n$ para $n$ inteiro. Use este resultado para reduzir os seguintes números à forma algébrica $z = x + iy$.
\begin{multicols}{2}
\begin{enumerate}[label=(\alph*)]
  \item $\displaystyle  z_1 = \frac{(1 + i)^{80} - (1 + i)^{82}}{i^{96}}$;
  \item $z_2 = (1+i)^{12} - (1-i)^{12}$;
  \item $\displaystyle  z_3 = \frac{i^3 - i^2 + i^{17} - i^{35}}{i^{16} - i^{13} + i^{30}}$;
  \item $\displaystyle z_4 = \frac{i^{11} + 2i^{13}}{i^{18} - i^{37}}$.
\end{enumerate}
\end{multicols}
}
\begin{solution}
Como $i^4 = 1$, então $i^n = i^{4q+r} = i^r$, onde $q$ é o quociente e $r$ é o resto da divisão de $n$ por 4. Dessa forma,
\begin{enumerate}[label=(\alph*)]
  \item $ z_1 = (1 + i)^{2\cdot 40} (1 - (1 + i)^{2}) = (2i)^{40} (1-2i) = 2^{40}(1-2i)$;
  \item $z_2 = (1+i)^{12} - (1-i)^{12} = (2i)^6 - (-2i)^6 = 0$;
  \item $\displaystyle  z_3 = \frac{-i + 1 + i + i}{1 - i -1} = -1+i$;
  \item $\displaystyle z_4 = \frac{-i + 2i}{-1 - i} = \frac{-i}{1 + i}\frac{1-i}{1-i} = \frac{1}{2}(-1-i)$.
\end{enumerate}
\end{solution}

% 5.3 % % % % % % % % % % % % % % % % % % % % %
% \setcounter{question}{6}
\question{
Determine números complexos $z_1$ e $z_2$ tais que $z_1 + i z_2 = i$ e $i z_1 + z_2 = 2i - 1$.
}

\begin{solution}
Multiplicando a segunda equação por $i$ e somando ambas equacões, temos que
    \[
        2iz_2 = -2 ~\Rightarrow \boxed{z_2 = i}.
    \]
Em seguida, usando a primeira equação temos que $\boxed{z_1 = 1+i}$.
\end{solution}

% 5.4 % % % % % % % % % % % % % % % % % % % % %
% \setcounter{question}{8}
\question{
Quais as condições sobre os números complexos $z_1$ e $z_2$ para que $z_1 z_2 \in \R$? Quais as condições sobre $z_1$ e $z_2$, com $z_2 \neq 0$, para que $z_1/z_2$ seja real?
}

\begin{solution}
    Seja $z_1 = x_1 + i y_1$ e $z_2 = x_2 + i y_2$.
    %
    Sabemos que 
    \[
        z_1 z_2 \in \R 
            ~\Leftrightarrow~ \Im[z_1 z_2] = 0
            ~\Leftrightarrow~ x_1y_2+x_2y_1 = 0,
    \]
    %
    \[
        z_1/z_2 \in \R 
            ~\Leftrightarrow~ z_1 \bar{z}_2/|z_2|^2 \in \R
            ~\Leftrightarrow~ z_1 \bar{z}_2 \in \R
            ~\Leftrightarrow~ -x_1y_2+x_2y_1 = 0.
    \]
\end{solution}

% 5.5 % % % % % % % % % % % % % % % % % % % % %
% \setcounter{question}{5}
\question{
Quais são as condições sobre $a,b \in \R$ para que o número $(a + bi)^4$ seja estritamente negativo?
}

\begin{solution}
    Se $(a + bi)^4<0$, então é fácil ver que $a\neq 0$ e $b\neq 0$.
    %
    Expandindo
    \[
        (a+bi)^4 = \left[(a^2-b^2) + 2abi \right]^2
            = \left[ (a^2-b^2)^2-4a^2b^2 \right] + \left[4(a^2-b^2)ab\right]i.
    \]
    Note que a parte imaginária se anula somente se $a=\pm b$. Quando isso ocorre \[(a + bi)^4 = -4a^2b^2 < 0.\]
\end{solution}


% 5.6  % % % % % % % % % % % % % % % % % % % % %
% \setcounter{question}{7}
\question{
Sejam $u,v \in \C$ dois números tais que $u^2 - v^2 = 6$ e $\bar u + \bar v = 1-i$. Calcule $u - v$.
}

\begin{solution}
Sabemos que $(u+v)(u-v) = u^2-v^2 = 6$ e que $u+v = \overline{\bar u + \bar v} = 1+i$. Assim,
\begin{align*}
    u-v &= \frac{6}{1+i} = 3(1-i).
\end{align*}
\end{solution}

% 5.7 % % % % % % % % % % % % % % % % % % % % %
% \setcounter{question}{5}
\question{
Variando o número inteiro $n$, quais são os possíveis valores para o quociente $\left(\frac{1+i}{1-i}\right)^n$?
}

\begin{solution}
    \[\left(\frac{1+i}{1-i}\right)^n = \left(\frac{(1+i)^2}{(1-i)(1+i)}\right)^n = i^n \in \{\pm 1,\pm i\}, \quad \text{vide Exercício 5.2.} \]
\end{solution}

% 5.8 % % % % % % % % % % % % % % % % % % % % %
% \setcounter{question}{8}
\question{
Seja $z = x+iy \neq 0$. Determine a condição para que $z + z^{-1}$ seja real.
}

\begin{solution}
    Sabemos que
    \begin{align*}
        z+z^{-1} = \frac{|z|^2 z + \bar{z}}{|z|^2} \in \R
        ~\Leftrightarrow~ |z|^2 z + \bar{z} \in \R
        ~\Leftrightarrow~ \Im[|z|^2 z + \bar{z}] = 0
        ~\Leftrightarrow~ y\,(|z|^2-1) = 0.
    \end{align*}
    Dessa forma, $z + z^{-1} \in \R$ se, e somente se, $y=0$ ou $|z|=1$.
\end{solution}

% 5.9 % % % % % % % % % % % % % % % % % % % % %
% \setcounter{question}{8}
\question{
{\bf (IME)} Sejam $z = a + bi$ e $w = 47 + ci$ numéros complexos tais que $z^3 + w = 0$. Determine os valores de $a$, $b$ e $c$ sabendo que estes números são inteiros positivos.
}
\begin{solution}
    Expandindo $(a+bi)^3$ encontramos que
    \[z^3 = -a(3b^2-a^2) - b(b^2-3a^2)i.\]
    Como $w=-z^3$, comparando as partes reais obtemos que $47 = a(3b^2-a^2)$.
    %
    Note que $47$ é primo, ou seja, \{$a=47$ e $3b^2-a^2=1$\} ou \{$a=1$ e $3b^2-a^2=47$\}. Excluímos a primeira solução, pois $1+a^2$ não é divisível por 3.
    %
    Usando a segunda solução temos que $a=1$ e $b=4$.
    %
    Enfim, comparando as partes imaginárias chega-se em $c = b(b^2-3a^2) = 52$.
\end{solution}

% 5.10 % % % % % % % % % % % % % % % % % % % % %
% \setcounter{question}{8}
\question{
Se $z_1$ e $z_2$ forem dois números complexos tais que $z_1 + z_2$ e $z_1 z_2$ forem ambos reais, o que podemos afirmar sobre $z_1$ e $z_2$?
}

\begin{solution}
    Como $z_1+z_2\in\R$, então $\Im\,z_1 = -\Im\,z_2$. Dessa forma, sejam $x_1,x_2,y\in\R$, então $z_1,z_2$ podem ser escritos como $z_1 = x_1+yi$ e $z_2=x_2-yi$. Como $z_1 z_2\in\R$ também, então
    \[
        \Im[z_1 z_2] = 0
        ~\Leftrightarrow~ x_1 y-x_2 y = 0
        ~\Leftrightarrow~ y\,(x_1 - x_2) = 0.
    \]
    Portanto, $y=0$ ou $x_1=x_2$, isto é, $z_1,z_2\in\R$ ou $z_1 = \overline{z_2}$.
\end{solution}

% 5.11 % % % % % % % % % % % % % % % % % % % % %
% \setcounter{question}{8}
\question{
Determine $x \in \R$ para que o número $\displaystyle z = \frac{2 - ix}{1 + i2x}$ seja imaginário puro.
}

\begin{solution}
    \[
        \Re\left[\frac{2 - ix}{1 + i2x}\right] = 0
        ~\Leftrightarrow~ \Re\left[\frac{(2 - ix)(1-i2x)}{1 + 4x^2}\right] = 0
        ~\Leftrightarrow~ 2(1-x^2) = 0.
    \]
    Assim, $x\in\{1,-1\}$ para que $z$ seja imaginário puro.
\end{solution}

% 5.14 % % % % % % % % % % % % % % % % % % % % %
\setcounter{question}{13}
\question{
Determine o menor valor de $n \in \mathbb{N}$ para o qual $(\sqrt{3} + i)^n$ é:
\begin{multicols}{3}
\begin{enumerate}[label=(\alph*)]
  \item real e positivo;
  \item real e negativo;
  \item imaginário puro.
\end{enumerate}
\end{multicols}
}
\begin{solution}
Vamos escrever a base em forma polar, ou seja, $\sqrt{3}+i = 2\,\euler^{i \pi/6}$.

Logo, $(\sqrt{3}+i)^n = 2^n\,\euler^{i n \pi/6}$ e assim, $\arg (\sqrt{3}+i)^n = n \pi/6$.
\begin{enumerate}[label=(\alph*)]
  \item Um número complexo é real e positivo se o argumento for da forma $2k\pi$, $k\in\Z$, ou seja, se $n = 12k$, $k\in\N$.
  \item Um número complexo é real e negativo se o argumento for da forma $\pi + 2k\pi$, $k\in\Z$, ou seja, se $n = 6+12k$, $k\in\N$.
  \item Um número complexo é imaginário puro se o argumento for da forma $\pi/2 + k\pi$, $k\in\Z$, ou seja, se $n = 3+6k$, $k\in\N$.
\end{enumerate}
\end{solution}

% 5.15 % % % % % % % % % % % % % % % % % % % % %
% \setcounter{question}{13}
\question{
Para os itens a seguir, considere o número complexo $z = \sigma + i \omega$.
\begin{enumerate}[label=(\alph*)]
  \item Para $\sigma = 1$, determine $\omega > 0$ para que $\arg\left(z^{-3}\right) = -\pi$.
  \item Para $\omega = 1$, determine $\sigma < 0$ para que $\left|z^{-3}\right| = 1/8$.
\end{enumerate}
}
\begin{solution}
Vamos usar a forma polar de $z = r \euler^{i\theta}$. Assim,
\begin{enumerate}[label=(\alph*)]
  \item 
  $\arg\left(z^{-3}\right) = \arg\left(r^{-3} \euler^{-3i\theta}\right) = -3\theta$. Logo, $\theta = \pi/3$.
  
  Para $\sigma = 1, \omega>0$, temos que $\omega = \sqrt{3}$, de forma que $\tan{\theta} = \omega/\sigma$.
  
  \item $\left|z^{-3}\right| = \left|r^{-3} \euler^{-3i\theta}\right| = r^{-3}$. Logo, $r = 2$.
  
  Para $\omega = 1,\sigma < 0$, temos que $\sigma = -\sqrt{3}$ para que $r = \sqrt{\sigma^2+\omega^2}$.
\end{enumerate}
\end{solution}

% 5.16 % % % % % % % % % % % % % % % % % % % % %
% \setcounter{question}{13}
\question{
{\bf (IME - modificada)} Considere o número complexo $\displaystyle z = \frac{a}{ib(1 + ib)^2}$, com $a,b$ reais positivos. Sabendo que o módulo e o argumento de $z$ valem, respectivamente, $1$ e $-\pi$, determine $a$.
}
\begin{solution}
Sabemos que
\[\arg z = \arg a - \arg ib - \arg (1+ib)^3 = 0 - \frac{\pi}{2} - 3\arctan b.\]
Como $\arg z = -\pi$, então $b = \tan \pi/6 = \sqrt{3}/3$.

Ademais,
\[|z| = \frac{|a|}{|ib|\,|1 + ib|^2} = \frac{a}{b(1+b^2)^{3/2}}.\]
Como $|z|=1$, então \[a = b(1+b^2)^{3/2} = \frac{4^{3/2}}{\sqrt{3}\,3^{3/2}} = \frac{2^3}{3^2} = \frac{8}{9}.\]
\end{solution}

% 5.21 % % % % % % % % % % % % % % % % % % % % %
\setcounter{question}{20}
\question{
Um hexágono regular, inscrito em uma circunferência de centro na origem, tem como um de seus vértices o ponto $(0,2)$. Determine os outros cinco vértices do hexágono.
}
\begin{solution}
    No plano complexo as coordenadas dos vértices do hexágono regular são justamente as soluções da equação $z^6 = (2i)^6$. Igualando os módulos e os argumentos tiramos que $|z| = 2$ e $6 \arg z = 6 \frac{\pi}{2} + 2k\pi$, $k\in\Z$, ou seja, $\arg z = \pi/2 + k \pi/3 $, $k\in\Z$. Portanto, \vspace{-10mm}
    \begin{multicols}{2}
        \begin{align*}
            z_0 &= 2 \euler^{i \pi/2} = (0,2), \\
            z_1 &= 2 \euler^{i (\pi/2+\pi/3)} = (-\sqrt{3},1), \\
            z_2 &= 2 \euler^{i (\pi/2+2\pi/3)} = (-\sqrt{3},-1),
        \end{align*}
        
        \begin{align*}
            z_3 &= 2 \euler^{i (\pi/2+3\pi/3)} = (0,-2), \\
            z_4 &= 2 \euler^{i (\pi/2+4\pi/3)} = (\sqrt{3},-1), \\
            z_5 &= 2 \euler^{i (\pi/2+5\pi/3)} = (\sqrt{3},1).
        \end{align*}
    \end{multicols}
    
\end{solution}

% 5.22 % % % % % % % % % % % % % % % % % % % % %
% \setcounter{question}{20}
\question{
{\bf (IME - modificada)} Seja $z$ um número complexo tal que $\frac{2z}{\bar z i}$ tenha argumento $\frac{3\pi}{2}$ e $\log_3(2z + 2\bar z + 1) = 2$. Determine o número complexo $z$.
}
\begin{solution}
    Sabemos que $3\pi/2 = \arg(2z/\bar z i) = \arg(z)-\arg(\bar z) - \pi/2$, ou seja, $\arg(z) = \arg(\bar z)$. Portanto, $z$ é real. Assim, $z=7/4$ para que $\log_3(4z+1)=2$.
\end{solution}

% 5.23 % % % % % % % % % % % % % % % % % % % % %
% \setcounter{question}{20}
\question{
Resolva as seguintes equações, para $z \in \C$:

\begin{multicols}{3}
\begin{enumerate}[label=(\alph*)]
  \item $\displaystyle \frac{z}{1-i} + \frac{z-1}{1+i} = \frac{5}{2} + i \frac{5}{2}$;
  \item $\bar z = -2zi$;
  \item $z\bar z + (z - \bar z) = 13 + i6$;
  \item $z^3 = \bar z$;
  \item $z^2 = i$;
  \item $z^2 + |z| = 0$;
  \item $z^6 + 8 = 0$;
  \item $z^4 + i = 0$;
  \item $z^3 - 27 = 0$;
  \item $z^8 - 17z^4 + 16 = 0$;
  \item $z^4 - 2z^2 + 2 = 0$.
\end{enumerate}
\end{multicols}
}
\begin{solution}
\begin{enumerate}[label=(\alph*)]
  \item A equação é satisfeita para $z = 3 + 2i$;
  \item A equação é satisfeita para $z = 0$;
  \item A equação é satisfeita para $z = 2 + 3i$;
  \item A equação é satisfeita para todo $z\in\{0,\pm 1,\pm i\}$;
  \item A equação é satisfeita para todo $z\in\{\pm(1+i)/\sqrt{2}\}$;
  \item A equação é satisfeita para todo $z\in\{0,\pm i\}$;
  \item A equação é satisfeita para todo $z \in \{ \pm i \sqrt{2}, (\sqrt{3}\pm i)/\sqrt{2}, (-\sqrt{3}\pm i)/\sqrt{2} \} $;
  \item A equação é satisfeita para $z = \euler^{i(\pi/8+k\pi/2)}, k\in\{0,1,2,3\}$;
  \item A equação é satisfeita para todo $z\in\{ 3, (-3 \pm 3\sqrt{3} i)/2 \}$;
  \item A equação é satisfeita para todo $z\in\{ \pm 1, \pm 2, \pm i, \pm 2i \}$;
  \item A equação é satisfeita para $z = \sqrt[4]{2}\, \euler^{i(\pi/8 + k\pi/2)}, k\in\{0,1,2,3\}$.
\end{enumerate}
\end{solution}

\end{questions}


\setcounter{section}{4}
\section{Problemas}

\begin{questions}

% 5.1  % % % % % % % % % % % % % % % % % % % % %
%\setcounter{question}{0}
\question{Seja $X$ uma variável aleatória com função densidade de probabilidade
\begin{equation*}
	f(x) = 
    \begin{cases}
		c\,(1-x^2) 	&-1<x<1 \\
        0			&\text{caso contrário}
	\end{cases}
\end{equation*}
\begin{parts}
	\part Qual é o valor de $c$?
	\part Qual é a função distribuição cumulativa de $X$?
\end{parts}
}
\begin{solution}
\begin{parts}
	\part Sabemos que $\int_{-\infty}^\infty f(x)\,\diff x = 1$, logo
    \begin{align*}
    		&\int_{-1}^1 c\,(1-x^2)\,\diff x = 1\\
        \Rightarrow\, & c\,\left.\left( x-\frac{x^3}{3} \right)\right\rvert_{-1}^1
        	= 1\\
        \Rightarrow\, & c = \tfrac{3}{4}.
    \end{align*}
    
	\part Por definição, $F(x) = \int_{-\infty}^x f(x)\,\diff x$. Assim, se $x\in[-1,1]$, então
    \begin{align*}
    	F(x) &= \int_{-1}^x \tfrac{3}{4}(1-{x'}^2)\,\diff x'\\
        	&= \frac{3}{4}\left.\left( x'-\frac{x'^3}{3} \right)\right\rvert_{-1}^x\\
            &= \frac{2 + 3x - x^3}{4}.
    \end{align*}
    Logo,
    \begin{align*}
    	F(x) = 
        \begin{cases}
        	0,						& x \le -1;\\
            \frac{2 + 3x - x^3}{4},	& -1 < x \le 1;\\
            1,						& x > 1.
        \end{cases}
    \end{align*}
\end{parts}
\end{solution}

% 5.2  % % % % % % % % % % % % % % % % % % % % %
%\setcounter{question}{0}
\question{Um sistema formado por uma peça original mais uma sobressalente pode funcionar por uma quantidade de tempo aleatória $X$. Se a densidade de $X$ é dada, em
unidades de meses, por
\begin{equation*}
	f(x) = 
    \begin{cases}
		C\,x\euler^{-x/2} 	&x>0 \\
        0			&x\le 0
	\end{cases}
\end{equation*}
qual é a probabilidade de que o sistema
funcione por pelo menos 5 meses?
}
\begin{solution}
	Note que $X$ segue uma distribuição Gamma de parâmetros $\alpha = 2$ e $\lambda = 1/2$. Sabemos que $C = \lambda^\alpha/\Gamma(\alpha) = (1/2)^2/1! = 1/4$.\\    
    A probabilidade que o sistema funcione por pelo menos 5 meses é dada por
    \begin{align*}
    	\int_5^\infty \tfrac{1}{4}\,x\euler^{-x/2}\,\diff x
        = \left.\left(-\tfrac{1}{2}(x+2)\,\euler^{-x/2}\right)\right\rvert_5^\infty
        = \tfrac{7}{2}\,\euler^{-5/2}
        \approx \text{0,287}.
    \end{align*}
\end{solution}

% 5.3  % % % % % % % % % % % % % % % % % % % % %
%\setcounter{question}{0}
\question{Considere a função
\begin{equation*}
	f(x) = 
    \begin{cases}
		C\,(2x-x^3) &0<x<\tfrac{5}{2} \\
        0			&\text{caso contrário}
	\end{cases}
\end{equation*}
Poderia $f$ ser uma função densidade de
probabilidade? Caso positivo, determine
$C$.Repita considerando que a função $f(x)$
seja dada por
\begin{equation*}
	f(x) = 
    \begin{cases}
		C\,(2x-x^2) &0<x<\tfrac{5}{2} \\
        0			&\text{caso contrário}
	\end{cases}
\end{equation*}
}
\begin{solution}
	Em ambos casos $f$ não pode ser uma função densidade de probabilidade, pois a integral de $f$ em qualquer intervalo deve ser não-negativa, uma vez que representa uma probabilidade. Note que os pontos que estão em uma vizinhança de $x=1$ tem sinal oposto aqueles que estão em uma vizinhança de $x=5/2$. Logo, independente do sinal escolhido para $C$, em algum dos dois casos a integral será negativa e, portanto, $f$ não pode ser uma função densidade de probabilidade.
\end{solution}

% 5.5  % % % % % % % % % % % % % % % % % % % % %
\setcounter{question}{4}
\question{Um posto de gasolina é abastecido com
gasolina uma vez por semana. Se o volume semanal de vendas em milhares de litros é uma variável aleatória com função
densidade de probabilidade
\begin{equation*}
	f(x) = 
    \begin{cases}
		5\,(1-x)^4 &0<x<1 \\
        0			&\text{caso contrário}
	\end{cases}
\end{equation*}
qual deve ser a capacidade do tanque
para que a probabilidade do fornecimento não ser suficiente em uma dada semana seja de 0,01?
}
\begin{solution}
	Queremos descobrir $x_0$ para o qual as vendas serem superiores à $x_0$ tenha probabilidade 0,01, ou seja,
    \begin{align*}
    	\int_{x_0}^1 5\,(1-x)^4\,\diff x = \text{0,01}
        ~\Rightarrow~ (1-x_0)^5 = \text{0,01}
        ~\Rightarrow~ x_0 = \text{0,602}.
    \end{align*}
    Logo, a capacidade do tanque deve ser de 602 litros.
\end{solution}

% 5.7  % % % % % % % % % % % % % % % % % % % % %
\setcounter{question}{6}
\question{A função densidade de $X$ é dada por
\begin{equation*}
	f(x) = 
    \begin{cases}
		a+bx^2 &0\le x\le 1 \\
        0			&\text{caso contrário}
	\end{cases}
\end{equation*}
Se $E[X]=3/5$,determine $a$ e $b$.
}
\begin{solution}
\begin{align*}
	\int_0^1(a+bx^2)\,\diff x = 1 ~\Rightarrow~ a+b/3 = 1.
\end{align*}
\begin{align*}
	E[X] = \int_0^1x(a+bx^2)\,\diff x
    ~\Rightarrow~ a/2+b/4 = 3/5.
\end{align*}
Resolvendo essas duas equações para $a$ e $b$ obtemos $a=3/5$ e $b=6/5$.
\end{solution}

% 5.10 % % % % % % % % % % % % % % % % % % % % %
\setcounter{question}{9}
\question{Trens em direção ao destino $A$ chegam
na estação em intervalos de 15 minutos a
partir das 7:00 da manhã, enquanto trens
em direção ao destino $B$ chegam à estação em intervalos de 15 minutos começando as 7:05 da manhã.
\begin{parts}
	\part Se certo passageiro chega à estação em um horário uniformemente distribuído entre 7:00 e 8:00 da manhã e
pega o primeiro trem que chega, em
que proporção de tempo ele vai para
o destino $A$?
	\part E se o passageiro chegar em um horário uniformemente distribuído entre 7:10 e 8:10 da manhã?
\end{parts}
}
\begin{solution}
\begin{parts}
	\part Note que a pessoa pegará o trem $A$ se ela chegar nos intervalos de tempo {7:05-7:15}, {7:20-7:30}, {7:35-7:45} ou {7:50-8:00}. O que totaliza 40 minutos. Como o horário que a pessoa chega é distribuído uniformemente em 60 minutos, então a proporção de vezes que ela pega o trem $A$ é $\frac{40}{60} = \frac{2}{3}$.
    
	\part Nesse novo cenário, os intervalos são {7:10-7:15}, {7:20-7:30}, {7:35-7:45}, {7:50-8:00} ou {8:05-8:10}. O que totaliza 40 minutos também. Logo, a proporção de vezes que ela pega o trem $A$ continua sendo $\frac{2}{3}$.
\end{parts}
\end{solution}

% 5.11 % % % % % % % % % % % % % % % % % % % % %
%\setcounter{question}{9}
\question{Um ponto é escolhido aleatoriamente em
um segmento de reta de comprimento $L$.
Interprete este enunciado e determine a
probabilidade de que a relação entre o
segmento mais curto e o mais longo seja
menor que $1/4$.
}
\begin{solution}
	Seja $\ell$ o tamanho do comprimento mais curto. Queremos saber quando $\ell/(L-\ell) \le 1/4$, ou seja, quando $\ell \le \frac{L}{5}$. Seja $X$ uma variável aleatória uniformemente distribuída em $(0,L)$, então nosso problema é equivalente à
    \begin{align*}
    	P(X\in(0,\tfrac{L}{5})\cup(\tfrac{4L}{5},L))
        	= \frac{(L/5-0)+(L-4L/5)}{L}
            = \frac{2}{5}.
    \end{align*}
\end{solution}

% 5.13 % % % % % % % % % % % % % % % % % % % % %
\setcounter{question}{12}
\question{Você chega na parada de ônibus as 10:00,
sabendo que o ônibus chegará em algum
horário uniformemente distribuído entre
10:OO e 10:30.
\begin{parts}
	\part Qual é a probabilidade de que você tenha que esperar mais de 10 minutos?
    
    \part Se, as 10:15, o ônibus ainda não tiver
chegado, qual é a probabilidade de
que você tenha que esperar pelo menos mais 10 minutos?
\end{parts}
}
\begin{solution}
Seja $X$ uma variável aleatória uniformemente distribuída em $(0,30)$
\begin{parts}
	\part \[P(X>10) = \frac{30-10}{30} = \frac{2}{3}.\]
    
    \part \[P(X>25\mid X>15) = \frac{P(X>25,X>15)}{P(X>15)}
    	= \frac{P(X>25)}{P(X>15)} = \frac{5/30}{15/30} = \frac{1}{3}.\]
\end{parts} 
\end{solution}

% 5.14 % % % % % % % % % % % % % % % % % % % % %
%\setcounter{question}{12}
\question{Seja $X$ uma variável aleatória uniforme
no intervalo $(0,1)$. Calcule $E[X^n]$ usando
a Proposição~2.1 e depois verifique o resultado usando a definição de esperança.
}
\begin{solution}
	Usando a Proposição~2.1, temos
    \begin{align*}
    	E[X^n] = \int_{-\infty}^{\infty} x^n\,f_X(x)\,\diff x
        	= \int_0^1 x^n\,\diff x = \frac{1}{n+1}.
    \end{align*}
    Usando a definição, primeiro é necessário encontrar a distribuição de $Y=X^n$.
    \begin{align*}
    	F_Y(x) = P(Y\le x) = P(X^n\le x) = P(X\le x^{1/n}) = F_X(x^{1/n}).
    \end{align*}
    Derivando ambos lados da equação, encontramos a densidade,
    \begin{align*}
    	f_Y(x) = \frac{x^{(1-n)/n}}{n} f_X(x^{1/n})
        	= \frac{x^{(1-n)/n}}{n}\quad 0<x<1.
    \end{align*}
    Logo,
    \begin{align*}
    	E[X^n] = E[Y] = \int_{-\infty}^{\infty} x\,f_Y(x)\,\diff x
        	= \int_0^1 x\frac{x^{(1-n)/n}}{n}\,\diff x
        	= \int_0^1 \frac{x^{1/n}}{n}\,\diff x = \frac{1}{n+1}.
    \end{align*}
\end{solution}

% 5.15 % % % % % % % % % % % % % % % % % % % % %
%\setcounter{question}{12}
\question{Se $X$ é uma variável aleatória normal com
parâmetros $\mu = 10$ e $\sigma^2 = 36$, calcule
\begin{parts}
	\part $P(X>5)$;
    \part $P(4<X<16)$;
	\part $P(X<8)$;
    \part $P(X<20)$;
    \part $P(X>16)$.
\end{parts}
}
\begin{solution}
Primeiramente, devemos colocar $X$ em função de uma variável aleatória normal padrão para podermos usar a Tabela~5.1. \\
Assim, $X = \mu+\sigma Z = 10 + 6Z$, onde $Z \sim \mathcal{N}(0,1)$.
\begin{parts}
	\part $P(X>5) = P(10+6Z > 5) = P(Z > -5/6) = 1-\Phi(-5/6)\\
    	{}\quad\quad\quad = \Phi(5/6) \approx \text{0,7967}$.
            
    \part $P(4<X<16) = P(X<16)-P(X<4) = P(Z<1)-P(Z<-1)\\
    	{}\quad\quad\quad = \Phi(1)-\Phi(-1)
        	= 2\Phi(1)-1 \approx \text{0,6826}$.
    
	\part $P(X<8) = P(Z<-1/3) = \Phi(-1/3)	= 1-\Phi(1/3) \approx\text{0,3707}$.

    \part $P(X<20) = P(Z<5/3) = \Phi(5/3) \approx \text{0,9515}$.
    
    \part $P(X>16) = P(Z>1) = 1-P(Z<1) = 1-\Phi(1) \approx\text{0,1587}$.

\end{parts}
\end{solution}

% 5.16 % % % % % % % % % % % % % % % % % % % % %
%\setcounter{question}{12}
\question{O volume anual de chuvas (em mm) em
certa região é normalmente distribuído
com $\mu = 40$ e $\sigma = 4$. Qual é a probabilidade
de que, a contar deste ano, sejam necessários
mais de 10 anos antes que o volume
de chuva em um ano supere 50 mm? Que
hipóteses você está adotando?
}
\begin{solution}
Seja $p$ a probabilidade que o volume de chuva não supere 50 mm em um ano e $Z$ uma variável aleatória normal padrão. Então,
\begin{align*}
	p = P(Z < \tfrac{50-40}{4}) = P(Z<5/2) = \Phi(5/2)
    	\approx \text{0,9938}.
\end{align*}
Se o volume de chuva de cada ano forem independentes, então a probabilidade que demore mais de 10 anos para superar os 50 mm em um ano é dado por $p^{10}\approx \text{0,9397}$.
\end{solution}

% 5.17 % % % % % % % % % % % % % % % % % % % % %
%\setcounter{question}{12}
\question{Um homem praticando tiro ao alvo 
recebe 10 pontos se o tiro estiver a 1 cm do
alvo, 5 pontos se estiver entre 1 e 3 cm do
alvo, e 3 pontos se estiver entre 3 e 5 cm
do alvo. Determine o número esperado
de pontos que ele receberá se a distância
do ponto de tiro até o alvo for uniformemente distribuída entre 0 e 10.
}
\begin{solution}
	Seja $g$ a função que associa a distância ao alvo à pontuação recebida e $X$ a distância ao alvo, então queremos saber
    \begin{align*}
    	E[g(X)]
        	&= \int_0^{10} g(x) f_X(x)\,\diff x \\
        	&= \int_{0}^1 10\tfrac{1}{10} \,\diff x
            	+\int_{1}^3 5\tfrac{1}{10} \,\diff x
                +\int_{3}^5 3\tfrac{1}{10} \,\diff x \\
            &= (1-0) 1 + (3-1) \tfrac{1}{2} + (5-3) \tfrac{3}{10}\\
            &= \frac{13}{5} = \text{2,6}.
    \end{align*}
\end{solution}

% 5.18 % % % % % % % % % % % % % % % % % % % % %
%\setcounter{question}{12}
\question{Suponha que $X$ seja uma variável aleatória
normal com média 0,5. Se $P(X > 9) = \text{0,2}$,
qual é o valor de $\Var(X)$, aproximadamente?
}
\begin{solution}
	Seja $Z$ uma variável aleatória normal padrão, então
    \[\text{0,2} = P(X>9) = P(Z>\text{8,5}/\sigma) = 1-\Phi(\text{8,5}/\sigma),\]
    Logo, $\Phi(\text{8,5}/\sigma) = \text{0,8}$. Da Tabela~5.1, obtemos que $\text{8,5}/\sigma \approx \text{0,84}$ e, portanto, $\sigma \approx \text{10,12}$ e $\Var(X) = \sigma^2 \approx \text{102,4}$.
\end{solution}

% 5.19 % % % % % % % % % % % % % % % % % % % % %
%\setcounter{question}{12}
\question{Seja $X$ uma variável aleatória normal com
média 12 e variância 4. Determine o valor
de $c$ tal que $P(X > c) = 0,1$.
}
\begin{solution}
	Seja $Z$ uma variável aleatória normal padrão, então
    \[\text{0,1} = P(X>c) = P(Z>(c-12)/2) = 1-\Phi((c-12)/2),\]
    Logo, $\Phi((c-12)/2) = \text{0,9}$. Da Tabela~5.1, obtemos que $(c-12)/2 \approx \text{1,28}$ e, portanto, $c \approx \text{14,56}$.
\end{solution}

% 5.20 % % % % % % % % % % % % % % % % % % % % %
%\setcounter{question}{12}
\question{Se 65\% da população de uma grande
comunidade são a favor de um aumento
proposto para as taxas escolares, obtenha
uma aproximação para a probabilidade
de que uma amostra aleatória de 100 pessoas contenha
\begin{parts}
	\part pelo menos 50 pessoas a favor da proposta;
    \part entre 60 e 70 pessoas (inclusive) a favor;
    \part menos de 75 pessoas a favor.
\end{parts}
}
\begin{solution}
A variável aleatória $N$ que representa o número de pessoas à favor é uma binomial de parâmetros $p=\text{0,65}$ e $n=100$. Porém, como $n$ é grande, podemos aproximar por uma normal $X$ de média $\mu = np = 65$ e $\sigma^2 = np(1-p) = \text{22,75}$ (Teorema limite de DeMoivre e Laplace). Seja $Z$ uma normal padrão.
\begin{parts}
	\part Usando \textit{correção de continuidade}, temos que
    \begin{align*}
    	P(N \ge 50) 
        	&\approx P(X\ge \text{49,5})
        		= 1-P(Z < \tfrac{\text{49,5}-65}{\sqrt{\text{22,75}}})\\
        	&= 1-\Phi(-\text{3,25}) = \Phi(\text{3,25}) \approx \text{0,9994}.
    \end{align*}
    
    \part Usando \textit{correção de continuidade}, temos que
    \begin{align*}
    	P(60\le N \le 70) 
        	&\approx P(\text{59,5}\le X\le \text{70,5})\\
        	&= P(Z < \tfrac{\text{70,5}-65}{\sqrt{\text{22,75}}})
            	- P(Z < \tfrac{\text{59,5}-65}{\sqrt{\text{22,75}}})\\
        	&= \Phi(\text{1,15})-\Phi(-\text{1,15}) = 2\Phi(\text{1,15})-1\\
            &\approx \text{0,7498}.
    \end{align*}
    
    \part Usando \textit{correção de continuidade}, temos que
    \begin{align*}
    	P(N < 75) 
        	\approx P(X < \text{74,5})
        	= P(Z < \tfrac{\text{74,5}-65}{\sqrt{\text{22,75}}})
        	= \Phi(\text{1,99}) \approx \text{0,9767}.
    \end{align*}
\end{parts}
\end{solution}

% 5.23 % % % % % % % % % % % % % % % % % % % % %
\setcounter{question}{22}
\question{Realizam-se mil jogadas independentes
de um dado honesto. Calcule a probabilidade
aproximada de que o número 6
apareça entre 150 e 200 vezes, inclusive.
Se o número 6 aparecer exatamente 200
vezes, determine a probabilidade de que
o número 5 apareça menos de 150 vezes.
}
\begin{solution}
	Seja $N$ uma variável aleatória com distribuição binomial $n$ e $p$. Quando $n$ é grande, podemos aproximar por uma variável aleatória $X$ distribuída como normal de parâmetros $\mu=np$ e $\sigma^2=np(1-p)$. Seja $Z$ a normal padrão.\\
    Para o número 6 aparecer entre 150 e 200 vezes em mil jogadas, temos que $n=1000$, $p=1/6$ e queremos saber
    \begin{align*}
    	P(150\le N\le 200)
        	&\approx P(\text{149,5} \le X \le \text{200,5})
            	= P(\tfrac{\text{200,5}-\text{166,7}}{\text{11,785}} \le Z 
            	\le \tfrac{\text{149,5}-\text{166,7}}{\text{11,785}})\\
            &= \Phi(\text{2,87}) - \Phi(-\text{1,5})
            	= \Phi(\text{2,87})+\Phi(\text{1,5})-1 \approx \text{0,9311}.
    \end{align*}
    Por outro lado, se sabemos que já saíram exatamente 200 dados marcando 6, sobram 800 dados para saírem números 5, agora com 1/5 de chance. Fazendo exatamente o mesmo procedimento feito acima, mas com $n=800$ e $p=1/5$, chegamos em $P(N<150)\approx P(X<\text{149,5})\approx \text{0,1762}$.
\end{solution}

% 5.28 % % % % % % % % % % % % % % % % % % % % %
\setcounter{question}{27}
\question{Em 10.000 jogadas independentes de uma
moeda, observou-se que deu cara 5800
vezes. É razoável supor que essa moeda
não seja honesta? Explique.
}
\begin{solution}
	Seja $N$ uma variável aleatória com distribuição binomial $n$ e $p$. Quando $n$ é grande, podemos aproximar por uma variável aleatória $X$ distribuída como normal de parâmetros $\mu=np$ e $\sigma^2=np(1-p)$. Seja $Z$ a normal padrão.\\
    Se a moeda for honesta, então $n=10\,000$, $p=1/2$. Nesse caso, não é necessário correção de continuidade, pois $n$ é realmente grande. É interessante saber a seguinte probabilidade
    \begin{align*}
    	P(N\ge 5\,800)
        	&\approx P( X \ge \text{5800} )
            	= P(Z \ge \tfrac{5800-5000}{\sqrt{2500}})\\
            &= 1-\Phi(16) \approx 0.
    \end{align*}
    Dessa forma, como a chance de pelo menos $5\,800$ caras em $10\,000$ lançamentos é praticamente nula, então é bem razoável supor que a moeda não seja honesta.
\end{solution}

% 5.31 % % % % % % % % % % % % % % % % % % % % %
\setcounter{question}{30}
\question{
\begin{parts}
	\part Uma estação de bombeiros deve ser
instalada ao longo de uma estrada
com comprimento $A$, $A < \infty$. Se incêndios
ocorrem em pontos uniformemente distribuídos
no intervalo $(0,A)$, qual deveria ser a localização da
estação de forma a minimizar-se a distância
esperada para o incêndio? Isto
é, escolha $a$ de forma que $E[|X-a|]$ seja 
minimizado quando $X$ for uniformemente
distribuído ao longo de $(0,A)$.
    \part Agora suponha que a estrada tenha
comprimento infinito -- indo do ponto
0 até $\infty$. Se a distância de um incêndio até o ponto 0 é exponencialmente distribuída com taxa $\lambda$, onde deveria
estar localizada a estação? Isto é, queremos
minimizar $E[|X - a|]$, onde $X$ é
agora exponencial com taxa $\lambda$.
\end{parts}
}
\begin{solution}
\begin{parts}
	\part Comecemos calculando a esperança em função de $a\in[0,A]$,
    \begin{align*}
    	E[|X-a|] 
        	&= \int_{-\infty}^\infty |x-a| f_X(x)\,\mathrm{d}x \\
        	&= \int_0^A |x-a| \frac{1}{A}\,\mathrm{d}x \\
            &= \int_0^a \frac{(a-x)}{A}\mathrm{d}x + \int_a^A \frac{(x-a)}{A}\mathrm{d}x\\
			&= \frac{a^2}{A} - a + \frac{A}{2}.
    \end{align*}
    Para encontrar o ponto de mínimo, derivamos a expressão acima em relação à $a$ e igualamos à zero, encontrando $a=A/2$. É um ponto de mínimo global, pois a segunda derivada é positiva em toda região de interesse (função é convexa).
    
    \part Analogamente,
    \begin{align*}
    	E[|X-a|] 
        	&= \int_{-\infty}^\infty |x-a| f_X(x)\,\mathrm{d}x \\
        	&= \int_0^\infty |x-a| \lambda\euler^{-\lambda x}\,\mathrm{d}x \\
            &= \int_0^a (a-x)\lambda\euler^{-\lambda x}\mathrm{d}x 
            	+ \int_a^\infty (x-a)\lambda\euler^{-\lambda x}\mathrm{d}x\\
			&= a + \frac{2\,\euler^{-\lambda a}-1}{\lambda}.
    \end{align*}
\end{parts}
	Novamente, derivamos em relação à $a$ e igualamos à zero,
    \begin{align*}
    	1 - 2\euler^{-\lambda a} = 0 \Rightarrow a = \frac{\ln(2)}{\lambda}.
    \end{align*}
    É um ponto de mínimo global, pois a segunda derivada da função é positiva em toda região de interesse (é convexa).
\end{solution}

% 5.32  % % % % % % % % % % % % % % % % % % % % %
%\setcounter{question}{36}
\question{O tempo (em horas) necessário para a
manutenção de uma máquina é uma variável aleatória
exponencialmente distribuída com $\lambda = 1/2$. Qual é
\begin{parts}
	\part a probabilidade de que um reparo
dure mais que 2 horas?
    \part a probabilidade condicional de que o
tempo de reparo dure pelo menos 10
horas, dado que a sua duração seja superior a 9 horas?
\end{parts}
}
\begin{solution}
Seja $X$ o tempo para manutenção da máquina, então $f_X(x) = \frac{1}{2} \euler^{-x/2}$.
\begin{parts}
	\part 
    \begin{align*}
    	P(X>2) 
        	= \int_{2}^\infty \frac{1}{2}\euler^{-x/2}\,\mathrm{d}x
        	= \euler^{-1} \approx \text{0,368}.
    \end{align*}
    
    \part 
    \begin{align*}
    	P(X>10\mid X>9) = \frac{P(X>10)}{P(X>9)} = \frac{\euler^{-10/2}}{\euler^{-9/2}}
        	= \euler^{-1/2} \approx \text{0,607}.
    \end{align*}
    A exponencial não tem memória, o problema é equivalente à $P(X>1)$.
\end{parts}
\end{solution}


% 5.37  % % % % % % % % % % % % % % % % % % % % %
\setcounter{question}{36}
\question{Se a variável aleatória $X$ é uniformemente distribuída ao longo do intervalo $(-1,1)$, determine:
\begin{parts}
	\part $P(|X| > 1/2)$;
    
    \part a função densidade da variável aleatória $|X|$.
\end{parts}
}
\begin{solution}
\begin{parts}
	\part
    \begin{align*}
    	P(|X| > 1/2) = P(X > 1/2) + P(X < -1/2) = 1/4+1/4 = 1/8.
    \end{align*}
    
    \part A função distribuição acumulada de $|X|$ é dada por
    \begin{align*}
    	F_{|X|}(x) = P(|X| \le x) = P(-x \le X \le x)
        	= F_X(x) - F_X(-x),
    \end{align*}
    quando $x > 0$. Por outro lado, $F_{|X|}(x) = 0$, se $x<0$.\\
    Derivando dos dois lados de cada equação, obtemos a densidade de $|X|$,
    \begin{align*}
    	f_{|X|}(x) = f_X(x) + f_X(-x) = 1, \quad &0<x<1,
    \end{align*}
    e $f_{|X|}(x) = 0$, caso contrário.
\end{parts}
\end{solution}

% 5.38  % % % % % % % % % % % % % % % % % % % % %
%\setcounter{question}{36}
\question{Se a variável aleatória $Y$ é uniformemente distribuída ao longo do intervalo $(0,5)$, qual é a probabilidade de que as raízes da
equação $4x^2 + 4xY + Y + 2 = 0$ sejam ambas reais?
}
\begin{solution}
	Para uma equação do segundo grau do tipo $ax^2+bx+c=0$ ter raízes reais é necessário e suficiente que $b^2-4ac\ge 0$. Logo, queremos saber
    \begin{align*}
    	P((4Y)^2-4\cdot4\,(Y+2) \ge 0)
        	&= P(Y^2-Y-2\ge 0) \\
        	&= P( (Y-2)(Y+1) \ge 0 )\\
            &= P(Y\le -1) + P(Y \ge 2)\\
            &= 0 + \frac{5-2}{5} = \frac{3}{5}.
    \end{align*}
\end{solution}

% 5.39  % % % % % % % % % % % % % % % % % % % % %
%\setcounter{question}{36}
\question{Se $X$ é uma variável aleatória exponencial
com parâmetro $\lambda = 1$, calcule a função
densidade de probabilidade da variável
aleatória $Y$ definida como $Y = \log X$.
}
\begin{solution}
	Comecemos pela função distribuição acumulada
    \begin{align*}
    	F_Y(x) = P(Y \le x) = P(\log X \le x) = P(X \le \euler^x) = F_X(\euler^x).
    \end{align*}
    Derivando ambos lados da equação em relação à $x$, obtemos que
    \begin{align*}
    	f_Y(x) = \euler^x\,f_x(\euler^x)
        	= \lambda \exp(x-\lambda\euler^x) = \exp(x-\euler^x).
    \end{align*}
\end{solution}

% 5.40  % % % % % % % % % % % % % % % % % % % % %
%\setcounter{question}{36}
\question{Se $X$ é uniformemente distribuída ao longo do intervalo $(0, 1)$, determine a função densidade de $Y = \euler^X$.
}
\begin{solution}
	Comecemos pela função distribuição acumulada ($x>0$)
    \begin{align*}
    	F_Y(x) = P(Y \le x) = P(\euler^X \le x) = P(X \le \log(x)) = F_X(\log(x)).
    \end{align*}
    Derivando ambos lados da equação em relação à $x$, obtemos que
    \begin{align*}
    	f_Y(x) = \frac{f_x(\log(x))}{x} = 
        \begin{cases}
        	\frac{1}{x},	&\quad 1<x<\euler;\\
            0,				&\quad \text{caso contrário}.
        \end{cases}
    \end{align*}
\end{solution}

% 5.41  % % % % % % % % % % % % % % % % % % % % %
%\setcounter{question}{36}
\question{Determine a distribuição de $R = A \sin\theta$,
onde $A$ é uma constante fixa, e $\theta$ é
uma variável aleatória uniformemente
distribuída em $(-\pi/2, \pi/2)$. A variável
aleatória $R$ surge da teoria da balística.
Se um projétil é disparado de sua origem
com um ângulo $\alpha$ em relação à superfície
da terra com uma velocidade $v$, então o
ponto $R$ no qual ele retorna à terra pode
ser escrito como $R = (v^2/g)\sin 2\alpha$, onde
$g$ é a aceleração da gravidade,que é igual
a 9,8 m/s$^2$.
}
\begin{solution}
	Comecemos pela função distribuição acumulada
    \begin{align*}
    	F_R(r) = P(R \le r) = P(A\sin\theta \le r) = P(\theta \le \arcsin(r/A) )
        	= F_\theta(\arcsin(r/A)).
    \end{align*}
    Derivando ambos lados da equação em relação à $r$, obtemos que
    \begin{align*}
    	f_R(r) = \frac{f_\theta(\arcsin(r/A))}{\sqrt{A^2-r^2}} = 
        \begin{cases}
        	\frac{1}{\pi\sqrt{A^2-r^2}},	&\quad -A<r<A;\\
            0,								&\quad \text{caso contrário}.
        \end{cases}
    \end{align*}
\end{solution}

\end{questions}


% \setcounter{section}{4}
% \section{Exercícios Teóricos}
% \begin{questions}

% 5.27 % % % % % % % % % % % % % % % % % % % % %
\setcounter{question}{26}
\question{
Se $X$ é uniformemente distribuída em $(a, b)$, qual variável aleatória que varia linearmente com $X$ é uniformemente distribuída em $(0, 1)$?
}
\begin{solution}
Seja $Y = (X-a)/(b-a)$, então
\begin{align*}
	F_Y(y) &= P(Y\le y) = P((X-a)/(b-a) \le y)\\
    	&= P(X \le (b-a)\,y+a) = F_X((b-a)\,y+a).
\end{align*}
Derivando os dois lados da equação em relação à $y$ obtemos que
\begin{align*}
	f_Y(y) =  (b-a)f_X((b-a)\,y+a) =
    \begin{cases}
    	1, &\text{se }y\in(0,1);\\
        0, &\text{caso contrário.}
    \end{cases}
\end{align*}
Logo, $Y$ é uniformemente distribuída em $(0,1)$.\\[1mm]
\textit{Observação:} outra opção é fazer $Y = (b-X)/(b-a)$.
\end{solution}

% 5.29 % % % % % % % % % % % % % % % % % % % % %
\setcounter{question}{28}
\question{
Seja $X$ uma variável aleatória contínua
com função distribuição cumulativa $F$.
Defina a variável aleatória $Y$ como $Y = F(X)$.
Mostre que $Y$ é uniformemente distribuída em $(0, 1)$.
}
\begin{solution}
	Por simplicidade, vamos supor que $F: \mathbb{R} \to [0,1]$ seja estritamente crescente. Logo, $F$ é inversível e para $y \in (0,1)$,
	\begin{align*}
		F_Y(y) = P(Y\le y) = P(F(X)\le y) = P(X \le F^{-1}(y)) = F(F^{-1}(y)) = y.
	\end{align*}
    Dessa forma, quando $y \in \mathbb{R}$,
    \begin{align*}
    	F_Y(y) =
        \begin{cases}
    	0, &\text{se }y \le 0;\\
        y, &\text{se }0 < y < 1;\\
        1, &\text{se }y \ge 1;\\
    	\end{cases}
    \end{align*}
    o que caracteriza uma distribuição uniforme em $(0,1)$.\\[1mm]
    \textit{Observação:} Isso também acontece quando $F$ não é inversível.
\end{solution}

% 5.30 % % % % % % % % % % % % % % % % % % % % %
%\setcounter{question}{28}
\question{
Suponha que $X$ tenha função densidade
de probabilidade $f_X$. Determine a função
densidade de probabilidade da variável
aleatória $Y$ definida como $Y = aX + b$.
}
\begin{solution}
	Seja $a>0$,
	\begin{align*}
		F_Y(y) = P(Y\le y) = P(aX+b\le y) = P(X\le (y-b)/a) = F_X((y-b)/a).
	\end{align*}
    Derivando ambos lados da equação em relação à $y$ leva à
    \begin{align*}
    	f_Y(y) = \frac{f_X((y-b)/a)}{a}.
    \end{align*}
    Por outro lado, se $a<0$, então
    \begin{align*}
    	F_Y(y) &= P(X\ge (y-b)/a) = 1-P(X<(y-b)/a) = 1-F_X([(y-b)/a]^-)\\
        	&= 1-F_X((y-b)/a) \quad\text{(a variável aleatória é contínua).}
    \end{align*}
    Novamente, derivando ambos lados da equação em relação à $y$ leva à
    \begin{align*}
    	f_Y(y) = \frac{f_X((y-b)/a)}{-a}.
    \end{align*}
    Portanto, quando $a\neq 0$,
    \begin{align*}
    	f_Y(y) = \frac{f_X((y-b)/a)}{|a|}.
    \end{align*}
\end{solution}

\end{questions}
%\newpage

\end{document}