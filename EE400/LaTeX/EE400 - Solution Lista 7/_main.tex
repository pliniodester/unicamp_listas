\documentclass[answers, 12pt]{exam}

\usepackage{graphicx,epsfig,psfrag,rotate,xcolor,url}
\usepackage{amsmath,amsthm,amsfonts,amssymb,mathrsfs,latexsym,multicol}
\usepackage{setspace,enumerate,enumitem,ifthen,subfig}
\usepackage{hyperref}
\usepackage{siunitx,tcolorbox}
\usepackage[portuges]{babel}
\usepackage[utf8]{inputenc}
\usepackage[algo2e,english,onelanguage,algoruled]{algorithm2e}
\usepackage[margin=0.7in]{geometry}
\usepackage{calrsfs,empheq}
\renewcommand{\rmdefault}{pplx}
\usepackage{cancel}
\renewcommand{\CancelColor}{\color{red}}
\usepackage{eulervm}
\DeclareMathAlphabet{\mathcal}{OMS}{zplm}{m}{n}
\def\defeq{\mathrel{\mathop:}=}

\usepackage{tikz, circuitikz, pgfplots}
\pgfplotsset{compat=1.15}

\newcommand{\euler}{\mathrm{e}}
\newcommand\diff{\mathrm{d}}
\newcommand{\Log}{\mathop{Log}}

\renewcommand{\thequestion}{\arabic{section}.\arabic{question}}
\renewcommand{\solutiontitle}{\noindent\textbf{Solução:}\enspace}
\renewcommand{\d}{\mathrm{d}}
\renewcommand{\Re}{\mathsf{Re}}
\renewcommand{\Im}{\mathsf{Im}}

\newtheorem{theorem}{Teorema}

\def\C{{\mathbb C}}
\def\N{{\mathbb N}}
\def\R{{\mathbb R}}
\def\Z{{\mathbb Z}}
\def\Q{{\mathbb Q}}
\def\E{{\mathbb E}}
\def\X{{\mathbb X}}
\def\ind{\mathds{1}}
\def\cal{\mathcal}
\def\T{\top}

\footer{}{\thepage}{}

\title{	%EA044 - Planejamento e Análise\\ de Sistemas de Produção - 2S 2018\\
		% {\large \textit{Docente}: Matheus Souza}\\[2mm]
        EE400 -- \textit{Métodos da Engenharia Elétrica} \\[-0mm]
        {\Large Docente: Matheus Souza} \\[+1mm]
        {\Large Solução para problemas selecionados}\\[-0mm]
        -- Lista de Exercícios 7 --
}
\author{Plínio S. Dester\\ (\url{pliniodester@hotmail.com})}

\begin{document}

%% Content goes here
\maketitle

Em caso de dúvidas, sugestões ou correções, não hesite em mandar um e-mail.

% Lista 07: Exercícios 1,2,3,4,5,6,7,8 + Problemas 1,2,6,8,11;

\setcounter{section}{6}
\section{Exercícios}

\begin{questions}

% 7.2  % % % % % % % % % % % % % % % % % % % % %
\setcounter{question}{1}
\question{
Encontre todos os valores de $z$ que verificam as seguintes igualdades.
\begin{multicols}{3}
    \begin{enumerate}[label=(\alph*)]
      \item $\displaystyle e^z = -2$;
      \item $\displaystyle e^z = 1 + \sqrt{3}i$;
      \item $\displaystyle e^{2z - 1} = 1$.
    \end{enumerate}
\end{multicols}
}
\begin{solution}
Seja $z = x+iy$.
    \begin{enumerate}[label=(\alph*)]
      \item Sabemos que
      $
          e^z = e^x e^{iy} = e^x \cos y + i e^x \sin y.
      $
      Assim, $e^x \cos y = -2$ e $e^x \sin y = 0$.
      Portanto, $x = \ln 2$ e $y = \pi + 2k\pi$, $k\in\Z$. Logo, $z = \ln 2 + i(\pi + 2k\pi)$, $k\in\Z$.
      
      \item $\displaystyle e^z = 1 + \sqrt{3}i = 2\,e^{i(\pi/3 + 2k\pi)}$, ou seja, $z = \ln 2 + i(\pi/3 + 2k\pi)$, $k\in\Z$.
      
      \item $\displaystyle e^{2z - 1} = 1 = e^{i 2k\pi}$ para todo $k\in\Z$. Portanto, $z = 1/2 + ik\pi$, $k\in\Z$.
    \end{enumerate}
\end{solution}

% 7.3 % % % % % % % % % % % % % % % % % % % % %
% \setcounter{question}{2}
\question{
A função $f$ dada por $f(z) = e^{z^2}$ é inteira? Calcule a sua derivada.
}
\begin{solution}
    A função $f$ é inteira, pois trata-se de uma composição de duas funções inteiras: $e^z$ e $z^2$. Usando a regra da cadeia, temos que
    \begin{align*}
        \frac{\d}{\d z}e^{z^2} = 2z\,e^{z^2}.
    \end{align*}
\end{solution}

% 7.4 % % % % % % % % % % % % % % % % % % % % %
% \setcounter{question}{8}
\question{
Quais as restrições sobre $z$ para que $e^z$ seja real? E quais as restrições sobre $z$ para que $e^z$ seja um número imaginário puro?
}

\begin{solution}
    Seja $z = x+iy$. Usando a fórmula de Euler $e^z = e^x e^{iy} = e^x \cos y + i e^x \sin y$. Portanto, $e^z$ será real se $\sin y = 0$, isto é, se $y = k\pi$, $k\in\Z$. Por outro lado, $e^z$ será imaginário puro se $\cos y = 0$, isto é, se $y = \pi/2+k\pi$, $k\in\Z$.
\end{solution}

% 7.7 % % % % % % % % % % % % % % % % % % % % %
\setcounter{question}{6}
\question{
Calcule os valores principais das seguintes potências.
\begin{multicols}{3}
    \begin{enumerate}[label=(\alph*)]
      \item $\displaystyle i^i$;
      \item $\displaystyle \left[ \frac{e}{2} (-1 - \sqrt{3} i) \right]^{3\pi i}$;
      \item $\displaystyle (1 - i)^{4i}$.
    \end{enumerate}
\end{multicols}
}

\begin{solution}
    \begin{enumerate}[label=(\alph*)]
      \item $\displaystyle i^i = e^{i \Log i} = e^{i(\ln 1 + i\pi/2)} = e^{-\pi/2}$;
      \item $\displaystyle \left[ \frac{e}{2} (-1 - \sqrt{3} i) \right]^{3\pi i}
        = e^{3\pi i \Log\left(e/2 (-1-\sqrt{3}i)\right)}
        = e^{3\pi i (\ln(e) - i 2 \pi/3 )}
        = e^{3\pi i + 2 \pi^2} = -e^{2\pi^2}$;
      \item $\displaystyle (1 - i)^{4i} = e^{4i\Log(1-i)}
        = e^{4i(\ln \sqrt 2 - i \pi/4)}
        = e^{\pi+ i 2\ln 2} = e^{\pi} (\cos (2\ln 2) + i \sin (2\ln2) )$.
    \end{enumerate}
\end{solution}


% 7.8  % % % % % % % % % % % % % % % % % % % % %
% \setcounter{question}{7}
\question{
Encontre todas as raízes das seguintes equações em $\C$.
\begin{multicols}{2}
    \begin{enumerate}[label=(\alph*)]
      \item $\displaystyle \log z = i \pi/2$;
      \item $\displaystyle \sin z = \cosh 4$;
      \item $\displaystyle \cos z = 2$;\\
      \item $\displaystyle \sinh z = i$;
      \item $\displaystyle \cosh z = -2$.
    \end{enumerate}
\end{multicols}
}
\begin{solution}
Seja $z = x+iy$.
\begin{enumerate}[label=(\alph*)]
    \item $\displaystyle \log z = \ln|z| + i\arg(z) = i \pi/2$.
    Logo, $|z| = 1$ e $\arg(z) = \pi/2$. Assim, $z = i$.
    
    \item $\displaystyle \sin z = \sin x \cosh y + i \cos x \sinh y = \cosh 4$.\\
    Logo, $x = \pi/2+2k\pi$ e $y = 4$, ou seja, $z = \pi/2+2k\pi + 4i$, $k\in\Z$.
    
    \item $\displaystyle \cos z = \cos x \cosh y + i \sin x \sinh y = 2$.\\
    Logo, $x = 2k\pi$ e $\cosh y = 2$, ou seja, $y = \ln(2\pm\sqrt{3})$.\\
    Dessa forma, $z = 2k\pi + i \ln(2\pm\sqrt{3})$, $k\in\Z$.
    
    \item $\displaystyle \sinh z = (e^z-e^{-z})/2 = i$, ou seja, $e^z-e^{-z} = 2i$. Logo, $z = i(\pi/2 + 2k\pi)$, $k\in\Z$.
    
    \item $\displaystyle \cosh z = -2$, ou seja, $e^z+e^{-z} = 4$.
    Logo, $z = \ln(2\pm\sqrt{3}) + i(\pi+2k\pi)$, $k\in\Z$.
    
\end{enumerate}
\end{solution}

\end{questions}


\setcounter{section}{6}
\section{Problemas}

\begin{questions}

% 7.8  % % % % % % % % % % % % % % % % % % % % %
\setcounter{question}{7}
\question{
{\bf (Fórmulas de Adição e Identidades Trigonométricas)} Use a definição das funções $\sin$ e $\cos$ para números complexos para mostrar que valem as fórmulas
 \[
 \sin(z_1 + z_2) = \sin(z_1)\cos(z_2) + \sin(z_2)\cos(z_1) \quad \text{e} \quad \cos(z_1 + z_2) = \cos(z_1)\cos(z_2) - \sin(z_1)\sin(z_2)
 \]   
 para quaisquer dois números complexos $z_1$ e $z_2$. A partir destas identidades, conclua que as fórmulas de arco duplo
 \[
 \sin(2z) = 2\sin z \cos z \quad \text{e} \quad \cos(2z) = \cos^2 z - \sin^2 z
 \]  
 são verificadas para todo $z \in \C$. Use as fórmulas de adição, também, para mostrar que
 \[
 \sin(z + 2\pi) = \sin(z) \quad \text{e} \quad \cos(z + 2\pi) = \cos(z)
 \]
 também valem para todo $z \in \C$. Isto é, as funções trigonométricas complexas continuam periódicas. Finalmente, use a fórmula de adição do cosseno (ou a definição) para mostrar que
 \[
 \sin^2 z + \cos^2 z = 1
 \]
 permanece válida para todo $z \in \C$.
 \begin{solution}
 Para a primeira identidade temos que
    \begin{align*}
        \sin(z_1)\cos(z_2) + \sin(z_2)\cos(z_1)
        &= \frac{1}{4i}\left( (e^{i z_1}-e^{-i z_1})(e^{i z_2}+e^{-i z_2}) +
            (e^{i z_2}-e^{-i z_2})(e^{i z_1}+e^{-i z_1})  \right)\\
        &= \frac{1}{4i} \left( e^{i (z_1+z_2)} + e^{i (z_1-z_2)} - e^{i (-z_1+z_2)} - e^{i (-z_1-z_2)} \right. \\ 
            &\quad \left. + e^{i (z_1+z_2)} + e^{i (-z_1+z_2)} -  e^{i (z_1-z_2)} -  e^{i (-z_1-z_2)} \right)\\
        &= \frac{e^{i (z_1+z_2)} - e^{-i (z_1+z_2)}}{2i} \\
        &= \sin(z_1+z_2).
    \end{align*}
    Analogamente, podemos chegar na segunda identidade.
    Ainda,
    \[\sin(z+2\pi) = \sin(z)\cos(2\pi) + \sin(2\pi)\cos(z) = \sin(z).\]
    Analogamente, podemos fazer o mesmo para a função $\cos$.
    Por fim,
    \begin{align*}
        \sin^2 z+\cos^2 z &= \left(\frac{e^{iz}-e^{-iz}}{2i}\right)^2 + \left(\frac{e^{iz} + e^{-iz}}{2}\right)^2 \\
            &= \frac{1}{4} \left( -e^{2iz} + 2 - e^{-2iz} + e^{2iz} + 2 +e^{-2iz} \right)\\
            &= 1.
    \end{align*}
\end{solution}

% 7.9  % % % % % % % % % % % % % % % % % % % % %
\question
{\bf (Funções Trigonométricas e Funções Hiperbólicas I)} Mostre que, se $z = x + i y$, então
 \[
 \sin z = \sin x \cosh y + i \cos x \sinh y \quad \text{e} \quad \cos z = \cos x \cosh y - i \sin x \sinh y.
 \]
 Use ambas as expressões para mostrar que $|\sin z|^2 = \sin^2 x + \sinh^2 y$ e que $|\cos z|^2 = \cos^2 x + \sinh^2 y$.
}
\begin{solution}
    \begin{align*}
        \sin z &= \frac{e^{iz}-e^{-iz}}{2i}
                = \frac{e^{ix}e^{-y}-e^{-ix}e^{y}}{2i} \\
            &= \frac{1}{2i}\left( e^{-y} \cos x + i e^{-y} \sin x - e^{y} \cos x + i e^{y} \sin x \right) \\
            &=  i \cos x\,\frac{e^{y} - e^{-y}}{2} + \sin x\,\frac{e^{y} + e^{-y}}{2} \\
            &= \sin x \cosh y + i \cos x \sinh y.
    \end{align*}
    Analogamente, podemos mostrar a identidade envolvendo $\cos z$.
    \begin{align*}
        |\sin z|^2 &= \sin^2 x \cosh^2 y + \cos^2 x \sinh^2 y \\
            &= \sin^2 x \cosh^2 y + (1 - \sin^2 x) \sinh^2 y \\
            &= \sin^2 x (\cosh^2 y - \sinh^2 y) + \sinh^2 y \\
            &= \sin^2 x + \sinh^2 y.
    \end{align*}
    Analogamente, podemos mostrar a identidade envolvendo $|\cos z|^2$.
\end{solution}

% 7.11  % % % % % % % % % % % % % % % % % % % % %
\setcounter{question}{10}
\question{
Use os resultados do Problema 8 para mostrar que, se $z = x + iy$, valem as desigualdades $|\sin z| \geq |\sin x|$ e $|\cos z| \geq |\cos x|$. Ademais, mostre também que $|\sin z|^2 + |\cos z|^2 = 1$ se, e apenas se, $z \in \R$.
}
\begin{solution}
Vamos usar os resultados do Problema 9, pois é mais direto.
    \begin{align*}
        |\sin z|^2 &= \sin^2 x + \sinh^2 y \ge \sin^2 x,\\
        |\cos z|^2 &= \cos^2 x + \sinh^2 y \ge \cos^2 x.
    \end{align*}
    Logo, $|\sin z| \geq |\sin x|$ e $|\cos z| \geq |\cos x|$. Além disso,
    \begin{align*}
        |\sin z|^2 + |\cos z|^2
            = \sin^2 x + \sinh^2 y + \cos^2 x + \sinh^2 y
            = 1 + 2\sinh^2 y.
    \end{align*}
    Porém, $1 + 2\sinh^2 y = 1$ se, e somente se, $y = 0$, o que é equivalente à  $z\in\R$.
\end{solution}

% 7.12  % % % % % % % % % % % % % % % % % % % % %
% \setcounter{question}{2}
\question{
{\bf (Paradoxo I)} Observe a série de igualdades:
 \[
 i = e^{i\frac{\pi}{2}} = e^{2\pi i\frac{1}{4}} = \left(e^{2\pi i} \right)^{\frac{1}{4}} = 1^\frac{1}{4} = 1. 
 \]
 Aponte o erro.
}
\begin{solution}
    O erro está no passo $e^{2\pi i\frac{1}{4}} = \left(e^{2\pi i} \right)^{\frac{1}{4}}$, pois em $\C$ a igualdade $(z^a)^b = z^{ab}$ nem sempre é satisfeita quando tomamos o valor principal. Um exemplo disso é para o caso $z=-1,a=1/3,b=3$.
\end{solution}

% 7.13  % % % % % % % % % % % % % % % % % % % % %
% \setcounter{question}{4}
\question{
{\bf (Paradoxo II)} Observe a seguinte sequência de implicações:
 \[
 (-z)^2 = z^2 \Rightarrow 2 \log(-z) = 2\log z \Rightarrow \log(-z) = \log(z).
 \]
 Aponte o erro.
}%
\begin{solution}%
    O erro está na última implicação, pois seja $z = \rho e^{i\theta}$, então
    \begin{align*}
        2 \log(-z) = 2\log(z) &\Rightarrow 2\ln \rho + 2i(\theta+\pi) = 2\ln \rho + 2\theta i + 2k\pi i,~ k\in\Z, \\
            &\Rightarrow 2\pi = 2k\pi,
     \end{align*}
     o que é verdade para $k = 1$. Porém,
     \begin{align*}
        \log(-z) = \log(z) &\Rightarrow \ln \rho + i(\theta+\pi) = \ln \rho + \theta i + 2k\pi i,~ k\in\Z, \\
            &\Rightarrow \pi = 2k\pi,
    \end{align*}
    o que não é satisfeito para nenhum $k\in\Z$.
\end{solution}

\end{questions}


% \setcounter{section}{4}
% \section{Exercícios Teóricos}
% \begin{questions}

% 5.27 % % % % % % % % % % % % % % % % % % % % %
\setcounter{question}{26}
\question{
Se $X$ é uniformemente distribuída em $(a, b)$, qual variável aleatória que varia linearmente com $X$ é uniformemente distribuída em $(0, 1)$?
}
\begin{solution}
Seja $Y = (X-a)/(b-a)$, então
\begin{align*}
	F_Y(y) &= P(Y\le y) = P((X-a)/(b-a) \le y)\\
    	&= P(X \le (b-a)\,y+a) = F_X((b-a)\,y+a).
\end{align*}
Derivando os dois lados da equação em relação à $y$ obtemos que
\begin{align*}
	f_Y(y) =  (b-a)f_X((b-a)\,y+a) =
    \begin{cases}
    	1, &\text{se }y\in(0,1);\\
        0, &\text{caso contrário.}
    \end{cases}
\end{align*}
Logo, $Y$ é uniformemente distribuída em $(0,1)$.\\[1mm]
\textit{Observação:} outra opção é fazer $Y = (b-X)/(b-a)$.
\end{solution}

% 5.29 % % % % % % % % % % % % % % % % % % % % %
\setcounter{question}{28}
\question{
Seja $X$ uma variável aleatória contínua
com função distribuição cumulativa $F$.
Defina a variável aleatória $Y$ como $Y = F(X)$.
Mostre que $Y$ é uniformemente distribuída em $(0, 1)$.
}
\begin{solution}
	Por simplicidade, vamos supor que $F: \mathbb{R} \to [0,1]$ seja estritamente crescente. Logo, $F$ é inversível e para $y \in (0,1)$,
	\begin{align*}
		F_Y(y) = P(Y\le y) = P(F(X)\le y) = P(X \le F^{-1}(y)) = F(F^{-1}(y)) = y.
	\end{align*}
    Dessa forma, quando $y \in \mathbb{R}$,
    \begin{align*}
    	F_Y(y) =
        \begin{cases}
    	0, &\text{se }y \le 0;\\
        y, &\text{se }0 < y < 1;\\
        1, &\text{se }y \ge 1;\\
    	\end{cases}
    \end{align*}
    o que caracteriza uma distribuição uniforme em $(0,1)$.\\[1mm]
    \textit{Observação:} Isso também acontece quando $F$ não é inversível.
\end{solution}

% 5.30 % % % % % % % % % % % % % % % % % % % % %
%\setcounter{question}{28}
\question{
Suponha que $X$ tenha função densidade
de probabilidade $f_X$. Determine a função
densidade de probabilidade da variável
aleatória $Y$ definida como $Y = aX + b$.
}
\begin{solution}
	Seja $a>0$,
	\begin{align*}
		F_Y(y) = P(Y\le y) = P(aX+b\le y) = P(X\le (y-b)/a) = F_X((y-b)/a).
	\end{align*}
    Derivando ambos lados da equação em relação à $y$ leva à
    \begin{align*}
    	f_Y(y) = \frac{f_X((y-b)/a)}{a}.
    \end{align*}
    Por outro lado, se $a<0$, então
    \begin{align*}
    	F_Y(y) &= P(X\ge (y-b)/a) = 1-P(X<(y-b)/a) = 1-F_X([(y-b)/a]^-)\\
        	&= 1-F_X((y-b)/a) \quad\text{(a variável aleatória é contínua).}
    \end{align*}
    Novamente, derivando ambos lados da equação em relação à $y$ leva à
    \begin{align*}
    	f_Y(y) = \frac{f_X((y-b)/a)}{-a}.
    \end{align*}
    Portanto, quando $a\neq 0$,
    \begin{align*}
    	f_Y(y) = \frac{f_X((y-b)/a)}{|a|}.
    \end{align*}
\end{solution}

\end{questions}
%\newpage

\end{document}