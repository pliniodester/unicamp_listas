\begin{questions}

% 5.27 % % % % % % % % % % % % % % % % % % % % %
\setcounter{question}{26}
\question{
Se $X$ é uniformemente distribuída em $(a, b)$, qual variável aleatória que varia linearmente com $X$ é uniformemente distribuída em $(0, 1)$?
}
\begin{solution}
Seja $Y = (X-a)/(b-a)$, então
\begin{align*}
	F_Y(y) &= P(Y\le y) = P((X-a)/(b-a) \le y)\\
    	&= P(X \le (b-a)\,y+a) = F_X((b-a)\,y+a).
\end{align*}
Derivando os dois lados da equação em relação à $y$ obtemos que
\begin{align*}
	f_Y(y) =  (b-a)f_X((b-a)\,y+a) =
    \begin{cases}
    	1, &\text{se }y\in(0,1);\\
        0, &\text{caso contrário.}
    \end{cases}
\end{align*}
Logo, $Y$ é uniformemente distribuída em $(0,1)$.\\[1mm]
\textit{Observação:} outra opção é fazer $Y = (b-X)/(b-a)$.
\end{solution}

% 5.29 % % % % % % % % % % % % % % % % % % % % %
\setcounter{question}{28}
\question{
Seja $X$ uma variável aleatória contínua
com função distribuição cumulativa $F$.
Defina a variável aleatória $Y$ como $Y = F(X)$.
Mostre que $Y$ é uniformemente distribuída em $(0, 1)$.
}
\begin{solution}
	Por simplicidade, vamos supor que $F: \mathbb{R} \to [0,1]$ seja estritamente crescente. Logo, $F$ é inversível e para $y \in (0,1)$,
	\begin{align*}
		F_Y(y) = P(Y\le y) = P(F(X)\le y) = P(X \le F^{-1}(y)) = F(F^{-1}(y)) = y.
	\end{align*}
    Dessa forma, quando $y \in \mathbb{R}$,
    \begin{align*}
    	F_Y(y) =
        \begin{cases}
    	0, &\text{se }y \le 0;\\
        y, &\text{se }0 < y < 1;\\
        1, &\text{se }y \ge 1;\\
    	\end{cases}
    \end{align*}
    o que caracteriza uma distribuição uniforme em $(0,1)$.\\[1mm]
    \textit{Observação:} Isso também acontece quando $F$ não é inversível.
\end{solution}

% 5.30 % % % % % % % % % % % % % % % % % % % % %
%\setcounter{question}{28}
\question{
Suponha que $X$ tenha função densidade
de probabilidade $f_X$. Determine a função
densidade de probabilidade da variável
aleatória $Y$ definida como $Y = aX + b$.
}
\begin{solution}
	Seja $a>0$,
	\begin{align*}
		F_Y(y) = P(Y\le y) = P(aX+b\le y) = P(X\le (y-b)/a) = F_X((y-b)/a).
	\end{align*}
    Derivando ambos lados da equação em relação à $y$ leva à
    \begin{align*}
    	f_Y(y) = \frac{f_X((y-b)/a)}{a}.
    \end{align*}
    Por outro lado, se $a<0$, então
    \begin{align*}
    	F_Y(y) &= P(X\ge (y-b)/a) = 1-P(X<(y-b)/a) = 1-F_X([(y-b)/a]^-)\\
        	&= 1-F_X((y-b)/a) \quad\text{(a variável aleatória é contínua).}
    \end{align*}
    Novamente, derivando ambos lados da equação em relação à $y$ leva à
    \begin{align*}
    	f_Y(y) = \frac{f_X((y-b)/a)}{-a}.
    \end{align*}
    Portanto, quando $a\neq 0$,
    \begin{align*}
    	f_Y(y) = \frac{f_X((y-b)/a)}{|a|}.
    \end{align*}
\end{solution}

\end{questions}