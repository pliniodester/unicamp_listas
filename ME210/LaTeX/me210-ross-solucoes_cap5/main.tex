
\documentclass[answers, 12pt]{exam}
\usepackage{amsmath}
\usepackage{amsthm}
\usepackage{amsfonts}
\usepackage{amssymb}
\usepackage{mathrsfs}
\usepackage[brazil]{babel}
\usepackage[utf8]{inputenc}

\renewcommand{\qedsymbol}{$\blacksquare$}
\renewcommand{\thequestion}{\arabic{section}.\arabic{question}}
\renewcommand{\solutiontitle}{\noindent\textbf{Solução:}\enspace}

\footer{}{\thepage}{}

\title{	ME210 - Probabilidade I - 2S 2017\\
		{\large \textit{Docente}: Marina Vachkovskaia}\\[2mm]
		{\large Soluções para problemas selecionados do livro\\[-2mm]
        \textit{Probabilidade: Um curso moderno com aplicações}
        8.ed. de Sheldon Ross}\\
}
\author{Plínio Santini Dester (\url{p103806@dac.unicamp.br})}

\begin{document}

%% Content goes here
\maketitle

Em caso de dúvidas, sugestões ou correções (inclusive erros de digitação), não hesite em mandar um e-mail.

\setcounter{section}{4}
\section{Problemas}

\begin{questions}

% 5.1  % % % % % % % % % % % % % % % % % % % % %
%\setcounter{question}{0}
\question{Seja $X$ uma variável aleatória com função densidade de probabilidade
\begin{equation*}
	f(x) = 
    \begin{cases}
		c\,(1-x^2) 	&-1<x<1 \\
        0			&\text{caso contrário}
	\end{cases}
\end{equation*}
\begin{parts}
	\part Qual é o valor de $c$?
	\part Qual é a função distribuição cumulativa de $X$?
\end{parts}
}
\begin{solution}
\begin{parts}
	\part Sabemos que $\int_{-\infty}^\infty f(x)\,\diff x = 1$, logo
    \begin{align*}
    		&\int_{-1}^1 c\,(1-x^2)\,\diff x = 1\\
        \Rightarrow\, & c\,\left.\left( x-\frac{x^3}{3} \right)\right\rvert_{-1}^1
        	= 1\\
        \Rightarrow\, & c = \tfrac{3}{4}.
    \end{align*}
    
	\part Por definição, $F(x) = \int_{-\infty}^x f(x)\,\diff x$. Assim, se $x\in[-1,1]$, então
    \begin{align*}
    	F(x) &= \int_{-1}^x \tfrac{3}{4}(1-{x'}^2)\,\diff x'\\
        	&= \frac{3}{4}\left.\left( x'-\frac{x'^3}{3} \right)\right\rvert_{-1}^x\\
            &= \frac{2 + 3x - x^3}{4}.
    \end{align*}
    Logo,
    \begin{align*}
    	F(x) = 
        \begin{cases}
        	0,						& x \le -1;\\
            \frac{2 + 3x - x^3}{4},	& -1 < x \le 1;\\
            1,						& x > 1.
        \end{cases}
    \end{align*}
\end{parts}
\end{solution}

% 5.2  % % % % % % % % % % % % % % % % % % % % %
%\setcounter{question}{0}
\question{Um sistema formado por uma peça original mais uma sobressalente pode funcionar por uma quantidade de tempo aleatória $X$. Se a densidade de $X$ é dada, em
unidades de meses, por
\begin{equation*}
	f(x) = 
    \begin{cases}
		C\,x\euler^{-x/2} 	&x>0 \\
        0			&x\le 0
	\end{cases}
\end{equation*}
qual é a probabilidade de que o sistema
funcione por pelo menos 5 meses?
}
\begin{solution}
	Note que $X$ segue uma distribuição Gamma de parâmetros $\alpha = 2$ e $\lambda = 1/2$. Sabemos que $C = \lambda^\alpha/\Gamma(\alpha) = (1/2)^2/1! = 1/4$.\\    
    A probabilidade que o sistema funcione por pelo menos 5 meses é dada por
    \begin{align*}
    	\int_5^\infty \tfrac{1}{4}\,x\euler^{-x/2}\,\diff x
        = \left.\left(-\tfrac{1}{2}(x+2)\,\euler^{-x/2}\right)\right\rvert_5^\infty
        = \tfrac{7}{2}\,\euler^{-5/2}
        \approx \text{0,287}.
    \end{align*}
\end{solution}

% 5.3  % % % % % % % % % % % % % % % % % % % % %
%\setcounter{question}{0}
\question{Considere a função
\begin{equation*}
	f(x) = 
    \begin{cases}
		C\,(2x-x^3) &0<x<\tfrac{5}{2} \\
        0			&\text{caso contrário}
	\end{cases}
\end{equation*}
Poderia $f$ ser uma função densidade de
probabilidade? Caso positivo, determine
$C$.Repita considerando que a função $f(x)$
seja dada por
\begin{equation*}
	f(x) = 
    \begin{cases}
		C\,(2x-x^2) &0<x<\tfrac{5}{2} \\
        0			&\text{caso contrário}
	\end{cases}
\end{equation*}
}
\begin{solution}
	Em ambos casos $f$ não pode ser uma função densidade de probabilidade, pois a integral de $f$ em qualquer intervalo deve ser não-negativa, uma vez que representa uma probabilidade. Note que os pontos que estão em uma vizinhança de $x=1$ tem sinal oposto aqueles que estão em uma vizinhança de $x=5/2$. Logo, independente do sinal escolhido para $C$, em algum dos dois casos a integral será negativa e, portanto, $f$ não pode ser uma função densidade de probabilidade.
\end{solution}

% 5.5  % % % % % % % % % % % % % % % % % % % % %
\setcounter{question}{4}
\question{Um posto de gasolina é abastecido com
gasolina uma vez por semana. Se o volume semanal de vendas em milhares de litros é uma variável aleatória com função
densidade de probabilidade
\begin{equation*}
	f(x) = 
    \begin{cases}
		5\,(1-x)^4 &0<x<1 \\
        0			&\text{caso contrário}
	\end{cases}
\end{equation*}
qual deve ser a capacidade do tanque
para que a probabilidade do fornecimento não ser suficiente em uma dada semana seja de 0,01?
}
\begin{solution}
	Queremos descobrir $x_0$ para o qual as vendas serem superiores à $x_0$ tenha probabilidade 0,01, ou seja,
    \begin{align*}
    	\int_{x_0}^1 5\,(1-x)^4\,\diff x = \text{0,01}
        ~\Rightarrow~ (1-x_0)^5 = \text{0,01}
        ~\Rightarrow~ x_0 = \text{0,602}.
    \end{align*}
    Logo, a capacidade do tanque deve ser de 602 litros.
\end{solution}

% 5.7  % % % % % % % % % % % % % % % % % % % % %
\setcounter{question}{6}
\question{A função densidade de $X$ é dada por
\begin{equation*}
	f(x) = 
    \begin{cases}
		a+bx^2 &0\le x\le 1 \\
        0			&\text{caso contrário}
	\end{cases}
\end{equation*}
Se $E[X]=3/5$,determine $a$ e $b$.
}
\begin{solution}
\begin{align*}
	\int_0^1(a+bx^2)\,\diff x = 1 ~\Rightarrow~ a+b/3 = 1.
\end{align*}
\begin{align*}
	E[X] = \int_0^1x(a+bx^2)\,\diff x
    ~\Rightarrow~ a/2+b/4 = 3/5.
\end{align*}
Resolvendo essas duas equações para $a$ e $b$ obtemos $a=3/5$ e $b=6/5$.
\end{solution}

% 5.10 % % % % % % % % % % % % % % % % % % % % %
\setcounter{question}{9}
\question{Trens em direção ao destino $A$ chegam
na estação em intervalos de 15 minutos a
partir das 7:00 da manhã, enquanto trens
em direção ao destino $B$ chegam à estação em intervalos de 15 minutos começando as 7:05 da manhã.
\begin{parts}
	\part Se certo passageiro chega à estação em um horário uniformemente distribuído entre 7:00 e 8:00 da manhã e
pega o primeiro trem que chega, em
que proporção de tempo ele vai para
o destino $A$?
	\part E se o passageiro chegar em um horário uniformemente distribuído entre 7:10 e 8:10 da manhã?
\end{parts}
}
\begin{solution}
\begin{parts}
	\part Note que a pessoa pegará o trem $A$ se ela chegar nos intervalos de tempo {7:05-7:15}, {7:20-7:30}, {7:35-7:45} ou {7:50-8:00}. O que totaliza 40 minutos. Como o horário que a pessoa chega é distribuído uniformemente em 60 minutos, então a proporção de vezes que ela pega o trem $A$ é $\frac{40}{60} = \frac{2}{3}$.
    
	\part Nesse novo cenário, os intervalos são {7:10-7:15}, {7:20-7:30}, {7:35-7:45}, {7:50-8:00} ou {8:05-8:10}. O que totaliza 40 minutos também. Logo, a proporção de vezes que ela pega o trem $A$ continua sendo $\frac{2}{3}$.
\end{parts}
\end{solution}

% 5.11 % % % % % % % % % % % % % % % % % % % % %
%\setcounter{question}{9}
\question{Um ponto é escolhido aleatoriamente em
um segmento de reta de comprimento $L$.
Interprete este enunciado e determine a
probabilidade de que a relação entre o
segmento mais curto e o mais longo seja
menor que $1/4$.
}
\begin{solution}
	Seja $\ell$ o tamanho do comprimento mais curto. Queremos saber quando $\ell/(L-\ell) \le 1/4$, ou seja, quando $\ell \le \frac{L}{5}$. Seja $X$ uma variável aleatória uniformemente distribuída em $(0,L)$, então nosso problema é equivalente à
    \begin{align*}
    	P(X\in(0,\tfrac{L}{5})\cup(\tfrac{4L}{5},L))
        	= \frac{(L/5-0)+(L-4L/5)}{L}
            = \frac{2}{5}.
    \end{align*}
\end{solution}

% 5.13 % % % % % % % % % % % % % % % % % % % % %
\setcounter{question}{12}
\question{Você chega na parada de ônibus as 10:00,
sabendo que o ônibus chegará em algum
horário uniformemente distribuído entre
10:OO e 10:30.
\begin{parts}
	\part Qual é a probabilidade de que você tenha que esperar mais de 10 minutos?
    
    \part Se, as 10:15, o ônibus ainda não tiver
chegado, qual é a probabilidade de
que você tenha que esperar pelo menos mais 10 minutos?
\end{parts}
}
\begin{solution}
Seja $X$ uma variável aleatória uniformemente distribuída em $(0,30)$
\begin{parts}
	\part \[P(X>10) = \frac{30-10}{30} = \frac{2}{3}.\]
    
    \part \[P(X>25\mid X>15) = \frac{P(X>25,X>15)}{P(X>15)}
    	= \frac{P(X>25)}{P(X>15)} = \frac{5/30}{15/30} = \frac{1}{3}.\]
\end{parts} 
\end{solution}

% 5.14 % % % % % % % % % % % % % % % % % % % % %
%\setcounter{question}{12}
\question{Seja $X$ uma variável aleatória uniforme
no intervalo $(0,1)$. Calcule $E[X^n]$ usando
a Proposição~2.1 e depois verifique o resultado usando a definição de esperança.
}
\begin{solution}
	Usando a Proposição~2.1, temos
    \begin{align*}
    	E[X^n] = \int_{-\infty}^{\infty} x^n\,f_X(x)\,\diff x
        	= \int_0^1 x^n\,\diff x = \frac{1}{n+1}.
    \end{align*}
    Usando a definição, primeiro é necessário encontrar a distribuição de $Y=X^n$.
    \begin{align*}
    	F_Y(x) = P(Y\le x) = P(X^n\le x) = P(X\le x^{1/n}) = F_X(x^{1/n}).
    \end{align*}
    Derivando ambos lados da equação, encontramos a densidade,
    \begin{align*}
    	f_Y(x) = \frac{x^{(1-n)/n}}{n} f_X(x^{1/n})
        	= \frac{x^{(1-n)/n}}{n}\quad 0<x<1.
    \end{align*}
    Logo,
    \begin{align*}
    	E[X^n] = E[Y] = \int_{-\infty}^{\infty} x\,f_Y(x)\,\diff x
        	= \int_0^1 x\frac{x^{(1-n)/n}}{n}\,\diff x
        	= \int_0^1 \frac{x^{1/n}}{n}\,\diff x = \frac{1}{n+1}.
    \end{align*}
\end{solution}

% 5.15 % % % % % % % % % % % % % % % % % % % % %
%\setcounter{question}{12}
\question{Se $X$ é uma variável aleatória normal com
parâmetros $\mu = 10$ e $\sigma^2 = 36$, calcule
\begin{parts}
	\part $P(X>5)$;
    \part $P(4<X<16)$;
	\part $P(X<8)$;
    \part $P(X<20)$;
    \part $P(X>16)$.
\end{parts}
}
\begin{solution}
Primeiramente, devemos colocar $X$ em função de uma variável aleatória normal padrão para podermos usar a Tabela~5.1. \\
Assim, $X = \mu+\sigma Z = 10 + 6Z$, onde $Z \sim \mathcal{N}(0,1)$.
\begin{parts}
	\part $P(X>5) = P(10+6Z > 5) = P(Z > -5/6) = 1-\Phi(-5/6)\\
    	{}\quad\quad\quad = \Phi(5/6) \approx \text{0,7967}$.
            
    \part $P(4<X<16) = P(X<16)-P(X<4) = P(Z<1)-P(Z<-1)\\
    	{}\quad\quad\quad = \Phi(1)-\Phi(-1)
        	= 2\Phi(1)-1 \approx \text{0,6826}$.
    
	\part $P(X<8) = P(Z<-1/3) = \Phi(-1/3)	= 1-\Phi(1/3) \approx\text{0,3707}$.

    \part $P(X<20) = P(Z<5/3) = \Phi(5/3) \approx \text{0,9515}$.
    
    \part $P(X>16) = P(Z>1) = 1-P(Z<1) = 1-\Phi(1) \approx\text{0,1587}$.

\end{parts}
\end{solution}

% 5.16 % % % % % % % % % % % % % % % % % % % % %
%\setcounter{question}{12}
\question{O volume anual de chuvas (em mm) em
certa região é normalmente distribuído
com $\mu = 40$ e $\sigma = 4$. Qual é a probabilidade
de que, a contar deste ano, sejam necessários
mais de 10 anos antes que o volume
de chuva em um ano supere 50 mm? Que
hipóteses você está adotando?
}
\begin{solution}
Seja $p$ a probabilidade que o volume de chuva não supere 50 mm em um ano e $Z$ uma variável aleatória normal padrão. Então,
\begin{align*}
	p = P(Z < \tfrac{50-40}{4}) = P(Z<5/2) = \Phi(5/2)
    	\approx \text{0,9938}.
\end{align*}
Se o volume de chuva de cada ano forem independentes, então a probabilidade que demore mais de 10 anos para superar os 50 mm em um ano é dado por $p^{10}\approx \text{0,9397}$.
\end{solution}

% 5.17 % % % % % % % % % % % % % % % % % % % % %
%\setcounter{question}{12}
\question{Um homem praticando tiro ao alvo 
recebe 10 pontos se o tiro estiver a 1 cm do
alvo, 5 pontos se estiver entre 1 e 3 cm do
alvo, e 3 pontos se estiver entre 3 e 5 cm
do alvo. Determine o número esperado
de pontos que ele receberá se a distância
do ponto de tiro até o alvo for uniformemente distribuída entre 0 e 10.
}
\begin{solution}
	Seja $g$ a função que associa a distância ao alvo à pontuação recebida e $X$ a distância ao alvo, então queremos saber
    \begin{align*}
    	E[g(X)]
        	&= \int_0^{10} g(x) f_X(x)\,\diff x \\
        	&= \int_{0}^1 10\tfrac{1}{10} \,\diff x
            	+\int_{1}^3 5\tfrac{1}{10} \,\diff x
                +\int_{3}^5 3\tfrac{1}{10} \,\diff x \\
            &= (1-0) 1 + (3-1) \tfrac{1}{2} + (5-3) \tfrac{3}{10}\\
            &= \frac{13}{5} = \text{2,6}.
    \end{align*}
\end{solution}

% 5.18 % % % % % % % % % % % % % % % % % % % % %
%\setcounter{question}{12}
\question{Suponha que $X$ seja uma variável aleatória
normal com média 0,5. Se $P(X > 9) = \text{0,2}$,
qual é o valor de $\Var(X)$, aproximadamente?
}
\begin{solution}
	Seja $Z$ uma variável aleatória normal padrão, então
    \[\text{0,2} = P(X>9) = P(Z>\text{8,5}/\sigma) = 1-\Phi(\text{8,5}/\sigma),\]
    Logo, $\Phi(\text{8,5}/\sigma) = \text{0,8}$. Da Tabela~5.1, obtemos que $\text{8,5}/\sigma \approx \text{0,84}$ e, portanto, $\sigma \approx \text{10,12}$ e $\Var(X) = \sigma^2 \approx \text{102,4}$.
\end{solution}

% 5.19 % % % % % % % % % % % % % % % % % % % % %
%\setcounter{question}{12}
\question{Seja $X$ uma variável aleatória normal com
média 12 e variância 4. Determine o valor
de $c$ tal que $P(X > c) = 0,1$.
}
\begin{solution}
	Seja $Z$ uma variável aleatória normal padrão, então
    \[\text{0,1} = P(X>c) = P(Z>(c-12)/2) = 1-\Phi((c-12)/2),\]
    Logo, $\Phi((c-12)/2) = \text{0,9}$. Da Tabela~5.1, obtemos que $(c-12)/2 \approx \text{1,28}$ e, portanto, $c \approx \text{14,56}$.
\end{solution}

% 5.20 % % % % % % % % % % % % % % % % % % % % %
%\setcounter{question}{12}
\question{Se 65\% da população de uma grande
comunidade são a favor de um aumento
proposto para as taxas escolares, obtenha
uma aproximação para a probabilidade
de que uma amostra aleatória de 100 pessoas contenha
\begin{parts}
	\part pelo menos 50 pessoas a favor da proposta;
    \part entre 60 e 70 pessoas (inclusive) a favor;
    \part menos de 75 pessoas a favor.
\end{parts}
}
\begin{solution}
A variável aleatória $N$ que representa o número de pessoas à favor é uma binomial de parâmetros $p=\text{0,65}$ e $n=100$. Porém, como $n$ é grande, podemos aproximar por uma normal $X$ de média $\mu = np = 65$ e $\sigma^2 = np(1-p) = \text{22,75}$ (Teorema limite de DeMoivre e Laplace). Seja $Z$ uma normal padrão.
\begin{parts}
	\part Usando \textit{correção de continuidade}, temos que
    \begin{align*}
    	P(N \ge 50) 
        	&\approx P(X\ge \text{49,5})
        		= 1-P(Z < \tfrac{\text{49,5}-65}{\sqrt{\text{22,75}}})\\
        	&= 1-\Phi(-\text{3,25}) = \Phi(\text{3,25}) \approx \text{0,9994}.
    \end{align*}
    
    \part Usando \textit{correção de continuidade}, temos que
    \begin{align*}
    	P(60\le N \le 70) 
        	&\approx P(\text{59,5}\le X\le \text{70,5})\\
        	&= P(Z < \tfrac{\text{70,5}-65}{\sqrt{\text{22,75}}})
            	- P(Z < \tfrac{\text{59,5}-65}{\sqrt{\text{22,75}}})\\
        	&= \Phi(\text{1,15})-\Phi(-\text{1,15}) = 2\Phi(\text{1,15})-1\\
            &\approx \text{0,7498}.
    \end{align*}
    
    \part Usando \textit{correção de continuidade}, temos que
    \begin{align*}
    	P(N < 75) 
        	\approx P(X < \text{74,5})
        	= P(Z < \tfrac{\text{74,5}-65}{\sqrt{\text{22,75}}})
        	= \Phi(\text{1,99}) \approx \text{0,9767}.
    \end{align*}
\end{parts}
\end{solution}

% 5.23 % % % % % % % % % % % % % % % % % % % % %
\setcounter{question}{22}
\question{Realizam-se mil jogadas independentes
de um dado honesto. Calcule a probabilidade
aproximada de que o número 6
apareça entre 150 e 200 vezes, inclusive.
Se o número 6 aparecer exatamente 200
vezes, determine a probabilidade de que
o número 5 apareça menos de 150 vezes.
}
\begin{solution}
	Seja $N$ uma variável aleatória com distribuição binomial $n$ e $p$. Quando $n$ é grande, podemos aproximar por uma variável aleatória $X$ distribuída como normal de parâmetros $\mu=np$ e $\sigma^2=np(1-p)$. Seja $Z$ a normal padrão.\\
    Para o número 6 aparecer entre 150 e 200 vezes em mil jogadas, temos que $n=1000$, $p=1/6$ e queremos saber
    \begin{align*}
    	P(150\le N\le 200)
        	&\approx P(\text{149,5} \le X \le \text{200,5})
            	= P(\tfrac{\text{200,5}-\text{166,7}}{\text{11,785}} \le Z 
            	\le \tfrac{\text{149,5}-\text{166,7}}{\text{11,785}})\\
            &= \Phi(\text{2,87}) - \Phi(-\text{1,5})
            	= \Phi(\text{2,87})+\Phi(\text{1,5})-1 \approx \text{0,9311}.
    \end{align*}
    Por outro lado, se sabemos que já saíram exatamente 200 dados marcando 6, sobram 800 dados para saírem números 5, agora com 1/5 de chance. Fazendo exatamente o mesmo procedimento feito acima, mas com $n=800$ e $p=1/5$, chegamos em $P(N<150)\approx P(X<\text{149,5})\approx \text{0,1762}$.
\end{solution}

% 5.28 % % % % % % % % % % % % % % % % % % % % %
\setcounter{question}{27}
\question{Em 10.000 jogadas independentes de uma
moeda, observou-se que deu cara 5800
vezes. É razoável supor que essa moeda
não seja honesta? Explique.
}
\begin{solution}
	Seja $N$ uma variável aleatória com distribuição binomial $n$ e $p$. Quando $n$ é grande, podemos aproximar por uma variável aleatória $X$ distribuída como normal de parâmetros $\mu=np$ e $\sigma^2=np(1-p)$. Seja $Z$ a normal padrão.\\
    Se a moeda for honesta, então $n=10\,000$, $p=1/2$. Nesse caso, não é necessário correção de continuidade, pois $n$ é realmente grande. É interessante saber a seguinte probabilidade
    \begin{align*}
    	P(N\ge 5\,800)
        	&\approx P( X \ge \text{5800} )
            	= P(Z \ge \tfrac{5800-5000}{\sqrt{2500}})\\
            &= 1-\Phi(16) \approx 0.
    \end{align*}
    Dessa forma, como a chance de pelo menos $5\,800$ caras em $10\,000$ lançamentos é praticamente nula, então é bem razoável supor que a moeda não seja honesta.
\end{solution}

% 5.31 % % % % % % % % % % % % % % % % % % % % %
\setcounter{question}{30}
\question{
\begin{parts}
	\part Uma estação de bombeiros deve ser
instalada ao longo de uma estrada
com comprimento $A$, $A < \infty$. Se incêndios
ocorrem em pontos uniformemente distribuídos
no intervalo $(0,A)$, qual deveria ser a localização da
estação de forma a minimizar-se a distância
esperada para o incêndio? Isto
é, escolha $a$ de forma que $E[|X-a|]$ seja 
minimizado quando $X$ for uniformemente
distribuído ao longo de $(0,A)$.
    \part Agora suponha que a estrada tenha
comprimento infinito -- indo do ponto
0 até $\infty$. Se a distância de um incêndio até o ponto 0 é exponencialmente distribuída com taxa $\lambda$, onde deveria
estar localizada a estação? Isto é, queremos
minimizar $E[|X - a|]$, onde $X$ é
agora exponencial com taxa $\lambda$.
\end{parts}
}
\begin{solution}
\begin{parts}
	\part Comecemos calculando a esperança em função de $a\in[0,A]$,
    \begin{align*}
    	E[|X-a|] 
        	&= \int_{-\infty}^\infty |x-a| f_X(x)\,\mathrm{d}x \\
        	&= \int_0^A |x-a| \frac{1}{A}\,\mathrm{d}x \\
            &= \int_0^a \frac{(a-x)}{A}\mathrm{d}x + \int_a^A \frac{(x-a)}{A}\mathrm{d}x\\
			&= \frac{a^2}{A} - a + \frac{A}{2}.
    \end{align*}
    Para encontrar o ponto de mínimo, derivamos a expressão acima em relação à $a$ e igualamos à zero, encontrando $a=A/2$. É um ponto de mínimo global, pois a segunda derivada é positiva em toda região de interesse (função é convexa).
    
    \part Analogamente,
    \begin{align*}
    	E[|X-a|] 
        	&= \int_{-\infty}^\infty |x-a| f_X(x)\,\mathrm{d}x \\
        	&= \int_0^\infty |x-a| \lambda\euler^{-\lambda x}\,\mathrm{d}x \\
            &= \int_0^a (a-x)\lambda\euler^{-\lambda x}\mathrm{d}x 
            	+ \int_a^\infty (x-a)\lambda\euler^{-\lambda x}\mathrm{d}x\\
			&= a + \frac{2\,\euler^{-\lambda a}-1}{\lambda}.
    \end{align*}
\end{parts}
	Novamente, derivamos em relação à $a$ e igualamos à zero,
    \begin{align*}
    	1 - 2\euler^{-\lambda a} = 0 \Rightarrow a = \frac{\ln(2)}{\lambda}.
    \end{align*}
    É um ponto de mínimo global, pois a segunda derivada da função é positiva em toda região de interesse (é convexa).
\end{solution}

% 5.32  % % % % % % % % % % % % % % % % % % % % %
%\setcounter{question}{36}
\question{O tempo (em horas) necessário para a
manutenção de uma máquina é uma variável aleatória
exponencialmente distribuída com $\lambda = 1/2$. Qual é
\begin{parts}
	\part a probabilidade de que um reparo
dure mais que 2 horas?
    \part a probabilidade condicional de que o
tempo de reparo dure pelo menos 10
horas, dado que a sua duração seja superior a 9 horas?
\end{parts}
}
\begin{solution}
Seja $X$ o tempo para manutenção da máquina, então $f_X(x) = \frac{1}{2} \euler^{-x/2}$.
\begin{parts}
	\part 
    \begin{align*}
    	P(X>2) 
        	= \int_{2}^\infty \frac{1}{2}\euler^{-x/2}\,\mathrm{d}x
        	= \euler^{-1} \approx \text{0,368}.
    \end{align*}
    
    \part 
    \begin{align*}
    	P(X>10\mid X>9) = \frac{P(X>10)}{P(X>9)} = \frac{\euler^{-10/2}}{\euler^{-9/2}}
        	= \euler^{-1/2} \approx \text{0,607}.
    \end{align*}
    A exponencial não tem memória, o problema é equivalente à $P(X>1)$.
\end{parts}
\end{solution}


% 5.37  % % % % % % % % % % % % % % % % % % % % %
\setcounter{question}{36}
\question{Se a variável aleatória $X$ é uniformemente distribuída ao longo do intervalo $(-1,1)$, determine:
\begin{parts}
	\part $P(|X| > 1/2)$;
    
    \part a função densidade da variável aleatória $|X|$.
\end{parts}
}
\begin{solution}
\begin{parts}
	\part
    \begin{align*}
    	P(|X| > 1/2) = P(X > 1/2) + P(X < -1/2) = 1/4+1/4 = 1/8.
    \end{align*}
    
    \part A função distribuição acumulada de $|X|$ é dada por
    \begin{align*}
    	F_{|X|}(x) = P(|X| \le x) = P(-x \le X \le x)
        	= F_X(x) - F_X(-x),
    \end{align*}
    quando $x > 0$. Por outro lado, $F_{|X|}(x) = 0$, se $x<0$.\\
    Derivando dos dois lados de cada equação, obtemos a densidade de $|X|$,
    \begin{align*}
    	f_{|X|}(x) = f_X(x) + f_X(-x) = 1, \quad &0<x<1,
    \end{align*}
    e $f_{|X|}(x) = 0$, caso contrário.
\end{parts}
\end{solution}

% 5.38  % % % % % % % % % % % % % % % % % % % % %
%\setcounter{question}{36}
\question{Se a variável aleatória $Y$ é uniformemente distribuída ao longo do intervalo $(0,5)$, qual é a probabilidade de que as raízes da
equação $4x^2 + 4xY + Y + 2 = 0$ sejam ambas reais?
}
\begin{solution}
	Para uma equação do segundo grau do tipo $ax^2+bx+c=0$ ter raízes reais é necessário e suficiente que $b^2-4ac\ge 0$. Logo, queremos saber
    \begin{align*}
    	P((4Y)^2-4\cdot4\,(Y+2) \ge 0)
        	&= P(Y^2-Y-2\ge 0) \\
        	&= P( (Y-2)(Y+1) \ge 0 )\\
            &= P(Y\le -1) + P(Y \ge 2)\\
            &= 0 + \frac{5-2}{5} = \frac{3}{5}.
    \end{align*}
\end{solution}

% 5.39  % % % % % % % % % % % % % % % % % % % % %
%\setcounter{question}{36}
\question{Se $X$ é uma variável aleatória exponencial
com parâmetro $\lambda = 1$, calcule a função
densidade de probabilidade da variável
aleatória $Y$ definida como $Y = \log X$.
}
\begin{solution}
	Comecemos pela função distribuição acumulada
    \begin{align*}
    	F_Y(x) = P(Y \le x) = P(\log X \le x) = P(X \le \euler^x) = F_X(\euler^x).
    \end{align*}
    Derivando ambos lados da equação em relação à $x$, obtemos que
    \begin{align*}
    	f_Y(x) = \euler^x\,f_x(\euler^x)
        	= \lambda \exp(x-\lambda\euler^x) = \exp(x-\euler^x).
    \end{align*}
\end{solution}

% 5.40  % % % % % % % % % % % % % % % % % % % % %
%\setcounter{question}{36}
\question{Se $X$ é uniformemente distribuída ao longo do intervalo $(0, 1)$, determine a função densidade de $Y = \euler^X$.
}
\begin{solution}
	Comecemos pela função distribuição acumulada ($x>0$)
    \begin{align*}
    	F_Y(x) = P(Y \le x) = P(\euler^X \le x) = P(X \le \log(x)) = F_X(\log(x)).
    \end{align*}
    Derivando ambos lados da equação em relação à $x$, obtemos que
    \begin{align*}
    	f_Y(x) = \frac{f_x(\log(x))}{x} = 
        \begin{cases}
        	\frac{1}{x},	&\quad 1<x<\euler;\\
            0,				&\quad \text{caso contrário}.
        \end{cases}
    \end{align*}
\end{solution}

% 5.41  % % % % % % % % % % % % % % % % % % % % %
%\setcounter{question}{36}
\question{Determine a distribuição de $R = A \sin\theta$,
onde $A$ é uma constante fixa, e $\theta$ é
uma variável aleatória uniformemente
distribuída em $(-\pi/2, \pi/2)$. A variável
aleatória $R$ surge da teoria da balística.
Se um projétil é disparado de sua origem
com um ângulo $\alpha$ em relação à superfície
da terra com uma velocidade $v$, então o
ponto $R$ no qual ele retorna à terra pode
ser escrito como $R = (v^2/g)\sin 2\alpha$, onde
$g$ é a aceleração da gravidade,que é igual
a 9,8 m/s$^2$.
}
\begin{solution}
	Comecemos pela função distribuição acumulada
    \begin{align*}
    	F_R(r) = P(R \le r) = P(A\sin\theta \le r) = P(\theta \le \arcsin(r/A) )
        	= F_\theta(\arcsin(r/A)).
    \end{align*}
    Derivando ambos lados da equação em relação à $r$, obtemos que
    \begin{align*}
    	f_R(r) = \frac{f_\theta(\arcsin(r/A))}{\sqrt{A^2-r^2}} = 
        \begin{cases}
        	\frac{1}{\pi\sqrt{A^2-r^2}},	&\quad -A<r<A;\\
            0,								&\quad \text{caso contrário}.
        \end{cases}
    \end{align*}
\end{solution}

\end{questions}

\newpage

\setcounter{section}{4}
\section{Exercícios Teóricos}
\begin{questions}

% 5.27 % % % % % % % % % % % % % % % % % % % % %
\setcounter{question}{26}
\question{
Se $X$ é uniformemente distribuída em $(a, b)$, qual variável aleatória que varia linearmente com $X$ é uniformemente distribuída em $(0, 1)$?
}
\begin{solution}
Seja $Y = (X-a)/(b-a)$, então
\begin{align*}
	F_Y(y) &= P(Y\le y) = P((X-a)/(b-a) \le y)\\
    	&= P(X \le (b-a)\,y+a) = F_X((b-a)\,y+a).
\end{align*}
Derivando os dois lados da equação em relação à $y$ obtemos que
\begin{align*}
	f_Y(y) =  (b-a)f_X((b-a)\,y+a) =
    \begin{cases}
    	1, &\text{se }y\in(0,1);\\
        0, &\text{caso contrário.}
    \end{cases}
\end{align*}
Logo, $Y$ é uniformemente distribuída em $(0,1)$.\\[1mm]
\textit{Observação:} outra opção é fazer $Y = (b-X)/(b-a)$.
\end{solution}

% 5.29 % % % % % % % % % % % % % % % % % % % % %
\setcounter{question}{28}
\question{
Seja $X$ uma variável aleatória contínua
com função distribuição cumulativa $F$.
Defina a variável aleatória $Y$ como $Y = F(X)$.
Mostre que $Y$ é uniformemente distribuída em $(0, 1)$.
}
\begin{solution}
	Por simplicidade, vamos supor que $F: \mathbb{R} \to [0,1]$ seja estritamente crescente. Logo, $F$ é inversível e para $y \in (0,1)$,
	\begin{align*}
		F_Y(y) = P(Y\le y) = P(F(X)\le y) = P(X \le F^{-1}(y)) = F(F^{-1}(y)) = y.
	\end{align*}
    Dessa forma, quando $y \in \mathbb{R}$,
    \begin{align*}
    	F_Y(y) =
        \begin{cases}
    	0, &\text{se }y \le 0;\\
        y, &\text{se }0 < y < 1;\\
        1, &\text{se }y \ge 1;\\
    	\end{cases}
    \end{align*}
    o que caracteriza uma distribuição uniforme em $(0,1)$.\\[1mm]
    \textit{Observação:} Isso também acontece quando $F$ não é inversível.
\end{solution}

% 5.30 % % % % % % % % % % % % % % % % % % % % %
%\setcounter{question}{28}
\question{
Suponha que $X$ tenha função densidade
de probabilidade $f_X$. Determine a função
densidade de probabilidade da variável
aleatória $Y$ definida como $Y = aX + b$.
}
\begin{solution}
	Seja $a>0$,
	\begin{align*}
		F_Y(y) = P(Y\le y) = P(aX+b\le y) = P(X\le (y-b)/a) = F_X((y-b)/a).
	\end{align*}
    Derivando ambos lados da equação em relação à $y$ leva à
    \begin{align*}
    	f_Y(y) = \frac{f_X((y-b)/a)}{a}.
    \end{align*}
    Por outro lado, se $a<0$, então
    \begin{align*}
    	F_Y(y) &= P(X\ge (y-b)/a) = 1-P(X<(y-b)/a) = 1-F_X([(y-b)/a]^-)\\
        	&= 1-F_X((y-b)/a) \quad\text{(a variável aleatória é contínua).}
    \end{align*}
    Novamente, derivando ambos lados da equação em relação à $y$ leva à
    \begin{align*}
    	f_Y(y) = \frac{f_X((y-b)/a)}{-a}.
    \end{align*}
    Portanto, quando $a\neq 0$,
    \begin{align*}
    	f_Y(y) = \frac{f_X((y-b)/a)}{|a|}.
    \end{align*}
\end{solution}

\end{questions}
%\newpage

\vspace{10mm} {\LARGE \textbf{Desafio!}}
\begin{enumerate}
\item Um investidor comprou uma ação muito instável. A cada mês, o valor dessa ação segue uma distribuição uniforme no intervalo $(0,1000)$ e é independente dos meses anteriores. O investidor pode vender a ação quando quiser, porém a cada mês que passa o dinheiro, para o investidor, vale $d$ vezes o mês anterior ($0<d<1$). Qual estratégia ele deve seguir para maximizar o retorno esperado na venda dessa ação? Se $d=4/5$, qual deve ter sido o valor máximo pago na compra da ação para que o investidor tenha um valor esperado de lucro positivo?
\end{enumerate}

\end{document}