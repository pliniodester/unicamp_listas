
\documentclass[answers, 12pt]{exam}
\usepackage{amsmath}
\usepackage{amsthm}
\usepackage{amsfonts}
\usepackage{amssymb}
\usepackage{mathrsfs}
\usepackage[brazil]{babel}
\usepackage[utf8]{inputenc}

\renewcommand{\qedsymbol}{$\blacksquare$}
\renewcommand{\thequestion}{\arabic{section}.\arabic{question}}
\renewcommand{\solutiontitle}{\noindent\textbf{Solução:}\enspace}

\footer{}{\thepage}{}

\title{	ME210 - Probabilidade I - 2S 2017\\
		{\large \textit{Docente}: Marina Vachkovskaia}\\[2mm]
		{\large Soluções para problemas selecionados do livro\\[-2mm]
        \textit{Probabilidade: Um curso moderno com aplicações}
        8.ed. de Sheldon Ross}\\
}
\author{Plínio Santini Dester (\url{p103806@dac.unicamp.br})}

\begin{document}

%% Content goes here
\maketitle

Em caso de dúvidas, sugestões ou correções (inclusive erros de digitação), não hesite em mandar um e-mail.

\setcounter{section}{5}
\section{Problemas}

\begin{questions}

% 6.1  % % % % % % % % % % % % % % % % % % % % %
% \setcounter{question}{6}
\question{Dois dados honestos são rolados. Determine a função de probabilidade conjunta de $X$ e $Y$ quando
\begin{parts}
	\part $X$ é o maior valor obtido em um dado
e $Y$ é a soma dos valores;
    \part $X$ é o valor no primeiro dado e $Y$ é o
maior dos dois valores;
    \part $X$ é o menor e $Y$ é o maior valor obtido com os dados.
\end{parts}
}
\begin{solution}
\begin{parts}
	\part
    \begin{align*}
    \begin{array}{c|ccccccccccc|l}
	_{\large X}\backslash^{\large Y} & 2 & 3 & 4 & 5 & 6 & 7 & 8 & 9 & 10 & 11 & 12 \\
	\hline
	1 & 1/36 & 0 & 0 & 0 & 0 & 0 & 0 & 0 & 0 & 0 & 0 & 1/36  \\
	2 & 0 & 2/36 & 1/36 & 0 & 0 & 0 & 0 & 0 & 0 & 0 & 0 & 3/36  \\
    3 & 0 & 0 & 2/36 & 2/36 & 1/36 & 0 & 0 & 0 & 0 & 0 & 0 & 5/36  \\
    4 & 0 & 0 & 0 & 2/36 & 2/36 & 2/36 & 1/36 & 0 & 0 & 0 & 0 & 7/36  \\
    5 & 0 & 0 & 0 & 0 & 2/36 & 2/36 & 2/36 & 2/36 & 1/36 & 0 & 0 & 9/36  \\
    6 & 0 & 0 & 0 & 0 & 0 & 2/36 & 2/36 & 2/36 & 2/36 & 2/36 & 1/36 & 11/36  \\
	\hline
	& 1/36 & 2/36 & 3/36 & 4/36 & 5/36 & 6/36 & 5/36 & 4/36 & 3/36 & 2/36 & 1/36
	\end{array}
    \end{align*}
    
    \part 
    \begin{align*}
    \begin{array}{c|cccccc|l}
	_{\large X}\backslash^{\large Y} & 1 & 2 & 3 & 4 & 5 & 6 \\
	\hline
	1 & 1/36 & 1/36 & 1/36 & 1/36 & 1/36 & 1/36 & 6/36  \\
	2 & 0	 & 2/36 & 1/36 & 1/36 & 1/36 & 1/36 & 6/36  \\
    3 & 0    & 0    & 3/36 & 1/36 & 1/36 & 1/36 & 6/36  \\
    4 & 0    & 0    & 0    & 4/36 & 1/36 & 1/36 & 6/36  \\
    5 & 0    & 0    & 0    & 0    & 5/36 & 1/36 & 6/36  \\
    6 & 0    & 0    & 0    & 0    & 0    & 6/36 & 6/36  \\
	\hline
	& 1/36 & 3/36 & 5/36 & 7/46 & 9/36 & 11/36 & \\
	\end{array}
    \end{align*}
    
    \part 
    \begin{align*}
    \begin{array}{c|cccccc|l}
	_{\large X}\backslash^{\large Y} & 1 & 2 & 3 & 4 & 5 & 6 \\
	\hline
	1 & 1/36 & 2/36 & 2/36 & 2/36 & 2/36 & 2/36 & 11/36  \\
	2 & 0	 & 1/36 & 2/36 & 2/36 & 2/36 & 2/36 & 9/36  \\
    3 & 0    & 0    & 1/36 & 2/36 & 2/36 & 2/36 & 7/36  \\
    4 & 0    & 0    & 0    & 1/36 & 2/36 & 2/36 & 5/36  \\
    5 & 0    & 0    & 0    & 0    & 1/36 & 2/36 & 3/36  \\
    6 & 0    & 0    & 0    & 0    & 0    & 1/36 & 1/36  \\
	\hline
	& 1/36 & 3/36 & 5/36 & 7/46 & 9/36 & 11/36 & \\
	\end{array}
    \end{align*}
\end{parts}
\end{solution}

% 6.2  % % % % % % % % % % % % % % % % % % % % %
% \setcounter{question}{6}
\question{Suponha que 3 bolas sejam sorteadas
sem reposição de uma urna consistindo
em 5 bolas brancas e 8 bolas vermelhas.
Considere $X_i = 1$ caso a $i$-ésima bola selecionada seja branca e $X_i=0$ caso contrário. Dê a função de probabilidade conjunta de
\begin{parts}
	\part $X_1,X_2$;
    \part $X_1,X_2,X_3$.
\end{parts}
}
\begin{solution}
Seja, $p(i,j,k) \triangleq P(X_1=i,X_2=j,X_3=k)$, então
\begin{parts}
	\part 
    \begin{align*}
    	p(0,0) &= \frac{8}{13}\frac{7}{12},& p(1,0) &= \frac{5}{13}\frac{8}{12},\\
        p(0,1) &= \frac{8}{13}\frac{5}{12},& p(1,1) &= \frac{5}{13}\frac{4}{12}.\\
    \end{align*}

	\part 
    \begin{align*}
    	p(0,0,0) &= \frac{8}{13}\frac{7}{12}\frac{6}{11},&
        p(0,0,1) &= \frac{8}{13}\frac{7}{12}\frac{5}{11},\\
        p(0,1,0) &= \frac{8}{13}\frac{5}{12}\frac{7}{11},&
        p(0,1,1) &= \frac{8}{13}\frac{5}{12}\frac{4}{11},\\
        p(1,0,0) &= \frac{5}{13}\frac{8}{12}\frac{7}{11},&
        p(1,0,1) &= \frac{5}{13}\frac{8}{12}\frac{4}{11},\\
        p(1,1,0) &= \frac{5}{13}\frac{4}{12}\frac{8}{11},&
        p(1,1,1) &= \frac{5}{13}\frac{4}{12}\frac{3}{11}.
    \end{align*}
    
\end{parts}
\end{solution}

% 6.3  % % % % % % % % % % % % % % % % % % % % %
% \setcounter{question}{6}
\question{No Problema 6.2, suponha que as bolas brancas sejam numeradas e considere $Y_i = 1$ se a $i$-ésima bola branca for selecionada, e 0 caso contrário. Determine a função de probabilidade conjunta de
\begin{parts}
	\part $Y_1,Y_2$;
    \part $Y_1,Y_2,Y_3$.
\end{parts}
}
\begin{solution}
Seja, $p(i,j,k) \triangleq P(Y_1=i,Y_2=j,Y_3=k)$, então
\begin{parts}
	\part 
    \begin{align*}
    	p(0,0) &= \frac{11}{13}\frac{10}{12}\frac{9}{11},
        	& p(1,0) &= 3\frac{1}{13}\frac{11}{12}\frac{10}{11},\\
        p(0,1) &= 3\frac{1}{13}\frac{11}{12}\frac{10}{11},
        	& p(1,1) &= 3\frac{2}{13}\frac{1}{12}\frac{11}{11}.\\
    \end{align*}

	\part 
    \begin{align*}
    	p(0,0,0) &= \frac{10}{13}\frac{9}{12}\frac{8}{11},&
        p(0,0,1) &= 3\frac{10}{13}\frac{9}{12}\frac{1}{11},\\
        p(0,1,0) &= 3\frac{10}{13}\frac{9}{12}\frac{1}{11},&
        p(0,1,1) &= 3\frac{10}{13}\frac{2}{12}\frac{1}{11},\\
        p(1,0,0) &= 3\frac{10}{13}\frac{9}{12}\frac{1}{11},&
        p(1,0,1) &= 3\frac{10}{13}\frac{2}{12}\frac{1}{11},\\
        p(1,1,0) &= 3\frac{10}{13}\frac{2}{12}\frac{1}{11},&
        p(1,1,1) &= \frac{3}{13}\frac{2}{12}\frac{1}{11}.
    \end{align*}
    
\end{parts}
\end{solution}

% 6.4  % % % % % % % % % % % % % % % % % % % % %
% \setcounter{question}{6}
\question{Repita o Problema 6.2 quando a bola selecionada é recolocada na urna antes da
próxima seleção.
}
\begin{solution}
Seja, $p(i,j,k) \triangleq P(X_1=i,X_2=j,X_3=k)$, então
\begin{parts}
	\part 
    \begin{align*}
    	p(0,0) &= \frac{8}{13}\frac{8}{13},& p(1,0) &= \frac{5}{13}\frac{8}{13},\\
        p(0,1) &= \frac{8}{13}\frac{5}{13},& p(1,1) &= \frac{5}{13}\frac{5}{13}.\\
    \end{align*}

	\part 
    \begin{align*}
    	p(0,0,0) &= \frac{8}{13}\frac{8}{13}\frac{8}{13},&
        p(0,0,1) &= \frac{8}{13}\frac{8}{13}\frac{5}{13},\\
        p(0,1,0) &= \frac{8}{13}\frac{5}{13}\frac{8}{13},&
        p(0,1,1) &= \frac{8}{13}\frac{5}{13}\frac{5}{13},\\
        p(1,0,0) &= \frac{5}{13}\frac{8}{13}\frac{8}{13},&
        p(1,0,1) &= \frac{5}{13}\frac{8}{13}\frac{5}{13},\\
        p(1,1,0) &= \frac{5}{13}\frac{5}{13}\frac{8}{13},&
        p(1,1,1) &= \frac{5}{13}\frac{5}{13}\frac{5}{13}.
    \end{align*}
    
\end{parts}
\end{solution}

% 6.5  % % % % % % % % % % % % % % % % % % % % %
% \setcounter{question}{6}
\question{Repita o Problema 6.3 (a) quando a bola selecionada é recolocada na urna antes da
próxima seleção
}
\begin{solution}
Seja, $p(i,j) \triangleq P(Y_1=i,Y_2=j)$, então
    \begin{align*}
    	p(0,0) &= \frac{11^3}{13^3},
        	& p(1,0) &= \frac{12^3-11^3}{13^3},\\
        p(0,1) &= \frac{12^3-11^3}{13^3},
        	& p(1,1) &= 1-\frac{2\,(12^3-11^3)+11^3}{13^3}.\\
    \end{align*}
\end{solution}

% 6.6  % % % % % % % % % % % % % % % % % % % % %
% \setcounter{question}{6}
\question{Sabe-se que um cesto com 5 transistores
contém 2 com defeito. Os transistores devem ser testados, um de cada vez, até que
os defeituosos sejam identificados. Suponha que $N_1$ represente o número de testes
feitos até que o primeiro transistor defeituoso seja identificado e $N_2$ o número de
testes adicionais feitos até que o segundo transistor defeituoso seja identificado.
Determine a função de probabilidade conjunta de $N_1$ e $N_2$
}
\begin{solution} O número de formas de retirar transistores queimados é dado por $\binom{5}{2}=10$ e cada combinação é equiprovável. Assim,
    \begin{align*}
    \begin{array}{c|cccc|l}
	_{\large N_1}\backslash^{\large N_2} & 1 & 2 & 3 & 4 \\
	\hline
	1 & 1/10 & 1/10 & 1/10 & 1/10 & 4/10  \\
	2 & 1/10 & 1/10 & 1/10 & 0    & 3/10  \\
    3 & 1/10 & 1/10 & 0    & 0    & 2/10  \\
    4 & 1/10 & 0    & 0    & 0    & 1/10  \\
	\hline
		& 4/10 & 3/10 & 2/10 & 1/10 & \\
	\end{array}
    \end{align*}
\end{solution}

% 6.7  % % % % % % % % % % % % % % % % % % % % %
% \setcounter{question}{6}
\question{Considere uma sequência de tentativas de Bernoulli independentes, cada uma com probabilidade de sucesso $p$. Sejam $X_1$ o número de fracassos precedendo o primeiro sucesso e $X_2$ o número de fracassos entre os dois primeiros sucessos. Determine a função de probabilidade conjunta de $X_1$ e $X_2$.
}
\begin{solution}
	Sabemos que $P(X_1=i) = (1-p)^i p$, $i\in\mathbb{Z}_+$ e $P(X_2=j) = (1-p)^j p$, $j\in\mathbb{Z}_+$. Como $X_1$ e $X_2$ são independentes, então
    \begin{align*}
    	P(X_1=i,X_2=j) = P(X_1=i)P(X_2=j) = (1-p)^{i+j} p^2, \quad i,j\in\mathbb{Z}_+.
    \end{align*}
\end{solution}

% 6.11 % % % % % % % % % % % % % % % % % % % % %
\setcounter{question}{10}
\question{O proprietário de uma loja de televisores
imagina que 45\% dos clientes que entram
em sua loja comprarão um televisor comum, 15\% comprarão um televisor de
plasma e 40\% estarão apenas dando uma
olhada. Se 5 clientes entrarem nesta loja
em um dia, qual é a probabilidade de que
ele venda exatamente 2 televisores comuns e 1 de plasma naquele mesmo dia?
}
\begin{solution}
	Esse problema é bem representado por uma distribuição multinomial. $X_1$ é o número de pessoas que não compraram nada, $X_2$ compraram televisor comum e $X_3$ compraram plasma. Queremos saber
    \begin{align*}
    	P(X_1=2,X_2=2,X_3=1) = \binom{5}{2,2,1}\,\text{0,4}^2\,\text{0,45}^2\,\text{0,15}
        	\approx \text{0,146}.
    \end{align*}
\end{solution}

% 6.35 % % % % % % % % % % % % % % % % % % % % %
\setcounter{question}{34}
\question{No Problema 4, calcule a função de probabilidade condicional de $X_1$ dado que
\begin{parts}
	\part $X_2=1$;
    \part $X_2=0$;
\end{parts}
}
\begin{solution}
\begin{parts}
	\part 
    \begin{align*}
    	P(X_1=0\mid X_2=1) &= \frac{P(X_1=0,X_2=1)}{P(X_1=0,X_2=1)+P(X_1=1,X_2=1)} \\
        	&= \frac{\frac{8}{13}\frac{5}{13}}{\frac{8}{13}\frac{5}{13}
            	+\frac{5}{13}\frac{5}{13}} = \frac{8}{13}.\\
		P(X_1=1\mid X_2=1) &= \frac{5}{13}.
    \end{align*}
    \part
    \begin{align*}
    	P(X_1=0\mid X_2=0) &= \frac{P(X_1=0,X_2=0)}{P(X_1=0,X_2=0)+P(X_1=1,X_2=0)} \\
        	&= \frac{\frac{8}{13}\frac{8}{13}}{\frac{8}{13}\frac{8}{13}
            	+\frac{5}{13}\frac{8}{13}} = \frac{8}{13}.\\
		P(X_1=1\mid X_2=0) &= \frac{5}{13}.
    \end{align*}
\end{parts}
\emph{Observação:} Esses resultados eram de se esperar, pois no Problema 4 as retiradas são independentes (com reposição).
\end{solution}

% 6.36 % % % % % % % % % % % % % % % % % % % % %
% \setcounter{question}{34}
\question{No Problema 3, calcule a função de probabilidade condicional de $Y_1$ dado que
\begin{parts}
	\part $Y_2=1$;
    \part $Y_2=0$;
\end{parts}
}
\begin{solution}
\begin{parts}
	\part 
    \begin{align*}
    	P(Y_1=0\mid Y_2=1) &= \frac{P(Y_1=0,Y_2=1)}{P(Y_1=0,Y_2=1)+P(Y_1=1,Y_2=1)} \\
        	&= \frac{3\frac{11\cdot10}{13\cdot12\cdot11}}
                	{3\frac{11\cdot10}{13\cdot12\cdot11}
                    +3\frac{11\cdot2}{13\cdot12\cdot11}} = \frac{5}{6}.\\
		P(Y_1=1\mid Y_2=1) &= \frac{1}{6}.
    \end{align*}
    \part
    \begin{align*}
    	P(Y_1=0\mid Y_2=0) &= \frac{P(Y_1=0,Y_2=0)}{P(Y_1=0,Y_2=0)+P(Y_1=1,Y_2=0)} \\
        	&= \frac{\frac{11\cdot10\cdot9}{13\cdot12\cdot11}}
                {\frac{11\cdot10\cdot9}{13\cdot12\cdot11}
                +3\frac{11\cdot10}{13\cdot12\cdot11}} = \frac{3}{4}.\\
		P(Y_1=1\mid Y_2=0) &= \frac{1}{4}.
    \end{align*}
\end{parts}
\end{solution}

% 6.37 % % % % % % % % % % % % % % % % % % % % %
% \setcounter{question}{34}
\question{No Problema 5, calcule a função de probabilidade condicional de $Y_1$ dado que
\begin{parts}
	\part $Y_2=1$;
    \part $Y_2=0$;
\end{parts}
}
\begin{solution}
\begin{parts}
	\part 
    \begin{align*}
    	P(Y_1=0\mid Y_2=1) &= \frac{P(Y_1=0,Y_2=1)}{P(Y_1=0,Y_2=1)+P(Y_1=1,Y_2=1)} \\
        	&= \frac{\frac{12^3-11^3}{13^3}}
                	{\frac{12^3-11^3}{13^3}
                    +\frac{13^3-2(12^3-11^3)-11^3}{13^3}} = \frac{397}{469}.\\
		P(Y_1=1\mid Y_2=1) &= \frac{72}{469}.
    \end{align*}
    \part
    \begin{align*}
    	P(Y_1=0\mid Y_2=0) &= \frac{P(Y_1=0,Y_2=0)}{P(Y_1=0,Y_2=0)+P(Y_1=1,Y_2=0)} \\
        	&= \frac{\frac{11^3}{13^3}}
            	{\frac{11^3}{13^3}+\frac{12^3-11^3}{13^3}} = \frac{1331}{1728}.\\
		P(Y_1=1\mid Y_2=0) &= \frac{397}{1728}.
    \end{align*}
\end{parts}
\end{solution}

% 6.38 % % % % % % % % % % % % % % % % % % % % %
% \setcounter{question}{34}
\question{Escolha aleatoriamente um número $X$ a
partir do conjunto de números $\{1,2,3,4,5\}$. Escolha agora,
do subconjunto formado por números que não são maiores
que $X$, isto é, de $\{1,\dots,X\}$ ,um novo número
de forma aleatória. Chame de $Y$ este segundo número
\begin{parts}
	\part Determine a função de probabilidade conjunta de $X$ e $Y$;
    \part Determine a função de probabilidade condicional de $X$ dado $Y = i$. Faça-o
para $i = 1,2,3,4,5$;
	\part $X$ e $Y$ são independentes? Por quê?
\end{parts}
}
\begin{solution} 
\begin{parts}
	\part Sabemos que $P(Y=i\mid X=j) = 1/j$, se $i\in\{1,\dots,j\}$. Dessa forma, se $i\in\{1,\dots,j\},~j\in\{1,\dots,5\}$,
    \[{P(X=j,Y=i) = P(Y=i\mid X=j)P(X=j) = \frac{1}{5\,j}}.\]
    
    \part Se $i\in\{1,\dots,j\},~j\in\{1,\dots,5\}$, então
    \begin{align*}
    	P(X=j\mid Y=i)
        	= \frac{P(X=j, Y=i)}{\sum_{k=1}^5P(X=k,Y=i)}
        	= \frac{1/5j}{\sum_{k=i}^5 1/5k}
            = \frac{1}{\sum_{k=i}^5 j/k}.
    \end{align*}
   
	\part $X$ e $Y$ não são independentes. Um contra-exemplo é $P(X=5,Y=5)=1/25$, $P(X\!=\!5)=1/5$, $P(Y\!=\!5)=1/25$. Logo, ${P(X\!=\!5,Y\!=\!5)\neq P(X\!=\!5)P(Y\!=\!5)}$.
    
\end{parts}
\end{solution}

% 6.39 % % % % % % % % % % % % % % % % % % % % %
% \setcounter{question}{34}
\question{Dois dados são rolados. Suponha que $X$ e
$Y$ representem, respectivamente, o maior
e o menor valor obtido. Calcule a função
de probabilidade condicional de $Y$ dado
que $X = i$, para $i = 1,2,..., 6$. São $X$ e $Y$
independentes? Por quê?
}
\begin{solution} 
	Do Problema 6.1 (c), podemos verificar que a seguinte fórmula é válida (note que é necessário inverter $X$ com $Y$ para comparar com o problema). Sejam $i,j\in\{1,2,\dots,6\}$,
    \begin{align*}
    	P(Y=j\mid X=i) = \frac{P(X=i,Y=j)}{P(X=i)} =
        \begin{cases}
        	0, & i<j\\
            \frac{1}{2i-1}, & i=j\\
            \frac{2}{2i-1}, & i>j
        \end{cases}
    \end{align*}
\end{solution}

% 6.40 % % % % % % % % % % % % % % % % % % % % %
% \setcounter{question}{34}
\question{A função de probabilidade conjunta de $X$
e $Y$ é dada por
\begin{align*}
	p(1,1)&=\frac{1}{8} & p(1,2)&=\frac{1}{4}\\
    p(2,1)&=\frac{1}{8} & p(2,2)&=\frac{1}{2}\\
\end{align*}
\begin{parts}
	\part Calcule a função de probabilidade condicional de $X$ dado que $Y=i$, $i=1,2$.
    \part $X$ e $Y$ são independentes?
    \part Calcule $P(XY \le 3), P(X + Y > 2),P(X/Y>1)$.
\end{parts}
}
\begin{solution} 
\begin{parts}
	\part Sabemos que se $i,j\in\{1,2\}$
    	\begin{align*}
    		P(X=j\mid Y=i) = \frac{P(X=j, Y=i)}{P(Y=i)} = \frac{p(j,i)}{p(1,i)+p(2,i)}.
    	\end{align*}
    Dessa forma,
    \begin{align*}
    	P(X=1\mid Y=1) &= \frac{1}{2}, & P(X=1\mid Y=2) &= \frac{1}{3},\\
        P(X=2\mid Y=1) &= \frac{1}{2}, & P(X=2\mid Y=2) &= \frac{2}{3}.\\
    \end{align*}
    \part $X$ e $Y$ não são independentes.\\
    Tome por exemplo, $P(X=1,Y=1)\neq P(X=1)P(Y=1)$.
    
    \part Note que
    \begin{align*}
    	& P(XY \le 3) = 1-P(XY>3) = 1-p(2,2) = \frac{1}{2};\\
        & P(X+Y>2) = 1-P(X+Y\le 2) = 1-p(1,1) = \frac{7}{8};\\
        & P(X/Y>1) = P(X>Y) = p(2,1) = \frac{1}{8}.
    \end{align*}
\end{parts}
\end{solution}

\end{questions}

\newpage

% \setcounter{section}{4}
% \section{Exercícios Teóricos}
% \begin{questions}

% 5.27 % % % % % % % % % % % % % % % % % % % % %
\setcounter{question}{26}
\question{
Se $X$ é uniformemente distribuída em $(a, b)$, qual variável aleatória que varia linearmente com $X$ é uniformemente distribuída em $(0, 1)$?
}
\begin{solution}
Seja $Y = (X-a)/(b-a)$, então
\begin{align*}
	F_Y(y) &= P(Y\le y) = P((X-a)/(b-a) \le y)\\
    	&= P(X \le (b-a)\,y+a) = F_X((b-a)\,y+a).
\end{align*}
Derivando os dois lados da equação em relação à $y$ obtemos que
\begin{align*}
	f_Y(y) =  (b-a)f_X((b-a)\,y+a) =
    \begin{cases}
    	1, &\text{se }y\in(0,1);\\
        0, &\text{caso contrário.}
    \end{cases}
\end{align*}
Logo, $Y$ é uniformemente distribuída em $(0,1)$.\\[1mm]
\textit{Observação:} outra opção é fazer $Y = (b-X)/(b-a)$.
\end{solution}

% 5.29 % % % % % % % % % % % % % % % % % % % % %
\setcounter{question}{28}
\question{
Seja $X$ uma variável aleatória contínua
com função distribuição cumulativa $F$.
Defina a variável aleatória $Y$ como $Y = F(X)$.
Mostre que $Y$ é uniformemente distribuída em $(0, 1)$.
}
\begin{solution}
	Por simplicidade, vamos supor que $F: \mathbb{R} \to [0,1]$ seja estritamente crescente. Logo, $F$ é inversível e para $y \in (0,1)$,
	\begin{align*}
		F_Y(y) = P(Y\le y) = P(F(X)\le y) = P(X \le F^{-1}(y)) = F(F^{-1}(y)) = y.
	\end{align*}
    Dessa forma, quando $y \in \mathbb{R}$,
    \begin{align*}
    	F_Y(y) =
        \begin{cases}
    	0, &\text{se }y \le 0;\\
        y, &\text{se }0 < y < 1;\\
        1, &\text{se }y \ge 1;\\
    	\end{cases}
    \end{align*}
    o que caracteriza uma distribuição uniforme em $(0,1)$.\\[1mm]
    \textit{Observação:} Isso também acontece quando $F$ não é inversível.
\end{solution}

% 5.30 % % % % % % % % % % % % % % % % % % % % %
%\setcounter{question}{28}
\question{
Suponha que $X$ tenha função densidade
de probabilidade $f_X$. Determine a função
densidade de probabilidade da variável
aleatória $Y$ definida como $Y = aX + b$.
}
\begin{solution}
	Seja $a>0$,
	\begin{align*}
		F_Y(y) = P(Y\le y) = P(aX+b\le y) = P(X\le (y-b)/a) = F_X((y-b)/a).
	\end{align*}
    Derivando ambos lados da equação em relação à $y$ leva à
    \begin{align*}
    	f_Y(y) = \frac{f_X((y-b)/a)}{a}.
    \end{align*}
    Por outro lado, se $a<0$, então
    \begin{align*}
    	F_Y(y) &= P(X\ge (y-b)/a) = 1-P(X<(y-b)/a) = 1-F_X([(y-b)/a]^-)\\
        	&= 1-F_X((y-b)/a) \quad\text{(a variável aleatória é contínua).}
    \end{align*}
    Novamente, derivando ambos lados da equação em relação à $y$ leva à
    \begin{align*}
    	f_Y(y) = \frac{f_X((y-b)/a)}{-a}.
    \end{align*}
    Portanto, quando $a\neq 0$,
    \begin{align*}
    	f_Y(y) = \frac{f_X((y-b)/a)}{|a|}.
    \end{align*}
\end{solution}

\end{questions}
%\newpage

% \vspace{10mm} {\LARGE \textbf{Desafio!}}
% \begin{enumerate}
% \item Um investidor comprou uma ação muito instável. A cada mês, o valor dessa ação segue uma distribuição uniforme no intervalo $(0,1000)$ e é independente dos meses anteriores. O investidor pode vender a ação quando quiser, porém a cada mês que passa o dinheiro, para o investidor, vale $d$ vezes o mês anterior ($0<d<1$). Qual estratégia ele deve seguir para maximizar o retorno esperado na venda dessa ação? Se $d=4/5$, qual deve ter sido o valor máximo pago na compra da ação para que o investidor tenha um valor esperado de lucro positivo?
% \end{enumerate}

\end{document}