\begin{questions}

\setcounter{question}{2}
\question{
De quantas maneiras podem $r$ objetos ser
selecionados de um conjunto de $n$ objetos
se a ordem de seleção for considerada relevante?
}
\begin{solution}
Seja $r \leq n$, note que o primeiro objeto a ser escolhido pode ser qualquer um dos $n$ objetos, em seguida temos $n-1$ objetos disponíveis para escolha, e assim sucessivamente até selecionarmos $r$ objetos, ou seja, temos
\[ n\,(n-1) \cdots (n-(r-1)) = \dfrac{n!}{(n-r)!} \]
maneiras para selecionar os objetos.\\
Outra forma de resolver, é multiplicar o número de grupos de $r$ objetos pelo número de formas de arranjá-los, \textit{i.e.}, $\binom{n}{r}\,r!$.
\end{solution}

% % % % % % % % % % % % % % % % % % % % % % % % %
\setcounter{question}{7}
\question{Demonstre que
\[\binom{n+m}{r} = \binom{n}{0}\binom{m}{r} + \binom{n}{1}
\binom{m}{r-1}+\cdots+\binom{n}{r}\binom{m}{0}.\]
\textit{Dica}: Considere um grupo de $n$ homens e
$m$ mulheres. Quantos grupos de tamanho
$r$ são possíveis?
}
\begin{solution}
Vamos supor que $n$ e $m$ são maiores ou iguais à $r$. Fica como exercício para o aluno mostrar para os demais casos. Lembre-se da convenção $\binom{n}{k}=0$, quando $k<0$ ou $k>n$.
\begin{proof}
Imagine que tenhamos um grupo de $n+m$ indivíduos. Sabemos que é possível selecionar $\binom{n+m}{r}$ grupos diferentes de tamanho $r$. Por outro lado, se pensarmos que temos $n$ homens e $m$ mulheres, esses grupos podem conter $0$ homens e $r$ mulheres, ou $1$ homem e $r-1$ mulheres, e assim sucessivamente até $r$ homens e $0$ mulheres. Cada uma dessas configurações tem $\binom{n}{k}\binom{m}{r-k}$ possibilidades de escolha, onde $0\leq k\leq r$ é o número de homens no grupo. Logo, podemos encontrar o número total de escolhas de grupos de tamanho $r$ ao somarmos para todo $k$ inteiro entre $0$ e $r$, o que completa a demonstração.
\end{proof}
\end{solution}

% % % % % % % % % % % % % % % % % % % % % % % % %
\question{Use o Exercício Teórico 1.8 para demonstrar que
\[\binom{2n}{n} = \sum_{k=0}^{n} \binom{n}{k}^2.\]
}
\begin{solution}
\begin{proof}
Fazendo $r=m=n$ no 1.8, temos que
\begin{align*}
\binom{2n}{n}
	&= \sum_{k=0}^{n} \binom{n}{k}\binom{n}{n-k}\\
	&= \sum_{k=0}^{n} \binom{n}{k}^2.
\end{align*}
\end{proof}
\end{solution}

% % % % % % % % % % % % % % % % % % % % % % % % %
\setcounter{question}{12}
\question{Mostre que, para $n > 0$,
\[\sum_{i=0}^n (-1)^i \binom{n}{i} = 0.\]
\textit{Dica}: Use o teorema binomial.
}
\begin{solution}
\begin{proof}
	Pelo teorema binomial,
    \[0 = (1-1)^n = \sum_{i=0}^n \binom{n}{i} (-1)^i\,1^{n-i} = \sum_{i=0}^n (-1)^i \binom{n}{i}.\]
\end{proof}
\end{solution}

\end{questions}