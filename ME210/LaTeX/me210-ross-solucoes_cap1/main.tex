
\documentclass[answers, 12pt]{exam}
%\usepackage{amsmath}
\usepackage{amsthm}
%\usepackage{amsfonts}
\usepackage{amssymb}
\usepackage{mathrsfs}
\usepackage{graphicx}
\usepackage{mathtools}
\usepackage{caption}
\usepackage[brazil]{babel}
\usepackage[utf8]{inputenc}

\usepackage{url}

\usepackage{multicol}

\usepackage{tikz}
\usepackage{pgfplots}
\pgfplotsset{compat=1.15}

\newcommand{\Var}{\mathrm{Var}}
\newcommand{\euler}{\mathrm{e}}
\newcommand\diff{\mathrm{d}}

\renewcommand{\qedsymbol}{$\blacksquare$}
\renewcommand{\thequestion}{\arabic{section}.\arabic{question}}
\renewcommand{\solutiontitle}{\noindent\textbf{Solução:}\enspace}

\newtheorem{theorem}{Teorema}

\def\C{{\mathbb C}}
\def\N{{\mathbb N}}
\def\R{{\mathbb R}}
\def\Z{{\mathbb Z}}
\def\Q{{\mathbb Q}}
\def\E{{\mathbb E}}
\def\X{{\mathbb X}}
\def\ind{\mathds{1}}
\def\cal{\mathcal}
\def\T{\top}

\footer{}{\thepage}{}

\title{	ME210 - Probabilidade I - 2S 2017\\
		{\large \textit{Docente}: Marina Vachkovskaia}\\[2mm]
		{\large Soluções para problemas selecionados do livro\\[-2mm]
        \textit{Probabilidade: Um curso moderno com aplicações}
        8.ed. de Sheldon Ross}\\
}
\author{Plínio Santini Dester (p103806@dac.unicamp.br)}

\begin{document}

%% Content goes here
\maketitle

Em caso de dúvidas, sugestões ou correções (inclusive erros de digitação), não hesite em mandar um e-mail.

\section{Problemas}

\begin{questions}

% 1.4  % % % % % % % % % % % % % % % % % % % % %
\setcounter{question}{3}
\question{
Mostre que, em um triângulo de vértices $a = (x_a,y_a)$, $b = (x_b,y_b)$ e $c = (x_c,y_c)$, o seu baricentro tem coordenadas
  \[
  g = \left(\frac{x_a + x_b + x_c}{3} ~,~ \frac{y_a + y_b + y_c}{3}\right).
  \]
}

\begin{solution}
Basta substituir a solução proposta no seguinte sistema e verificar que é satisfeito.
    \[\begin{cases}
        g_y - a_y = m_a (g_x-a_x),\\
        g_y - b_y = m_b (g_x-b_x),
    \end{cases}\]
    onde $m_a = [(b_y+c_y)/2-a_y]/[(b_x+c_x)/2-a_x]$ e $m_b = [(a_y+c_y)/2-b_y]/[(a_x+c_x)/2-b_x]$ são as inclinações das medianas que passam por $a$ e $b$, respectivamente.
\end{solution}

% 1.10  % % % % % % % % % % % % % % % % % % % % %
\setcounter{question}{9}
\question{
({\bf Cônicas e Excentricidade}) A seguinte equação em coordenadas polares
  \[
  r = \frac{1}{1 + e\cos\theta}
  \]
  representa uma cônica com excentricidade $e$. Discrimine a curva em função de $e \geq 0$.
}

\begin{solution}
    A ideia é tentar re-escrever a equação para coordenadas retangulares e deixar no formato padrão de cônicas $x^2/a^2 + y^2/b^2 = 1$, ou seja,
    \begin{align*}
                    & r+e r \cos \theta = 1 \\
        \Rightarrow~ & r + ex = 1 \\   
        \Rightarrow~ & r^2 = (1 - ex)^2 \\
        \Rightarrow~ & x^2+y^2 = 1 - 2ex + e^2 x^2 \\
        % \Rightarrow~ & (1-e^2)x^2 + 2ex + \tfrac{e^2}{1-e^2} + y^2 = 1 + \tfrac{e^2}{1-e^2} \\
        \Rightarrow~ & 
            \begin{cases}
                (1-e^2)^2 \left( x + \frac{e}{1-e^2} \right)^2 + (1-e^2) y^2 = 1, & \text{se}\ e\neq1, \\
                x = \frac{y^2-1}{2}, & \text{se}\ e = 1.
            \end{cases}
    \end{align*}
    Dessa forma,
    \begin{itemize}
        \item Se $e=0$, trata-se de uma circunferência,
        \item Se $e<1$, trata-se de uma elipse,
        \item Se $e=1$, trata-se de uma parábola,
        \item Se $e>1$, trata-se de uma hipérbole.
    \end{itemize}
\end{solution}

\end{questions}


\setcounter{section}{0}
\section{Exercícios Teóricos}
\begin{questions}

\setcounter{question}{2}
\question{
De quantas maneiras podem $r$ objetos ser
selecionados de um conjunto de $n$ objetos
se a ordem de seleção for considerada relevante?
}
\begin{solution}
Seja $r \leq n$, note que o primeiro objeto a ser escolhido pode ser qualquer um dos $n$ objetos, em seguida temos $n-1$ objetos disponíveis para escolha, e assim sucessivamente até selecionarmos $r$ objetos, ou seja, temos
\[ n\,(n-1) \cdots (n-(r-1)) = \dfrac{n!}{(n-r)!} \]
maneiras para selecionar os objetos.\\
Outra forma de resolver, é multiplicar o número de grupos de $r$ objetos pelo número de formas de arranjá-los, \textit{i.e.}, $\binom{n}{r}\,r!$.
\end{solution}

% % % % % % % % % % % % % % % % % % % % % % % % %
\setcounter{question}{7}
\question{Demonstre que
\[\binom{n+m}{r} = \binom{n}{0}\binom{m}{r} + \binom{n}{1}
\binom{m}{r-1}+\cdots+\binom{n}{r}\binom{m}{0}.\]
\textit{Dica}: Considere um grupo de $n$ homens e
$m$ mulheres. Quantos grupos de tamanho
$r$ são possíveis?
}
\begin{solution}
Vamos supor que $n$ e $m$ são maiores ou iguais à $r$. Fica como exercício para o aluno mostrar para os demais casos. Lembre-se da convenção $\binom{n}{k}=0$, quando $k<0$ ou $k>n$.
\begin{proof}
Imagine que tenhamos um grupo de $n+m$ indivíduos. Sabemos que é possível selecionar $\binom{n+m}{r}$ grupos diferentes de tamanho $r$. Por outro lado, se pensarmos que temos $n$ homens e $m$ mulheres, esses grupos podem conter $0$ homens e $r$ mulheres, ou $1$ homem e $r-1$ mulheres, e assim sucessivamente até $r$ homens e $0$ mulheres. Cada uma dessas configurações tem $\binom{n}{k}\binom{m}{r-k}$ possibilidades de escolha, onde $0\leq k\leq r$ é o número de homens no grupo. Logo, podemos encontrar o número total de escolhas de grupos de tamanho $r$ ao somarmos para todo $k$ inteiro entre $0$ e $r$, o que completa a demonstração.
\end{proof}
\end{solution}

% % % % % % % % % % % % % % % % % % % % % % % % %
\question{Use o Exercício Teórico 1.8 para demonstrar que
\[\binom{2n}{n} = \sum_{k=0}^{n} \binom{n}{k}^2.\]
}
\begin{solution}
\begin{proof}
Fazendo $r=m=n$ no 1.8, temos que
\begin{align*}
\binom{2n}{n}
	&= \sum_{k=0}^{n} \binom{n}{k}\binom{n}{n-k}\\
	&= \sum_{k=0}^{n} \binom{n}{k}^2.
\end{align*}
\end{proof}
\end{solution}

% % % % % % % % % % % % % % % % % % % % % % % % %
\setcounter{question}{12}
\question{Mostre que, para $n > 0$,
\[\sum_{i=0}^n (-1)^i \binom{n}{i} = 0.\]
\textit{Dica}: Use o teorema binomial.
}
\begin{solution}
\begin{proof}
	Pelo teorema binomial,
    \[0 = (1-1)^n = \sum_{i=0}^n \binom{n}{i} (-1)^i\,1^{n-i} = \sum_{i=0}^n (-1)^i \binom{n}{i}.\]
\end{proof}
\end{solution}

\end{questions}

\vspace{3mm} {\LARGE \textbf{Desafio!}}
\begin{enumerate}
\item Se $n_1$ objetos indistinguíveis do tipo 1, $n_2$ objetos indistinguíveis do tipo 2, e assim sucessivamente até $n_M$ objetos indistinguíveis do tipo $M$ são colocados em $r$ caixas numeradas, qual é o número de arranjos distinguíveis? % E se as caixas forem idênticas?
\item Em uma escola, existem $2n$ estudantes, $n\in\mathbb{N}$ e $n\geq2$. Cada semana $n$ estudantes vão à uma viagem. Após algumas viagens a seguinte condição é satisfeita: cada par de estudantes já esteve junto em pelo menos uma viagem. Mostre que o número mínimo de viagens para isso acontecer é 6.\\
\textit{Dica:} comece pelo caso em que $n$ é par.
\end{enumerate}

\end{document}