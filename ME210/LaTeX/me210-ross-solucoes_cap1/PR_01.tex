
\begin{questions}

% 1.1  % % % % % % % % % % % % % % % % % % % % %
\question{
\begin{parts}
	\part Quantas placas de carro diferentes
com 7 caracteres podem ser formadas
se os dois primeiros campos da placa
forem reservados para as letras e os
outros cinco para os números?
	\part Repita a letra (a) supondo que nenhuma letra ou número possa ser repetido em uma mesma placa.
\end{parts}
}
\begin{solution}
\begin{parts}
	\part Pelo princípio básico da contagem, temos $26^2\cdot10^5 = 67\,600\,000$ placas diferentes.
	\part Quando não podemos repetir letras ou números, a cada nova letra/número escolhido, temos uma possibilidade a menos. Nesse caso, o número total de placas é $(26\cdot25)\cdot(10\cdot9\cdot8\cdot7\cdot6) = 19\,656\,000.$
\end{parts}
\end{solution}

% 1.2  % % % % % % % % % % % % % % % % % % % % %
\question{
Quantas sequências de resultados são
possíveis quando um dado é rolado quatro vezes, supondo, por exemplo, que 3, 4, 3, 1 é o resultado obtido se o primeiro
dado lançado cair no 3, o segundo no 4, o
terceiro no 3 e o quarto no 1?
}
\begin{solution}
Pelo princípio básico da contagem temos $6^4 = 1296$ sequências possíveis.
\end{solution}

% 1.3  % % % % % % % % % % % % % % % % % % % % %
\question{
Vinte trabalhadores serão alocados em
vinte tarefas diferentes, um em cada tarefa. Quantas alocações diferentes são
possíveis?
}
\begin{solution}
Para a primeira tarefa temos 20 possibilidades de trabalhadores, em seguida para a segunda temos 19, e assim sucessivamente até a vigésima tarefa, ou seja, temos $20!$ alocações possíveis.
\end{solution}

% 1.4  % % % % % % % % % % % % % % % % % % % % %
\question{
João, Júlio, Jonas e Jacques formaram
uma banda com quatro instrumentos. Se
cada um dos garotos é capaz de tocar todos os instrumentos, quantas diferentes combinações são possíveis? E se João e
Júlio souberem tocar todos os quatro instrumentos, mas Jonas e Jacques souberem tocar cada um deles apenas o piano e a
bateria?
}
\begin{solution}
	Se todos podem tocar todos os instrumentos, então analogamente ao Problema 1.3, temos $4! = 24$ possibilidades. Por outro lado, se Jonas e Jacques estão restritos a dois instrumentos, então João e Júlio também estão e, dessa forma, temos $2\cdot2 = 4$ possibilidades.
\end{solution}

% 1.5  % % % % % % % % % % % % % % % % % % % % %
\question{
Por muitos anos, os códigos telefônicos de
área nos EUA e no Canadá eram formados por uma sequência de três algarismos. O primeiro algarismo era um inteiro entre
2 e 9, o segundo algarismo era 0 ou 1, e o terceiro digito era um inteiro entre 1 e 9. Quantos códigos de área eram possíveis?
Quantos códigos de área começando com um 4 eram possíveis?
}
\begin{solution}
	Para o primeiro dígito temos 8 possibilidades, para o segundo temos 2 e para o terceiro temos 9. Logo, eram possíveis $8\cdot2\cdot9 = 144$ códigos de área. Ao restringirmos o primeiro algarismo, sobram apenas $2\cdot9 = 18$ possibilidades.
\end{solution}

% 1.6  % % % % % % % % % % % % % % % % % % % % %
\question{
Uma famosa canção de ninar começa
com os versos
``Quando ia para São Ives,
Encontrei um homem com 7 mulheres.
Cada mulher tinha 7 sacos.
Cada saco tinha 7 gatos.
Cada gato tinha 7 gatinhos...''
Quantos gatinhos o viajante encontrou?
}
\begin{solution}
	Pelo princípio básico de contagem, temos $7^4 = 2401$ gatinhos.
\end{solution}

% 1.7  % % % % % % % % % % % % % % % % % % % % %
\question{
\begin{parts}
\part De quantas maneiras diferentes 3 garotos e 3 garotas podem sentar-se em fila?
\part  De quantas maneiras diferentes 3 garotos e 3 garotas podem sentar-se em fila se os garotos e as garotas sentarem-se juntos?
\part E se apenas os garotos sentarem-se juntos?
\part E se duas pessoas do mesmo sexo não puderem se sentar juntas?
\end{parts}
}
\begin{solution}
\begin{parts}
\part Se conseguirmos discernir cada pessoa, então teremos $6! = 720$ maneiras diferentes.
\part Dentro de cada grupo temos $3!$ combinações diferentes e entre os grupos temos $2!$ combinações. Logo, no total temos $2!\cdot3!\cdot3! = 72$ maneiras diferentes.
\part Dentro do grupo de meninos temos $3!$ combinações. O número de combinações entre o grupo de meninos e as meninas é $4!$. Logo, no total temos $3!\cdot4! = 144$.
\part Nesse caso, a fila deve ser alternada entre meninos e meninas, podendo começar com qualquer um dos dois. Dessa forma, temos $2\cdot3!\cdot3! = 72$ maneiras diferentes de arranjar a fila.
\end{parts}
\end{solution}

% 1.8  % % % % % % % % % % % % % % % % % % % % %
\question{
Quantos arranjos de letras diferentes podem ser feitos a partir de
\begin{parts}
\part Sorte?
\part Propose?
\part Mississippi?
\part Arranjo?
\end{parts}
}
\begin{solution}
\begin{parts}
\part $5! = 120$.
\part $\dfrac{7!}{2!\,2!} = 1\,260$.
\part $\dfrac{11!}{4!\,4!\,2!} = 34\,650$.
\part $\dfrac{7!}{2!\,2!} = 1\,260$.
\end{parts}
\end{solution}

% 1.9  % % % % % % % % % % % % % % % % % % % % %
\question{
Uma criança tem 12 blocos, dos quais 6
são pretos, 4 são vermelhos, 1 é branco e
1 é azul. Se a criança colocar os blocos em
linha, quantos arranjos são possíveis?
}
\begin{solution}
$\dfrac{12!}{6!\,4!} = 27\,720$
\end{solution}

% 1.10 % % % % % % % % % % % % % % % % % % % % %
\question{
De quantas maneiras 8 pessoas podem se
sentar em fila se
\begin{parts}
\part não houver restrições com relação à
ordem dos assentos?
\part as pessoas $A$ e $B$ tiverem que se sentar uma ao lado da outra?
\part houver 4 homens e 4 mulheres e não
for permitido que dois homens ou
duas mulheres se sentem em posições
adjacentes?
\part houver 5 homens e for necessário que
eles se sentem lado a lado?
\part houver 4 casais e cada casal precisar
sentar-se junto?
\end{parts}
}
\begin{solution}
\begin{parts}
\part $8! = 40\,320$.
\part $2!\,7! = 10\,080$ (semelhante ao Problema 1.7(c)).
\part $2\,4!\,4! = 34\,650$ (semelhante ao Problema 1.7(d)).
\part $5!\,4! = 2\,880$ (semelhante ao Problema 1.7(c)).
\part $(2!)^4\,4! = 384$.
\end{parts}
\end{solution}

% 1.11 % % % % % % % % % % % % % % % % % % % % %
\question{
De quantas maneiras três romances, dois
livros de matemática e um livro de química
podem ser arranjados em uma prateleira se
\begin{parts}
\part eles puderem ser colocados em qualquer ordem?
\part for necessário que os livros de matemática fiquem juntos e os romances
também?
\part for necessário que os romances fiquem
juntos, podendo os demais livros ser
organizados de qualquer maneira?
\end{parts}
}
\begin{solution}
\begin{parts}
\part $6! = 720$.
\part Entre os grupos de livro temos $3!$ combinações e contando com as combinações dentro de cada grupo ficamos com $3!\,2!\,3! = 72$.
\part $4!\,3! = 144$ (semelhante ao Problema 1.7(c)).
\end{parts}
\end{solution}

% 1.12 % % % % % % % % % % % % % % % % % % % % %
\question{
Cinco prêmios diferentes (melhor desempenho escolar, melhores qualidades de
liderança, e assim por diante) serão dados
a estudantes selecionados de uma classe
de trinta alunos. Quantos resultados diferentes são possíveis se
\begin{parts}
\part um estudante puder receber qualquer
número de prêmios?
\part cada estudante puder receber no máximo um prêmio?
\end{parts}
}
\begin{solution}
\begin{parts}
\part $30^5 = 24\,300\,000$.
\part $30\cdot29\cdot28\cdot27\cdot26 = 17\,100\,720$.
\end{parts}
\end{solution}

% 1.13 % % % % % % % % % % % % % % % % % % % % %
\question{
Considere um grupo de vinte pessoas. Se
todos cumprimentarem uns aos outros
com um aperto de mãos, quantos apertos
de mão serão dados?
}
\begin{solution}
$\binom{20}{2} = 190$.
\end{solution}

% 1.14 % % % % % % % % % % % % % % % % % % % % %
\question{
Quantas mãos de pôquer de cinco cartas
existem?
}
\begin{solution}
$\binom{52}{5} = 2\,598\,960$.
\end{solution}

% 1.15 % % % % % % % % % % % % % % % % % % % % %
\question{
Uma turma de dança é formada por 22
estudantes, dos quais 10 são mulheres e
12 são homens. Se 5 homens e 5 mulheres forem escolhidos para formar pares,
quantas combinações diferentes serão possíveis?
}
\begin{solution}
Podemos selecionar o grupo de mulheres e homens de $\binom{10}{5}$ e $\binom{12}{5}$ maneiras diferentes, respectivamente. Uma vez escolhidos os dançarinos, podemos formar os pares de $5!$ formas diferentes. Logo, no total teremos $\binom{10}{5}\binom{12}{5}\, 5! = 23\,950\,080$ combinações diferentes.
\end{solution}

% 1.16 % % % % % % % % % % % % % % % % % % % % %
\question{
Um estudante tem que vender 2 livros de uma coleção formada por 6 livros de matemática, 7 de ciências e 4 de economia. Quantas escolhas serão possíveis se
\begin{parts}
\part ambos os livros devem tratar do mesmo assunto?
\part os livros devem tratar de assuntos
diferentes?
\end{parts}
}
\begin{solution}
\begin{parts}
\part $\binom{6}{2} + \binom{7}{2} + \binom{4}{2} = 42$.
\part Isso é o complemento do item anterior em relação à possibilidade de vender dois livros. Assim, temos $\binom{17}{2} - 42 = 94$ escolhas possíveis.
\end{parts}
\end{solution}

% 1.17 % % % % % % % % % % % % % % % % % % % % %
\question{
Sete presentes diferentes devem ser distribuídos entre 10 crianças. Quantos resultados diferentes são possíveis se nenhuma criança puder receber mais de um
presente?
}
\begin{solution}
``O presente escolhe a criança'': $10\cdot9\cdots4 = 604\,800$.
\end{solution}

% 1.18 % % % % % % % % % % % % % % % % % % % % %
\question{
Um comitê de 7 pessoas, formado por 2
petistas, 2 democratas e 3 peemedebistas deve ser escolhido de um grupo de 5
petistas, 6 democratas e 4 peemedebistas.
Quantos comitês são possíveis?
}
\begin{solution}
$\binom{5}{2}\binom{6}{2}\binom{4}{3} = 600$.
\end{solution}

% 1.19 % % % % % % % % % % % % % % % % % % % % %
\question{
De um grupo de 8 mulheres e 6 homens,
pretende-se formar um comitê formado
por 3 homens e 3 mulheres. Quantos comitês diferentes são possíveis se
\begin{parts}
\part 2 dos homens se recusarem a trabalhar juntos?
\part 2 das mulheres se recusarem a trabalhar juntas?
\part 1 homem e 1 mulher se recusarem a trabalhar juntos?
\end{parts}
}
\begin{solution}
\begin{parts}
\part Consideramos os casos nos quais não selecionamos os 2 homens e selecionamos apenas um deles, temos $\binom{8}{3}\cdot\left( \binom{4}{3} + 2\binom{4}{2} \right) = 896$.
\part Análogo ao item anterior, $\binom{6}{3}\cdot\left( \binom{6}{3} + 2\binom{6}{2} \right) = 1\,000$
\part Consideramos os casos nos quais não selecionamos nem o homem, nem a mulher; os casos nos quais selecionamos a mulher e enfim os casos nos quais selecionamos o homem, temos $\binom{7}{3}\binom{5}{3} + \binom{7}{2}\binom{5}{3} + \binom{7}{3}\binom{5}{2} = 910$.
\end{parts}
\end{solution}

% 1.20 % % % % % % % % % % % % % % % % % % % % %
\question{
Uma pessoa tem 8 amigos, dos quais 5 serão convidados para uma festa.
\begin{parts}
\part Quantas escolhas existem se dois dos
amigos estiverem brigados e por esse
motivo não puderem comparecer?
\part Quantas escolhas existem se dois dos
amigos puderem ir apenas se forem
juntos?
\end{parts}
}
\begin{solution}
\begin{parts}
\part Considerando os casos sem os amigos e com apenas um dos amigos, temos $\binom{6}{5} + 2\binom{6}{4} = 36$
\part Considerando os casos sem os amigos e com os dois amigos, temos $\binom{6}{5} + \binom{6}{3} = 26$.
\end{parts}
\end{solution}

% 1.21 % % % % % % % % % % % % % % % % % % % % %
\question{
Considere a malha de pontos mostrada no livro. Suponha que, começando do ponto $A$, você possa ir um passo para cima
ou para direita em cada movimento. Esse
procedimento continua até que o ponto $B$
seja atingido. Quantos caminhos possíveis
existem entre $A$ e $B$?\\
\textit{Dica}:Note que, para atingir $B$ a partir de
$A$, você deve dar quatro passos à direita e
três passos para cima.
}
\begin{solution}
Podemos pensar nesse problema como o número de arranjos entre as letras DDDDCCC, ou seja, $\frac{7!}{4!\,3!} = \binom{7}{3} = 35$ caminhos possíveis.
\end{solution}

% 1.22 % % % % % % % % % % % % % % % % % % % % %
\question{
No Problema 21, quantos caminhos diferentes existem entre $A$ e $B$ que passam
pelo ponto circulado mostrado na figura do livro?
}
\begin{solution}
Basta calcular o número de caminhos até o ponto circulado e multiplicar pelo número de caminhos até o ponto $B$, ou seja, $\binom{4}{2}\binom{3}{1} = 18$ caminhos possíveis.
\end{solution}

% 1.23 % % % % % % % % % % % % % % % % % % % % %
\question{
Um laboratório de psicologia dedicado a
pesquisar os sonhos possui 3 quartos com
2 camas cada. Se 3 conjuntos de gêmeos
idênticos forem colocados nessas 6 camas
de forma que cada par de gêmeos durma
em camas diferentes em um mesmo quarto,
quantas diferentes combinações são possíveis?
}
\begin{solution}
$(2!)^3\cdot3! = 48$.
\end{solution}

% 1.24 % % % % % % % % % % % % % % % % % % % % %
\question{
Expanda $(3x^2+y)^5$.
}
\begin{solution}
Utilizando o teorema binomial, temos que
\begin{equation*}
	(3x^2+y)^5 = \sum_{k=0}^{5} \binom{5}{k} (3x^2)^{5-k} y^k = 
    243 x^{10} + 405 x^8 y + 270 x^6 y^2 + 90 x^4 y^3 + 15 x^2 y^4 + y^5
\end{equation*}
\end{solution}

% 1.25 % % % % % % % % % % % % % % % % % % % % %
\question{
O jogo de bridge é jogado por 4 jogadores, cada um deles com 13 cartas. Quantas
jogadas de bridge são possíveis?
}
\begin{solution}
Como a ordem das cartas nas mãos dos jogadores não importa, então temos $\binom{52}{13,13,13,13} \approx \text{5,36} \cdot 10^{28} $ condições iniciais possíveis para o jogo.
\end{solution}

% 1.26 % % % % % % % % % % % % % % % % % % % % %
\question{
Expanda $(x_1 + 2 x_2 + 3 x_3)^4$.
}
\begin{solution}
Utilizando o teorema multinomial, temos que
\begin{align*}
	(x_1 + 2 x_2 + 3 x_3)^4 
    &= \sum_{n_1+n_2+n_3=4} \binom{4}{n_1,n_2,n_3} x_1^{n_1} (2x_2)^{n_2} (3x_3)^{n_3}\\
    &= x_1^4 + 16 x_2^4 + 81 x_3^4 + 8 x_1^3 x_2 + 12 x_1^3 x_3 + 		32 x_1 x_2^3 + 108 x_1 x_3^3 +\\
    &\quad 216 x_2 x_3^3 + 96 x_2^3 x_3 + 24 x_1^2 x_2^2 + 54 x_1^2 		x_3^2 + 216 x_2^2 x_3^2 + 72 x_1^2 x_2 x_3 +\\
    &\quad 216 x_1 x_2 x_3^2 + 144 x_1 x_2^2 x_3.
\end{align*}
\end{solution}

% 1.27 % % % % % % % % % % % % % % % % % % % % %
\question{
Se 12 pessoas vão ser divididas em 3 comitês de 3, 4 e 5 pessoas. Quantas divisões
são possíveis?
}
\begin{solution}
$\binom{12}{3,4,5} = 27\,720$.
\end{solution}

% 1.29 % % % % % % % % % % % % % % % % % % % % %
\setcounter{question}{28}
\question{
Dez halterofilistas disputam uma competição de levantamento de peso por
equipes. Destes, 3 são dos EUA, 4 da
Rússia, 2 da China e 1 do Canadá. Se
a soma de pontos considerar os países
que os atletas representam, mas não as
identidades desses atletas, quantos diferentes 
resultados são possíveis? Quantos resultados diferentes correspondem
à situação em que os EUA possuem um
atleta entre os três primeiros e 2 entre
os três últimos?
}
\begin{solution}
Como apenas a nacionalidade importa na ordenação, então temos $\binom{10}{3,4,2,1} = 12\,600$ resultados possíveis. Entretanto, com as restrições sobre os EUA, temos apenas $\binom{7}{4,2,1}\binom{3}{1}\binom{3}{2} = 945$ resultados, onde o primeiro coeficiente refere-se à ordem das nacionalidades (exceto os EUA) e os demais coeficientes referem-se às combinações que podem ser feitas com os EUA nas três primeiras e nas três últimas posições.
\end{solution}

% 1.30 % % % % % % % % % % % % % % % % % % % % %
\question{
Delegados de 10 países, incluindo Rússia, França, Inglaterra e os EUA, devem
sentar-se lado a lado. Quantos arranjos de
assentos diferentes são possíveis se os delegados franceses e ingleses tiverem que
sentar-se lado a lado e os delegados da
Rússia e dos EUA não puderem sentar-se
lado a lado.
}
\begin{solution}
Primeiramente, consideremos a situação que delegados franceses e ingleses sentem-se lado a lado, o que nos dá $2!\,9!$ combinações.\\
Agora, vamos considerar a situação que delegados franceses e ingleses; e delegados dos EUA e Rússia sentam-se lado a lado, o que nos dá $2!\,2!\,8!$ combinações.\\
Finalmente, basta notar que o que queremos é exatamente a primeira situação tirando as combinações da segunda situação, ou seja, $2!\,9! - 2!\,2!\,8! = 564\,480$.
\end{solution}

% 1.31 % % % % % % % % % % % % % % % % % % % % %
\question{
Se 8 quadros-negros idênticos forem divididos entre quatro escolas, quantas divisões
são possíveis? E se cada escola tiver que receber pelo menos um quadro-negro?
}
\begin{solution}
Trata-se de uma aplicação direta da Proposição 6.2, isto é, temos $\binom{11}{3} = 165$ divisões possíveis. Por outro lado, quando exigimos que cada escola receba pelo menos um quadro-negro, então usamos a Proposição 6.1, isto é, temos $\binom{7}{3} = 35$ divisões possíveis.
\end{solution}

\end{questions}
