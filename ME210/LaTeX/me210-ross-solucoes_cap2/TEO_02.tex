\begin{questions}
% 2.1   % % % % % % % % % % % % % % % % % % % % %
\question{
Prove que $E\,F \subset E \subset E \cup F$.
}
\begin{solution}
\begin{proof}
	~\\
	Seja $\omega \in E\,F$. Por definição de intersecção, $\omega \in E$ e $\omega \in F$. Em particular, $\omega \in E$. Logo, $E\,F \subset E$.\\
    Seja $\omega' \in E$. Então, $\omega' \in E$ ou $\omega' \in F$. Por definição de união, $\omega' \in E \cup F$. Logo, $E \subset E \cup F$.
\end{proof}
\end{solution}

% 2.2   % % % % % % % % % % % % % % % % % % % % %
\question{
Prove que se $E \subset F$, então $F^c \subset E^c$.
}
\begin{solution}
\begin{proof}
    \begin{align*}
    	E \subset F 
        	&\Leftrightarrow \forall \omega \in \Omega, 
            	(\omega \in E \Rightarrow \omega \in F)\\
			&\Leftrightarrow \forall \omega \in \Omega, 
            	(\omega \notin F \Rightarrow \omega \notin E)\\
			&\Leftrightarrow F^c \subset E^c.
    \end{align*}
\end{proof}
\end{solution}

% 2.3   % % % % % % % % % % % % % % % % % % % % %
\question{
Prove que $F = F\,E\cup F\,E^c$ e $E\cup F = E\cup E^c\,F$.
}
\begin{solution}
\begin{proof}
Primeiramente, vamos mostrar que $F \subset F\,E\cup F\,E^c$. Seja $\omega \in F$. Se $\omega \in E$, então $\omega \in F\,E$, por outro lado, se $\omega \notin E$, então $\omega \in F\,E^c$. Logo, $\omega \in F\,E$ ou $\omega \in F\,E^c$. Por definição de união. $\omega \in F\,E\cup F\,E^c$.\\
Por outro lado, do exercício 2.1, temos que $F\,E\cup F\,E^c \subset F \cup F = F$. Logo,  como temos inclusão nos dois sentidos $F = F\,E\cup F\,E^c$.\\

Agora, mostremos que $E\cup F \subset E\cup E^c\,F$. Seja $\omega \in E\cup F$. Se $\omega \notin E$, então $\omega \in F$, isto é, $\omega \in F\,E^c$. Logo, $\omega \in E$ ou $\omega \in F\,E^c$. Por definição de união, $\omega \in E\cup E^c\,F$.\\
Por outro lado, do exercício 2.1, $E\cup E^c\,F \subset E\cup F$. Assim, como temos inclusão nos dois sentidos, $E\cup F = E\cup E^c\,F$.
\end{proof}
\end{solution}

% 2.6   % % % % % % % % % % % % % % % % % % % % %
\setcounter{question}{5}
\question{Sejam três eventos $E$, $F$ e $G$. Determine
expressões para esses eventos de forma que, de $E$, $F$ e $G$,
\begin{parts}
	\part apenas $E$ ocorra;
    \part $E$ e $G$ ocorram, mas não $F$;
    \part pelo menos um dos eventos ocorra;
    \part pelo menos dois dos eventos ocorram;
    \part todos os três eventos ocorram;
    \part nenhum dos eventos ocorra;
    \part no máximo um dos eventos ocorra;
    \part no máximo dois dos eventos ocorram;
    \part no máximo três dos eventos ocorram.
\end{parts}
}
\begin{solution}
\begin{parts}
	\part $E \, F^c \, G^c$;
    \part $E \, F^c \, G$;
    \part $(E^c\,F^c\,G^c)^c = E \cup F \cup G$;
    \part $E\,F\,G \cup E\,F\,G^c \cup E\,F^c\,G \cup E^c\,F\,G$;
    \part $E\,F\,G$;
    \part $E^c\,F^c\,G^c = (E\cup F\cup G)^c$;
    \part $E^c\,F^c\,G^c \cup E\,F^c\,G^c \cup E^c\,F\,G^c \cup E^c\,F^c\,G$;
    \part $(E\,F\,G)^c = E^c\cup F^c\cup G^c$;
    \part $\Omega$.
\end{parts}
\end{solution}

% 2.7   % % % % % % % % % % % % % % % % % % % % %
\question{Determine a expressão mais simples para
os seguintes eventos:
\begin{parts}
	\part $(E\cup F)(E\cup F^c)$;
    \part $(E\cup F)(E^c\cup F)(E\cup F^c)$;
    \part $(E\cup F)(F\cup G)$;
\end{parts}
}
\begin{solution}
\begin{parts}
	\part Pela lei distributiva,
    	$(E\cup F)(E\cup F^c) = E\cup F\,F^c = E$.
    \part Do item (a), temos que
    	$(E\cup F)(E^c\cup F)(E\cup F^c) = E\,(E^c\cup F) = E\,F$.
    \part Pela lei distributiva,
    $(E\cup F)(F\cup G) = F\cup E\,G$.
\end{parts}
\end{solution}

% 2.10  % % % % % % % % % % % % % % % % % % % % %
\setcounter{question}{9}
\question{Demonstre que $P(E\cup F\cup G) = P(E)+P(F)+P(G)-P(E^c\,F\,G)-P(E\,F^c\,G)-P(E\,F\,G^c)-2\,P(EFG)$.
}
\begin{solution}
\begin{proof}
	Aplicando a Proposição~4.3 sucessivas vezes, obtemos que
    \begin{align*}
    	P( E\cup F\cup G) 
        &= P( (E\cup F)\cup G) \\
        &= P(E\cup F) + P(G) - P((E\cup F)\,G) \\
        &= P(E) + P(F) + P(G) - P(E\,F) - P(E\,G\cup F\,G) \\
        &= P(E) + P(F) + P(G) - P(E\,F) - P(E\,G) - P(F\,G) + P(E\,F\,G),
    \end{align*}
    que é um caso particular da Proposição~4.4. \\
    Sabemos que $P(E\,F) = P(E\,F\,G) + P(E\,F\,G^c)$, pois $E\,F\,G$ e $E\,F\,G^c$ são eventos disjuntos, cuja união resulta em $E\,F$. Podemos fazer o mesmo para $P(E\,G)$ e $P(F\,G)$. Substituindo essas probabilidades na expressão acima completa a demonstração.
\end{proof}
\end{solution}

% 2.11  % % % % % % % % % % % % % % % % % % % % %
\question{Se $P(E) = \text{0,9}$ e $P(F) = \text{0,8}$, mostre que
$P(E\,F) \geq \text{0,7}$. De forma geral,demonstre a
desigualdade de Bonferroni, isto é, $P(E\,F) \geq P(E)+P(F)-1$.
}
\begin{solution}
\begin{proof}
	Da Proposição 4.3 e do Axioma 1, sabemos que 
	\[P(E)+P(F)-P(E\,F) = P(E\cup F) \leq 1.\]
    Logo, $P(E\,F) \geq P(E)+P(F)-1$.
\end{proof}
\end{solution}

\end{questions}