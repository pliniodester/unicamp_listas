
\begin{questions}

% 2.1   % % % % % % % % % % % % % % % % % % % % %
\question{ Uma caixa contém 3 bolas de gude: 1 vermelha, 1 verde e uma azul. Considere um experimento que consiste em retirar uma bola de gude da caixa, colocar outra em seu lugar e então retirar uma segunda bola da caixa. Descreva o espaço amostral. Repita considerando que a segunda bola seja retirada sem que a primeira seja substituída
}
\begin{solution}
	Para o primeiro experimento o espaço amostral é
    \[\Omega_1 = \{ (1,1),(1,2),(1,3),(2,1),(2,2),(2,3),(3,1),(3,2),(3,3) \},\]
    onde 1, 2 e 3 representam as cores vermelha, verde e azul, respectivamente. Para o segundo experimento, o espaço amostral é
    \[\Omega_2 = \{ (1,2),(1,3),(2,1),(2,3),(3,1),(3,2) \}.\]
\end{solution}

% 2.2   % % % % % % % % % % % % % % % % % % % % %
\question{ Em um experimento, um dado é rolado
continuamente até que um 6 apareça,
momento em que o experimento é interrompido. Qual é o espaço amostral do experimento? Chame de $E_n$ o evento em
que o dado é rolado $n$ vezes para que o
experimento seja finalizado. Que pontos
do espaço amostral estão contidos em $E_n$? O que é $\left( \bigcup_{n=1}^\infty E_n \right)^c$?
}
\begin{solution}
	O espaço amostral do experimento é
    \[\Omega = \{ (i_1,i_2,\cdots,i_{n-1},6) \mid i_k\in\{1,2,\cdots,5\}, k\in\{1,2,\cdots,n-1\}, n\in\mathbb{N} \}.\]
    Seja $n\in\mathbb{N}$, o conjunto 
    \[\{ (i_1,i_2,\cdots,i_{n-1},6) \mid i_k\in\{1,2,\cdots,5\}, k\in\{1,2,\cdots,n-1\}\} = E_n.\]
    Assim, $E_n$ representa não tirar 6 nas $n-1$ primeiras rolagens e tirar 6 na $n$-ésima rolagem. O evento $\left( \bigcup_{n=1}^\infty E_n \right)^c$ representa nunca rolar um 6.
\end{solution}

% 2.3   % % % % % % % % % % % % % % % % % % % % %
\question{Dois dados são lançados. Seja $E$ o evento em que a soma dos dados é ímpar, $F$ o evento em que o número 1 sai em pelo
menos um dos dados, e $G$ o evento em que a soma dos dados é igual a 5. Descreva os eventos $E\,F$, $E\cup F$, $F\,G$, $E\,F^c$ e $E\,F\,G$.
}
\begin{solution}
\begin{align*}
	E\,F 		&= \{(1,2),(1,4),(1,6),(2,1),(4,1),(6,1)\},\\
    E\cup F 	&= \{(1,1),(1,2),(1,3),(1,4),(1,5),(1,6),
    				(2,1),(2,3),(2,5),(3,1),(3,2),\\
    			&\quad\quad	(3,4),(3,6),(4,1),(4,3),(4,5),
                	(5,1),(5,2),(5,4),(5,6),(6,1),(6,3),(6,5)\},\\
	F\,G		&= \{(1,4),(4,1)\},\\
    E\,F^c		&= \{(2,3),(2,5),(3,2),(3,4),(3,6),(4,3),(4,5),(5,2),
    				(5,4),(5,6),(6,3),(6,5)\},\\
	E\,F\,G		&= F\,G.
\end{align*}
\end{solution}

% 2.8   % % % % % % % % % % % % % % % % % % % % %
\setcounter{question}{7}
\question{Suponha que $A$ e $B$ sejam eventos mutuamente exclusivos para os quais $P(A) = \text{0,3}$ e $P(B) = \text{0,5}$. Qual é a probabilidade de que:
\begin{parts}
	\part $A$ ou $B$ ocorra?
	\part $A$ ocorra, mas $B$ não ocorra?
	\part $A$ e $B$ ocorram?
\end{parts}
}
\begin{solution}
\begin{parts}
	\part Como são eventos mutuamente exclusivos, $P(A\cup B) = P(A) + P(B) = \text{0,8}$.
	\part $P(A\backslash B) = P(A) = \text{0,3}$.
	\part $P(A\,B) = P(\emptyset) = 0$.
\end{parts}
\end{solution}

% 2.15  % % % % % % % % % % % % % % % % % % % % %
\setcounter{question}{14}
\question{Se é assumido que todas as $\binom{52}{5}$ mãos de pôquer são igualmente prováveis, qual é a probabilidade de alguém sair com
\begin{parts}
	\part um \textit{flush} (uma mão é chamada de \textit{flush} se todas as 5 cartas são do mesmo naipe)?
	\part um par (que ocorre quando as cartas
são do tipo a, a, b, c, d, onde a, b, c e d
são cartas distintas)?
	\part dois pares (que ocorre quando as cartas
são do tipo a, a, b, b, c,onde a, b e c são cartas distintas)?
	\part trinca (que ocorre quando as cartas
são do tipo a, a, a, b, c, onde a, b e c
são cartas distintas)?
	\part quadra (que ocorre quando as cartas
são do tipo a, a, a, a, b, onde a e b são
cartas distintas)?
\end{parts}
}
\begin{solution}
\begin{parts}
	\part Temos 4 naipes para escolher e para cada naipe 13 cartas. Logo, a probabilidade de um \textit{flush} é dada por
	\[\dfrac{4\binom{13}{5}}{\binom{52}{5}} = \dfrac{33}{16660} \approx \text{0,002}.\]
	\part Temos 13 opções de pares, para cada opção de par temos $\binom{4}{2}$ maneiras de escolher os naipes. Restam 3 cartas para escolhermos entre as 12 que sobraram e para cada carta 4 naipes. Logo, a probabilidade de um par é
	\[\dfrac{\left(13\binom{4}{2}\right)\left(4^3\binom{12}{3}\right)}
    	{\binom{52}{5}} = \dfrac{352}{833} \approx \text{0,423}.\]
	\part Temos 13 cartas para escolher as duas opções de pares. Para cada opção de par temos $\binom{4}{2}$ maneiras de escolher os naipes. Sobra apenas uma opção de carta e seu naipe para escolher entre as 11 que restaram. Logo, a probabilidade de dois pares é
	\[\dfrac{\binom{13}{2}\binom{4}{2}^2 \, 11\cdot 4  }
    	{\binom{52}{5}} = \dfrac{198}{4165} \approx \text{0,048}.\]
	\part Temos 13 opções de trinca e $\binom{4}{3}$ maneiras de escolher os naipes. Restam 2 cartas para escolhermos entre as 12 que sobraram e para cada carta 4 naipes. Logo, a probabilidade da trinca é
	\[\dfrac{\left(13\binom{4}{3}\right)\left(4^2\binom{12}{2}\right)}
    	{\binom{52}{5}} = \dfrac{88}{4165} \approx \text{0,021}.\]
	\part Temos 13 opções de quadra e apenas uma maneira de escolher os naipes. Resta apenas 1 opção de carta para escolher entre as 12 que sobraram e seu naipe. Logo, a probabilidade da quadra é
	\[\dfrac{13 \cdot 12 \cdot 4}
    	{\binom{52}{5}} = \dfrac{1}{4165} \approx \text{0,00024}.\]
\end{parts}
\end{solution}

% 2.16  % % % % % % % % % % % % % % % % % % % % %
\question{Pôquer com dados é jogado com o lançamento
simultâneo de 5 dados. Mostre que:
\begin{parts}
	\part $P$[nenhum dado de mesmo valor] = 0,0926.
	\part $P$[um par] = 0,4630.
	\part $P$[dois pares] = 0,2315.
	\part $P$[trinca] = 0,1543.
	\part $P$[uma trinca e um par] = 0,0386.
	\part $P$[quatro dados iguais]= 0,0193.
	\part $P$[cinco dados iguais] = 0,0008.
\end{parts}
}
\begin{solution}
\begin{parts}
	\part Nenhum dado de mesmo valor só tem uma configuração possível e $6\cdot 5\cdot 4\cdot 3\cdot 2$ escolhas para os valores dos dados. Logo, $P\text{[nenhum dado de mesmo valor]} = 	\dfrac{6\cdot 5\cdot 4\cdot 3\cdot 2}{6^5} \approx \text{0,0926}$.
	\part O número de configurações de um par de dados de mesmo valor entre 5 dados é $\binom{5}{2}$ e as escolhas de valores para os dados são $6\cdot 5\cdot 4\cdot 3$. Dessa forma, $P\text{[um par]} = \displaystyle\binom{5}{2}\dfrac{6\cdot 5\cdot 4\cdot 3}{6^5} \approx \text{0,4630}$.
	\part O número de configurações de dois pares de dados de mesmo valor entre 5 dados é $\frac{1}{2!}\binom{5}{2,2,1}$, onde a divisão por 2! vem do fato que não adianta trocar de lugar os grupos de pares, e as escolhas de valores para os dados são $6\cdot 5\cdot 4$. Dessa forma, $P\text{[dois pares]} = \displaystyle\dfrac{1}{2!}\binom{5}{2,2,1}\dfrac{6\cdot 5\cdot 4}{6^5} \approx \text{0,2315}$.
	\part O número de configurações de 3 dados de mesmo valor entre 5 dados é $\binom{5}{3}$ e as escolhas de valores para os dados são $6\cdot 5\cdot 4$. Dessa forma, $P\text{[trinca]} = \displaystyle\binom{5}{3}\dfrac{6\cdot 5\cdot 4}{6^5} \approx \text{0,1543}$.
	\part O número de configurações de 3 dados de mesmo valor e 2 dados de mesmo valor é $\binom{5}{2}$ e as escolhas de valores para os dados são $6\cdot 5$. Dessa forma, $P\text{[uma trinca e um par]} = \displaystyle\binom{5}{3}\dfrac{6\cdot 5}{6^5} \approx \text{0,0386}$.
	\part O número de configurações de 4 dados de mesmo valor entre 5 dados é $\binom{5}{4}$ e as escolhas de valores para os dados são $6\cdot 5$. Dessa forma, $P\text{[quatro dados iguais]} = \displaystyle\binom{5}{4}\dfrac{6\cdot 5}{6^5} \approx \text{0,0193}$.
	\part Cinco dados de mesmo valor só tem uma configuração possível e 6 escolhas para o valor dos dados. Assim, $P\text{[cinco dados iguais]} = \dfrac{6}{6^5} \approx \text{0,0008}$.
\end{parts}
\end{solution}

% 2.18  % % % % % % % % % % % % % % % % % % % % %
\setcounter{question}{17}
\question{Duas cartas são selecionadas
aleatoriamente de um baralho comum. Qual é a
probabilidade de que elas formem um
vinte e um? Isto é, qual é a probabilidade
de que uma das cartas seja um ás e a outra
seja ou um dez, um valete, uma dama ou um rei?
}
\begin{solution}
Temos $\binom{52}{2}$ resultados possíveis, dos quais apenas nos interessa tirar um dos 4 ás do baralho e uma das $4\cdot4$ figuras ou dez. Logo, a probabilidade que buscamos é dada por $\dfrac{4\cdot(4\cdot4)}{\binom{52}{2}} \approx \text{0,0483}$.
\end{solution}

% 2.19  % % % % % % % % % % % % % % % % % % % % %
\question{Dois dados simétricos têm dois de seus lados
pintados de vermelho, dois de preto, um de amarelo e o
outro de branco. Quando esse par de dados é rolado,
qual é a probabilidade de que ambos os dados saiam
com uma face de mesma cor para cima?
}
\begin{solution}
Basta somar as probabilidade de cada evento desejado, ou seja, $\left(\frac{1}{3}\right)^2+\left(\frac{1}{3}\right)^2+\left(\frac{1}{6}\right)^2+\left(\frac{1}{6}\right)^2 = \frac{5}{18} \approx \text{0,2778}$.
\end{solution}

% 2.21  % % % % % % % % % % % % % % % % % % % % %
\setcounter{question}{20}
\question{Uma pequena organização comunitária é
formada por 20 famílias, das quais 4 têm
uma criança, 8 têm duas crianças, 5 têm
três crianças, 2 têm quatro crianças e 1
tem 5 crianças.
\begin{parts}
	\part Se uma dessas famílias é escolhida
aleatoriamente, qual é a probabilidade de que ela tenha $i$ crianças, $i = 1,2,3,4,5$?
	\part Se uma das crianças é escolhida aleatoriamente, qual é a probabilidade de
que a criança venha de uma família
com $i$ crianças, $i= 1,2,3,4,5$?
\end{parts}
}
\begin{solution}
\begin{parts}
	\part 	$P_1 = \frac{4}{20} = \frac{1}{5}$,\\[1mm]
    		$P_2 = \frac{8}{20} = \frac{2}{5}$,\\[1mm]
            $P_3 = \frac{5}{20} = \frac{1}{4}$,\\[1mm]
            $P_4 = \frac{2}{20} = \frac{1}{10}$,\\[1mm]
    		$P_5 = \frac{1}{20}$.
	\part O total de crianças é $4\cdot 1 + 8\cdot 2 + 5\cdot 3 + 2\cdot 4 + 1\cdot 5 = 48$. Logo,\\
    		$P_1 = \frac{4\cdot 1}{48} = \frac{1}{12}$,\\[1mm]
    		$P_2 = \frac{8\cdot 2}{48} = \frac{1}{3}$,\\[1mm]
            $P_3 = \frac{5\cdot 3}{48} = \frac{5}{16}$,\\[1mm]
            $P_4 = \frac{2\cdot 4}{48} = \frac{1}{6}$,\\[1mm]
    		$P_5 = \frac{1\cdot 5}{48} = \frac{5}{48}$.
\end{parts}
\end{solution}

% 2.23  % % % % % % % % % % % % % % % % % % % % %
\setcounter{question}{22}
\question{Rola-se um par de dados honestos. Qual
é a probabilidade de o segundo dado sair
com um valor maior do que o primeiro?
}
\begin{solution}
Temos $6^2$ resultados possíveis, dos quais nos interessam os seguintes resultados: se sair 6 no segundo dado, temos 5 possibilidades para o primeiro, se sair 5 no segundo dado, temos 4 possibilidades para o primeiro, e assim sucessivamente, ou seja, temos $5+4+3+2+1 = 15$ resultados desejados. Logo, a probabilidade do segundo dado ser maior que o primeiro é $\frac{15}{36} = \frac{5}{12} \approx \text{0,41667}$.
\end{solution}

% 2.24  % % % % % % % % % % % % % % % % % % % % %
\question{Se dois dados são rolados, qual é a probabilidade de que a soma das faces para
cima seja igual a $i$? Determine essa probabilidade para $i = 2,3,..., 11,12$.
}
\begin{solution}
O total de resultados ordenados possíveis é $1/6^2$. Para resolver esse problema, basta contar quantos resultados ordenados possíveis levam à soma igual à $i$. Por exemplo, $i=7$ pode ser obtido através dos resultados $(1,6),(5,2),(4,3),(3,4),(2,5)$ e $(1,6)$. Logo, a probabilidade da soma ser $7$ é $6/6^2 = 1/6$. Enfim, podemos verificar que a fórmula geral é dada por 
	\[P_i = \begin{cases}
		 \dfrac{i-1}{6^2}, \quad 2\leq i\leq 7\\[3mm]
         \dfrac{13-i}{6^2}, \quad 7\leq i\leq 12
	\end{cases}\]
\end{solution}

% 2.27  % % % % % % % % % % % % % % % % % % % % %
\setcounter{question}{26}
\question{Uma urna contém 3 bolas vermelhas e 7
bolas pretas. Os jogadores $A$ e $B$ retiram
bolas da urna alternadamente até que
uma bola vermelha seja selecionada. 
Determine a probabilidade de $A$ selecionar
uma bola vermelha. ($A$ tira a primeira
bola, depois $B$,e assim por diante. Não há
devolução das bolas retiradas.)
}
\begin{solution}
A probabilidade de $A$ selecionar uma bola vermelha é dada por
\begin{align*}
P(V) + P(2P,V) + P(4P,V) + P(6P,V)
	&= \dfrac{3}{10} + \dfrac{7}{10}\dfrac{6}{9}\dfrac{3}{8} + \dfrac{7}{10}\dfrac{6}{9}
    	\dfrac{5}{8}\dfrac{4}{7}\dfrac{3}{6} + \dfrac{7}{10}\dfrac{6}{9}\dfrac{5}{8}
        \dfrac{4}{7}\dfrac{3}{6}\dfrac{2}{5}\dfrac{3}{4}\\
    &= \dfrac{7}{12} \approx \text{0,5833},
\end{align*}
onde $nP,V$ significa tirar $n$ bolas pretas antes de tirar uma vermelha.
\end{solution}

% 2.28  % % % % % % % % % % % % % % % % % % % % %
\question{Uma urna contém 5 bolas vermelhas, 6
bolas azuis e 8 bolas verdes. Se um conjunto de 3 bolas
é selecionado aleatoriamente, qual é a probabilidade de que cada
uma das bolas seja (a) da mesma cor? (b)
de cores diferentes? Repita esse problema
considerando que, sempre que uma bola
seja selecionada, sua cor seja anotada e
ela seja recolocada na urna antes da próxima seleção.
Esse experimento é conhecido como
\textit{amostragem com devolução}.
}
\begin{solution}
\begin{parts}
	\part Sem reposição: $\dfrac{5}{19}\dfrac{4}{18}\dfrac{3}{17}+\dfrac{6}{19}\dfrac{5}{18}\dfrac{4}{17}+\dfrac{8}{19}\dfrac{7}{18}\dfrac{6}{17} = \dfrac{86}{969} \approx \text{0,08875}$.\\[1mm]
	Com reposição: $\left(\dfrac{5}{19}\right)^3+\left(\dfrac{6}{19}\right)^3+\left(\dfrac{8}{19}\right)^3 = \dfrac{853}{6859} \approx \text{0,12436}$.
	\part Sem reposição: $3!\,\dfrac{5\cdot 6\cdot 8}{19\cdot 18\cdot 17} = \dfrac{80}{323} \approx \text{0,24768}$.\\[1mm]
	Com reposição: $3!\,\dfrac{5}{19}\dfrac{6}{19}\dfrac{8}{19} = \dfrac{1440}{6859} \approx \text{0,20994}$.
\end{parts}
\end{solution}

% 2.29  % % % % % % % % % % % % % % % % % % % % %
\question{Uma urna contém $n$ bolas brancas e $m$
bolas pretas, onde $n$ e $m$ são números positivos.
\begin{parts}
	\part Se duas bolas são retiradas da urna
aleatoriamente, qual é a probabilidade de que elas sejam da mesma cor?
	\part Se uma bola é retirada da urna aleatoriamente e então recolocada antes
que a segunda bola seja retirada, qual
é a probabilidade de que as bolas sacadas sejam da mesma cor?
	\part Mostre que a probabilidade calculada
na letra (b) é sempre maior do que
aquela calculada na letra (a)?
\end{parts}
}
\begin{solution}
\begin{parts}
	\part Podemos tirar duas bolas brancas ou duas pretas e a probabilidade disso acontecer é dada por $\dfrac{n\,(n-1)+m\,(m-1)}{(n+m)(n+m-1)}$.
	\part Analogamente, $\dfrac{n^2+m^2}{(n+m)^2}$
	\part Manipulando a expressão $\frac{n^2+m^2}{(n+m)^2} \geq \frac{n\,(n-1)+m\,(m-1)}{(n+m)(n+m-1)}$ chega-se em \[\left(\frac{n}{m}+\frac{m}{n}\right)\left(n+m-1\right) \geq \left(\frac{n}{m}+\frac{m}{n}\right)\left(n+m\right) - (n+m)\left(\frac{1}{m}+\frac{1}{n}\right),\]
da qual é fácil chegar em $2 \geq 0$, o que mostra a validade da primeira inequação.
\end{parts}
\end{solution}

% 2.30  % % % % % % % % % % % % % % % % % % % % %
\question{Os clubes de xadrez de duas escolas são
formados por 8 e 9 jogadores, respectivamente. 
Quatro membros de cada um dos
clubes são selecionados aleatoriamente
para participar de uma competição entre
as duas escolas. Os jogadores escolhidos
de um time então formam pares com
aqueles do outro time, e cada um dos pares
jogam uma partida de xadrez entre
si. Suponha que Rebeca e sua irmã
Elisa pertençam aos clubes de xadrez, mas
joguem por escolas diferentes. Qual é a
probabilidade de que
\begin{parts}
	\part Rebeca e Elisa joguem uma partida?
	\part Rebeca e Elisa sejam escolhidas para
representar as suas escolas mas não
joguem uma contra a outra?
	\part Rebeca ou Elisa sejam escolhidas
para representar suas escolas?
\end{parts}
}
\begin{solution}
\begin{parts}
	\part A probabilidade de Rebeca ser selecionada é $4/8$, enquanto que a probabilidade de Elisa ser selecionada é $4/9$. Se ambos eventos aconteceram, então a probabilidade de uma jogar contra a outra é $1/4$. Logo, a probabilidade de uma jogar contra a outra é $\frac{4}{8}\frac{4}{9}\frac{1}{4} = \frac{1}{18} \approx \text{0,056}$.
	\part $\frac{4}{8}\frac{4}{9}\left(1-\frac{1}{4}\right) = \frac{1}{6} \approx \text{0,167}$.
	\part $\frac{4}{8}\left(1-\frac{4}{9}\right)+\frac{4}{9}\left(1-\frac{4}{8}\right) = \frac{1}{2}$.
\end{parts}
\end{solution}

% 2.32  % % % % % % % % % % % % % % % % % % % % %
\setcounter{question}{31}
\question{Um grupo de indivíduos contendo $m$
meninos e $g$ garotas é alinhado de forma
aleatória; isto é, supõe-se que cada uma
das $(m + g)!$ permutações seja igualmente provável. Qual é a probabilidade de
que a pessoa na $i$-ésima posição, $1 \leq i \leq
m + g$, seja uma garota?
}
\begin{solution}
Fixa-se uma garota na $i$-ésima posição, para a qual temos $g$ opções possíveis. Restam $(m+g-1)!$ formas de escolher as outras posições, do total de $(m+g)!$ permutações possíveis. Logo, a probabilidade desejada é dada por $\frac{g(m+g-1)!}{(m+g)!} = \frac{g}{m+g}$.
\end{solution}

% 2.33  % % % % % % % % % % % % % % % % % % % % %
\question{Em uma floresta vivem 20 renas, das quais
5 são capturadas, marcadas e então soltas.
Certo tempo depois, 4 das renas são capturadas.
Qual é a probabilidade de que
2 dessas 4 renas tenham sido marcadas?
Que suposições você está fazendo?
}
\begin{solution}
Supomos que a probabilidade de capturar uma rena seja independente e igual para todas as renas. Existem $\binom{20}{4}$ grupos de 4 renas possíveis de serem capturadas, porém queremos duas renas marcadas, ou seja, $\binom{5}{2}$ possibilidades e 2 renas não marcadas, ou seja, $\binom{15}{2}$ possibilidades. No total, temos $\binom{5}{2}\binom{15}{2}$ formas de escolher 2 renas marcadas e 2 não marcadas. Enfim, a probabilidade disso acontecer é dada por
	\[\dfrac{\binom{5}{2}\binom{15}{2}}{\binom{20}{4}} = \dfrac{70}{323} \approx \text{0,2167}\]
\end{solution}

% 2.35  % % % % % % % % % % % % % % % % % % % % %
\setcounter{question}{34}
\question{Sete bolas são retiradas aleatoriamente
de uma urna que contém 12 bolas vermelhas,
16 bolas azuis e 18 bolas verdes.
Determine a probabilidade de que
\begin{parts}
	\part 3 bolas vermelhas, 2 bolas azuis e 2 bolas verdes sejam sacadas;
	\part pelo menos duas bolas vermelhas sejam sacadas;
	\part todas as bolas sacadas sejam de mesma cor;
	\part exatamente 3 bolas vermelhas ou exatamente 3 bolas azuis sejam sacadas.
\end{parts}
}
\begin{solution}
\begin{parts}
\part Temos $\binom{12}{3}$ formas de escolher as bolas vermelhas, $\binom{16}{2}$ formas de escolher as bolas azuis e $\binom{18}{2}$ as verdes. Logo, a probabilidade é dada por
	\[\dfrac{\binom{12}{3}\binom{16}{2}\binom{18}{2}}{\binom{46}{7}}
    	= \dfrac{3060}{40549} \approx \text{0,0755}.\]
        
	\part O evento complementar seria não tirar nenhuma bola vermelha ou tirar uma bola vermelha. Logo, a probabilidade desejada é
	\[1 - \dfrac{\binom{34}{7}+\binom{12}{1}\binom{34}{6}}{\binom{46}{7}}
    	= \dfrac{363707}{608235} \approx \text{0,598}.\]
        
	\part Calculemos a probabilidade de tirar todas vermelhas ou
	todas azuis ou todas verdes, ou seja,
	\[\dfrac{\binom{12}{7}+\binom{16}{7}+\binom{18}{7}}{\binom{46}{7}}
    	= \dfrac{5507}{6690585} \approx \text{0,0008}.\]

	\part Note que os eventos exatamente 'tirar 3 bolas vermelhas' e 'tirar exatamente 3 bolas azuis' não são disjuntos. Logo, é necessário descontar a probabilidade da intersecção desses eventos, isto é, tirar exatamente 3 bolas azuis e 3 bolas vermelhas. Dessa forma, a probabilidade desejada é dada por
	\[\dfrac{\binom{12}{3}\binom{34}{4}+\binom{16}{3}\binom{30}{4}
    	-\binom{12}{3}\binom{16}{3}\binom{18}{1}}{\binom{46}{7}}
    	= \dfrac{583298}{1338117} \approx \text{0,4359}.\]
\end{parts}
\end{solution}

% 2.36  % % % % % % % % % % % % % % % % % % % % %
\question{Duas cartas são escolhidas aleatoriamente de um baralho de 52 cartas. Qual é a
probabilidade de
\begin{parts}
\part ambas serem ases?
\part ambas terem o mesmo valor?
\end{parts}
}
\begin{solution}
\begin{parts}
\part $\dfrac{1\cdot\binom{4}{2}}{\binom{52}{2}} = \dfrac{1}{221}$.
\part $\dfrac{13\cdot\binom{4}{2}}{\binom{52}{2}} = \dfrac{1}{17}$.
\end{parts}
\end{solution}

% 2.37  % % % % % % % % % % % % % % % % % % % % %
\question{Um instrutor propõe para a classe um
conjunto de 10 problemas com a informação de que o exame final será formado
por uma seleção aleatória de 5 deles. Se
um estudante tiver descoberto como resolver 7 dos problemas, qual é a probabilidade de que ele ou ela venha a-responder corretamente:
\begin{parts}
\part todos os 5 problemas?
\part pelo menos 4 dos problemas?
\end{parts}
}
\begin{solution}
\begin{parts}
\part $\dfrac{\binom{7}{5}}{\binom{10}{5}} = \dfrac{1}{12}$.
\part $\dfrac{\binom{7}{5} + 3\cdot\binom{7}{4}}{\binom{10}{5}} = \dfrac{1}{2}$.
\end{parts}
\end{solution}

% 2.38  % % % % % % % % % % % % % % % % % % % % %
\question{Existem $n$ meias em uma gaveta, 3 das
quais são vermelhas. Qual é o valor de $n$
se a probabilidade de que duas meias vermelhas
sejam retiradas aleatoriamente da
gaveta é igual a $1/2$?
}
\begin{solution}
Existem $\binom{3}{2}$ conjuntos de duas meias vermelhas dos $\binom{n}{2}$ conjuntos possíveis. Logo,
	\[\frac{\binom{3}{2}}{\binom{n}{2}} = \frac{1}{2} \Rightarrow (n+3)(n-4)=0.\]
Assim, $n=-3$ ou $n=4$. Como não podemos ter um número negativo de meias, então a resposta é $n=4$ meias no total.
\end{solution}

% 2.41  % % % % % % % % % % % % % % % % % % % % %
\setcounter{question}{40}
\question{Se um dado é rolado 4 vezes, qual é a
probabilidade de que o 6 saia pelo menos
uma vez?
}
\begin{solution}
A probabilidade do 6 não sair nenhuma vez é dada por $(5/6)^4$. O complemento desse evento é a probabilidade do 6 sair pelo menos uma vez, ou seja, $1-(5/6)^4 \approx \text{0,518}$.
\end{solution}

% 2.43  % % % % % % % % % % % % % % % % % % % % %
\setcounter{question}{42}
\question{
\begin{parts}
	\part Se $N$ pessoas, incluindo $A$ e $B$, são dispostas aleatoriamente em linha, qual
é a probabilidade de que $A$ e $B$ estejam uma ao lado da outra?
	\part E se as pessoas tivessem sido dispostas aleatoriamente em círculo?
\end{parts}
}
\begin{solution}
Seja $N\geq 3$.
\begin{parts}
	\part As configurações nas quais $A$ e $B$ estão uma ao lado da outra são $2\cdot(N-1)!$ do total de configurações possíveis $N!$. Logo, a probabilidade desejada é $2/N$.
	\part Numa mesa circular não tem início, então as configurações possíveis se reduzem à $(N-1)!$.  Fixando a posição do $A$, temos que as configurações desejadas são o $B$ sentar-se à esquerda ou à direita de $A$, e para o restante das pessoas, temos $(N-2)!$ configurações possíveis. Portanto, a probabilidade desejada é \[\frac{2\cdot(N-2)!}{(N-1)!} = \frac{2}{N-1},\]
que é maior do que $2/N$.
\end{parts}
\end{solution}

% 2.44  % % % % % % % % % % % % % % % % % % % % %
\setcounter{question}{43}
\question{Cinco pessoas, designadas como $A, B, C,
D, E$, são arranjadas em uma sequência linear. Supondo
que cada uma das ordenações possíveis seja igualmente provável,
qual é a probabilidade de que:
\begin{parts}
	\part exista exatamente uma pessoa entre $A$ e $B$?
	\part existam exatamente duas pessoas entre $A$ e $B$?
	\part existam três pessoas entre $A$ e $B$?
\end{parts}
}
\begin{solution}
\begin{parts}
	\part Temos duas opções em relação à $A$ ou $B$ vir primeiro; temos 3 opções em relação à pessoa que está no meio; temos $3!$ arranjos possíveis entre o grupo e o restante das pessoas. O total de arranjos possíveis é $5!$. Portanto, a probabilidade desejada é $(2\cdot 3\cdot 3!)/5!=3/10$.
	\part Aplicando raciocínio análogo ao item anterior temos $(2\cdot(3\cdot2)\cdot2!)/5! = 1/5$.
	\part Analogamente, temos $(2\cdot(3\cdot2\cdot1)\cdot1!)/5! = 1/10$.
\end{parts}
\end{solution}

% 2.45  % % % % % % % % % % % % % % % % % % % % %
\question{Uma mulher tem $n$ chaves,das quais uma
abre a sua porta
\begin{parts}
	\part Se ela tentar usar as chaves aleatoriamente,
descartando aquelas que não funcionam, qual é a probabilidade de
ela abrir a porta em sua $k$-ésima tentativa?
	\part E se ela não descartar as chaves já utilizadas?
\end{parts}
}
\begin{solution}
\begin{parts}
	\part Para abrir a porta exatamente na $k$-ésima tentativa, ela deve falhar $k-1$ vezes e obter sucesso na $k$-ésima vez. A probabilidade desse evento é
    	\[\dfrac{N-1}{N}\,\dfrac{N-2}{N-1}\cdots\dfrac{N-(k-1)}{N-(k-2)}\,\dfrac{1}{N-(k-1)}
        	= \dfrac{1}{N}.\]
	\part Quando não se descarta chaves utilizadas, a probabilidade se torna
		\[\left(\dfrac{N-1}{N}\right)^{k-1}\dfrac{1}{N}.\]
\end{parts}        
\end{solution}

% 2.48  % % % % % % % % % % % % % % % % % % % % %
\setcounter{question}{47}
\question{Dadas 20 pessoas, qual é a probabilidade
de que, entre os 12 meses do ano, existam
4 meses contendo exatamente 2 aniversários
e 4 contendo exatamente 3 aniversários?
}
\begin{solution}
Supomos que a probabilidade de uma pessoa fazer em um determinado mês seja igual à qualquer outro mês e independente das demais pessoas. Os meses são divididos em 3 grupos de 4 meses, os com nenhum aniversariante, com 2 aniversariantes e 3 aniversariantes. Assim, temos $\binom{12}{4,4,4}$ possibilidades de meses. Em relação à distribuição dos aniversariantes, contamos quantas possibilidades existem quando dividimos os mesmos em 4 grupos de 2 e 4 grupos de 3, ou seja, $\binom{20}{2,2,2,2,3,3,3,3}$ possibilidades. O total de possibilidades de aniversário é dada por $12^{20}$. Portanto, a probabilidade desejada é dada por
	\[\dfrac{1}{12^{20}}\binom{12}{4,4,4}\binom{20}{2,2,2,2,3,3,3,3} \approx\text{0,00106}.\]
\end{solution}

% 2.49  % % % % % % % % % % % % % % % % % % % % %
\question{Um grupo de 6 homens e 6 mulheres é
dividido aleatoriamente em 2 grupos de 6
pessoas cada. Qual é a probabilidade de
que ambos os grupos possuam o mesmo
número de homens?
}
\begin{solution}
A única forma de terem o mesmo número de homens nos dois grupos é se tiverem 3 homens em cada grupo. O número de possibilidade de fazer isso é escolher os 3 homens e escolher as 3 mulheres para o primeiro grupo, ou seja, $\binom{6}{3}^2$ possibilidades. O total de divisões possíveis de 12 pessoas em 2 grupos é $\binom{12}{6}$. Logo, a probabilidade que buscamos é
	\[\dfrac{\binom{6}{3}^2}{\binom{12}{6}}
    	= \dfrac{100}{231} \approx \text{0,433}\]
\end{solution}

% 2.52  % % % % % % % % % % % % % % % % % % % % %
\setcounter{question}{51}
\question{Um armário contém 10 pares de sapatos.
Se 8 sapatos são selecionados aleatoriamente, qual é a probabilidade de
\begin{parts}
	\part nenhum par completo ser formado?
	\part ser formado exatamente 1 par completo?
\end{parts}
}
\begin{solution}
\begin{parts}
	\part Para nenhum par completo, escolhemos 8 dos 10 pares disponíveis, ou seja, $\binom{10}{8}$ e para cada par escolhido temos 2 possibilidades de sapato. O total de combinações possíveis de serem escolhidas é $\binom{20}{8}$. Logo, supondo probabilidades iguais para escolha de cada sapato, temos que a probabilidade de não tirar nenhum par completo é
    \[\dfrac{2^8\binom{10}{8}}{\binom{20}{8}} \approx \text{0,091}.\]
    
	\part Para formar exatamente um par completo, escolhemos um dos 10 pares para ser o par completo e 6 dos 9 pares restantes para ser os incompletos, lembrando que para cada par incompleto temos 2 possibilidades. Assim, temos que a probabilidade de exatamente um par completo é
    \[\dfrac{10\cdot 2^6\binom{9}{6}}{\binom{20}{8}} \approx \text{0,427}.\]
\end{parts}        
\end{solution}

\end{questions}
