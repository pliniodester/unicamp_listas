
\documentclass[answers, 12pt]{exam}
%\usepackage{amsmath}
\usepackage{amsthm}
%\usepackage{amsfonts}
\usepackage{amssymb}
\usepackage{mathrsfs}
\usepackage{graphicx}
\usepackage{mathtools}
\usepackage{caption}
\usepackage[brazil]{babel}
\usepackage[utf8]{inputenc}

\usepackage{url}

\usepackage{multicol}

\usepackage{tikz}
\usepackage{pgfplots}
\pgfplotsset{compat=1.15}

\newcommand{\Var}{\mathrm{Var}}
\newcommand{\euler}{\mathrm{e}}
\newcommand\diff{\mathrm{d}}

\renewcommand{\qedsymbol}{$\blacksquare$}
\renewcommand{\thequestion}{\arabic{section}.\arabic{question}}
\renewcommand{\solutiontitle}{\noindent\textbf{Solução:}\enspace}

\newtheorem{theorem}{Teorema}

\def\C{{\mathbb C}}
\def\N{{\mathbb N}}
\def\R{{\mathbb R}}
\def\Z{{\mathbb Z}}
\def\Q{{\mathbb Q}}
\def\E{{\mathbb E}}
\def\X{{\mathbb X}}
\def\ind{\mathds{1}}
\def\cal{\mathcal}
\def\T{\top}

\footer{}{\thepage}{}

\title{	ME210 - Probabilidade I - 2S 2017\\
		{\large \textit{Docente}: Marina Vachkovskaia}\\[2mm]
		{\large Soluções para problemas selecionados do livro\\[-2mm]
        \textit{Probabilidade: Um curso moderno com aplicações}
        8.ed. de Sheldon Ross}\\
}
\author{Plínio Santini Dester (p103806@dac.unicamp.br)}

\begin{document}

%% Content goes here
\maketitle

Em caso de dúvidas, sugestões ou correções (inclusive erros de digitação), não hesite em mandar um e-mail.

\setcounter{section}{1}
\section{Problemas}

\begin{questions}

% 2.3  % % % % % % % % % % % % % % % % % % % % %
\setcounter{question}{2}
\question{
{\bf (Lançamento Oblíquo)} Um projétil é disparado com um ângulo de elevação $\alpha$ medido a partir da horizontal e velocidade inicial $v_0$. Assumindo que a resistência do ar seja desprezível e que a única força externa sobre o móvel seja a força peso, determine o vetor de posição do móvel $\vec r(t)$ para $t \in [0,t_f]$, sendo $t_f$ o tempo que o móvel leva até atingir o solo novamente. Para que valor de $\alpha$ obtemos o maior alcance horizontal? Em que ponto da trajetória a curvatura é máxima? Interprete geometricamente.
}

\begin{solution}
    Sabemos que a posição do móvel
    \begin{align*}
        \vec r(t) = \left( v_0 \cos(\alpha)\,t, v_0 \sin(\alpha)\,t - g\,t^2/2 \right), \quad t \in [0,t_f].
    \end{align*}
    Ademais, $t_f>0$ deve satisfazer
    \begin{align*}
        v_0 \sin(\alpha)\,t_f - g\,t_f^2/2 = 0
            ~\Rightarrow t_f (v_0 \sin(\alpha) - g\,t_f/2) = 0
            ~\Rightarrow t_f = \frac{2 v_0}{g} \sin(\alpha).
    \end{align*}
    Dessa forma, o alcance horizontal máximo
    \begin{align*}
        D_H = v_0 \cos(\alpha)\,t_f =  \frac{v_0^2}{g}\,2 \sin(\alpha)\cos(\alpha)
            = \frac{v_0^2}{g} \sin(2\alpha) \le \frac{v_0^2}{g},
    \end{align*}
    e satisfaz a igualdade para $\alpha = \pi/4$.
    
    A curvatura é dada por
    \begin{align*}
        \kappa(t) 
            = \frac{|\dot{\vec r}(t) \times \ddot{\vec r}(t)|}{|\dot{\vec r}(t)|^3} 
            = \frac{|(v_0 \cos(\alpha),v_0 \sin(\alpha) - g\,t) \times (0,-g)|}{|(v_0 \cos(\alpha),v_0 \sin(\alpha) - g\,t)|^3}
            = \frac{g v_0 \cos(\alpha)}{(v_0^2 - 2 g v_0 \sin(\alpha)\,t+ g^2 t^2 )^{3/2}},
    \end{align*}
    que atinge um máximo em $t^* = \frac{v_0}{g} \sin(\alpha)$. Nesse instante, $\kappa(t^*) = \frac{g}{v_0^2}\sec^2(\alpha)$.
    
    O ponto de máxima curvatura
        $
        \vec r(t^*) = \frac{v_0^2}{2g} \left(\sin(2\alpha), \sin^2(\alpha) \right)
        $
    corresponde ao máximo da trajetória, ou seja, é o ponto no qual toda força peso atua como normal, portanto faz sentido ser o ponto de máxima curvatura.
    %
    Ademais, a trajetória é parabólica e esse ponto trata-se do vértice da parábola, que é o ponto de maior curvatura.
\end{solution}

\end{questions}


\setcounter{section}{1}
\section{Exercícios Teóricos}
\begin{questions}
% 2.1   % % % % % % % % % % % % % % % % % % % % %
\question{
Prove que $E\,F \subset E \subset E \cup F$.
}
\begin{solution}
\begin{proof}
	~\\
	Seja $\omega \in E\,F$. Por definição de intersecção, $\omega \in E$ e $\omega \in F$. Em particular, $\omega \in E$. Logo, $E\,F \subset E$.\\
    Seja $\omega' \in E$. Então, $\omega' \in E$ ou $\omega' \in F$. Por definição de união, $\omega' \in E \cup F$. Logo, $E \subset E \cup F$.
\end{proof}
\end{solution}

% 2.2   % % % % % % % % % % % % % % % % % % % % %
\question{
Prove que se $E \subset F$, então $F^c \subset E^c$.
}
\begin{solution}
\begin{proof}
    \begin{align*}
    	E \subset F 
        	&\Leftrightarrow \forall \omega \in \Omega, 
            	(\omega \in E \Rightarrow \omega \in F)\\
			&\Leftrightarrow \forall \omega \in \Omega, 
            	(\omega \notin F \Rightarrow \omega \notin E)\\
			&\Leftrightarrow F^c \subset E^c.
    \end{align*}
\end{proof}
\end{solution}

% 2.3   % % % % % % % % % % % % % % % % % % % % %
\question{
Prove que $F = F\,E\cup F\,E^c$ e $E\cup F = E\cup E^c\,F$.
}
\begin{solution}
\begin{proof}
Primeiramente, vamos mostrar que $F \subset F\,E\cup F\,E^c$. Seja $\omega \in F$. Se $\omega \in E$, então $\omega \in F\,E$, por outro lado, se $\omega \notin E$, então $\omega \in F\,E^c$. Logo, $\omega \in F\,E$ ou $\omega \in F\,E^c$. Por definição de união. $\omega \in F\,E\cup F\,E^c$.\\
Por outro lado, do exercício 2.1, temos que $F\,E\cup F\,E^c \subset F \cup F = F$. Logo,  como temos inclusão nos dois sentidos $F = F\,E\cup F\,E^c$.\\

Agora, mostremos que $E\cup F \subset E\cup E^c\,F$. Seja $\omega \in E\cup F$. Se $\omega \notin E$, então $\omega \in F$, isto é, $\omega \in F\,E^c$. Logo, $\omega \in E$ ou $\omega \in F\,E^c$. Por definição de união, $\omega \in E\cup E^c\,F$.\\
Por outro lado, do exercício 2.1, $E\cup E^c\,F \subset E\cup F$. Assim, como temos inclusão nos dois sentidos, $E\cup F = E\cup E^c\,F$.
\end{proof}
\end{solution}

% 2.6   % % % % % % % % % % % % % % % % % % % % %
\setcounter{question}{5}
\question{Sejam três eventos $E$, $F$ e $G$. Determine
expressões para esses eventos de forma que, de $E$, $F$ e $G$,
\begin{parts}
	\part apenas $E$ ocorra;
    \part $E$ e $G$ ocorram, mas não $F$;
    \part pelo menos um dos eventos ocorra;
    \part pelo menos dois dos eventos ocorram;
    \part todos os três eventos ocorram;
    \part nenhum dos eventos ocorra;
    \part no máximo um dos eventos ocorra;
    \part no máximo dois dos eventos ocorram;
    \part no máximo três dos eventos ocorram.
\end{parts}
}
\begin{solution}
\begin{parts}
	\part $E \, F^c \, G^c$;
    \part $E \, F^c \, G$;
    \part $(E^c\,F^c\,G^c)^c = E \cup F \cup G$;
    \part $E\,F\,G \cup E\,F\,G^c \cup E\,F^c\,G \cup E^c\,F\,G$;
    \part $E\,F\,G$;
    \part $E^c\,F^c\,G^c = (E\cup F\cup G)^c$;
    \part $E^c\,F^c\,G^c \cup E\,F^c\,G^c \cup E^c\,F\,G^c \cup E^c\,F^c\,G$;
    \part $(E\,F\,G)^c = E^c\cup F^c\cup G^c$;
    \part $\Omega$.
\end{parts}
\end{solution}

% 2.7   % % % % % % % % % % % % % % % % % % % % %
\question{Determine a expressão mais simples para
os seguintes eventos:
\begin{parts}
	\part $(E\cup F)(E\cup F^c)$;
    \part $(E\cup F)(E^c\cup F)(E\cup F^c)$;
    \part $(E\cup F)(F\cup G)$;
\end{parts}
}
\begin{solution}
\begin{parts}
	\part Pela lei distributiva,
    	$(E\cup F)(E\cup F^c) = E\cup F\,F^c = E$.
    \part Do item (a), temos que
    	$(E\cup F)(E^c\cup F)(E\cup F^c) = E\,(E^c\cup F) = E\,F$.
    \part Pela lei distributiva,
    $(E\cup F)(F\cup G) = F\cup E\,G$.
\end{parts}
\end{solution}

% 2.10  % % % % % % % % % % % % % % % % % % % % %
\setcounter{question}{9}
\question{Demonstre que $P(E\cup F\cup G) = P(E)+P(F)+P(G)-P(E^c\,F\,G)-P(E\,F^c\,G)-P(E\,F\,G^c)-2\,P(EFG)$.
}
\begin{solution}
\begin{proof}
	Aplicando a Proposição~4.3 sucessivas vezes, obtemos que
    \begin{align*}
    	P( E\cup F\cup G) 
        &= P( (E\cup F)\cup G) \\
        &= P(E\cup F) + P(G) - P((E\cup F)\,G) \\
        &= P(E) + P(F) + P(G) - P(E\,F) - P(E\,G\cup F\,G) \\
        &= P(E) + P(F) + P(G) - P(E\,F) - P(E\,G) - P(F\,G) + P(E\,F\,G),
    \end{align*}
    que é um caso particular da Proposição~4.4. \\
    Sabemos que $P(E\,F) = P(E\,F\,G) + P(E\,F\,G^c)$, pois $E\,F\,G$ e $E\,F\,G^c$ são eventos disjuntos, cuja união resulta em $E\,F$. Podemos fazer o mesmo para $P(E\,G)$ e $P(F\,G)$. Substituindo essas probabilidades na expressão acima completa a demonstração.
\end{proof}
\end{solution}

% 2.11  % % % % % % % % % % % % % % % % % % % % %
\question{Se $P(E) = \text{0,9}$ e $P(F) = \text{0,8}$, mostre que
$P(E\,F) \geq \text{0,7}$. De forma geral,demonstre a
desigualdade de Bonferroni, isto é, $P(E\,F) \geq P(E)+P(F)-1$.
}
\begin{solution}
\begin{proof}
	Da Proposição 4.3 e do Axioma 1, sabemos que 
	\[P(E)+P(F)-P(E\,F) = P(E\cup F) \leq 1.\]
    Logo, $P(E\,F) \geq P(E)+P(F)-1$.
\end{proof}
\end{solution}

\end{questions}

\vspace{3mm} {\LARGE \textbf{Desafio!}}
\begin{enumerate}
\item Uma mesa redonda tem $n$ assentos disponíveis. Suponha que em cada assento sente-se um homem ou uma mulher com igual probabilidade. Encontre a probabilidade de nenhuma mulher sentar-se ao lado de outra mulher. Mostre que essa probabilidade tende à $0$ quando $n$ tende à infinito.
\end{enumerate}

\end{document}