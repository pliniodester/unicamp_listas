
\begin{questions}

% 3.4  % % % % % % % % % % % % % % % % % % % % %
\setcounter{question}{3}
\question{Qual é a probabilidade de que pelo menos
um de um par de dados honestos caia no 6,
dado que a soma dos dados seja $i$, $i
= 2,3,\dots,12$?
}
\begin{solution}
	Sejam os eventos $A$ de sair pelo menos um 6 e $B_i$ da soma dos dados ser igual à $i$, temos que
    \[P(A\mid B_i) = \dfrac{P(A\,B_i)}{P(B_i)}.\]
    A probabilidade $P(B_i)$ já foi calculada no Problema 2.24. A intersecção de ambos eventos é simples de ser contada. De fato, se $2 \le i \le 6$, então $A\,B_i = \emptyset$, pois não pode ter saído nenhum 6 para a soma ser menor ou igual à 6. Se $7 \le i \le 11$, temos 2 elementos possíveis do espaço amostral: $\{(6,i-6),(i-6,6)\}$. Enfim, se $i = 12$ temos apenas um elemento possível: $\{(6,6)\}$. Com isso é fácil mostrar que
    \[P(A\mid B_i) = \begin{cases}
		 0, & 2\leq i\leq 6\\[3mm]
         \dfrac{2}{13-i}, & 7\leq i\leq 11\\[3mm]
         1, & i = 12.
	\end{cases}\]
\end{solution}

% 3.5  % % % % % % % % % % % % % % % % % % % % %
\question{Uma urna contém 6 bolas brancas e 9 bolas pretas.
Se 4 bolas devem ser selecionadas aleatoriamente sem
devolução, qual é a probabilidade de que as 2 primeiras
bolas selecionadas sejam brancas e as 2 últimas sejam pretas?
}
\begin{solution}
	Seja $A$ o evento duas primeiras bolas brancas e $B$ o evento duas últimas bolas pretas. Então,
    \[P(A\,B) = P(B\mid A)\,P(A)
    	= \dfrac{9\cdot 8}{13\cdot 12}\,\dfrac{6\cdot 5}{15\cdot 14}
        = \dfrac{6}{91} \approx \text{0,0659}.\]
\end{solution}

% 3.9  % % % % % % % % % % % % % % % % % % % % %
\setcounter{question}{8}
\question{Considere 3 urnas. A urna $A$ contém 2 bolas brancas e 4 bolas vermelhas, a urna $B$ contém 8 bolas brancas e 4 bolas vermelhas, e a urna $C$ contém 1 bola branca e 4 bolas vermelhas. Se 1 bola é selecionada
de cada urna, qual é a probabilidade de que a bola escolhida da urna $A$ seja branca dado que exatamente 2 bolas brancas tenham sido selecionadas?
}
\begin{solution}
	Seja $E$ o evento selecionar exatamente 2 bolas brancas e $F$ o evento selecionar uma bola branca da urna $A$, então o evento $E\,F$ é selecionar uma bola branca da urna $A$ e exatamente uma bola branca de outra urna. Assim,
    \[P(F\mid E) = \dfrac{P(E\,F)}{P(E)}
    	= \dfrac{\frac{2}{6}\frac{8}{12}\frac{4}{5}
        	+ \frac{2}{6}\frac{4}{12}\frac{1}{5}}
            {\frac{2}{6}\frac{8}{12}\frac{4}{5}
        	+ \frac{2}{6}\frac{4}{12}\frac{1}{5}
            + \frac{4}{6}\frac{8}{12}\frac{1}{5}}
		= \dfrac{9}{13} \approx \text{0,6923}.\]
\end{solution}

% 3.10 % % % % % % % % % % % % % % % % % % % % %
\question{\textit{O enunciado em português está confuso, então usarei o enunciado em inglês:}\\
Three cards are randomly selected, without
replacement, from an ordinary deck of 52 playing
cards. Compute the conditional probability that
the first card selected is a spade given that the second and third cards are spades.
}
\begin{solution}
	Seja $E$ o evento a primeira carta selecionada é do naipe de espadas e seja $F$ o evento a segunda e terceira cartas selecionadas são de espada. Então, $E\,F$ representa o evento das três cartas serem de espada. Para calcular $P(E)$ separamos em dois casos, o caso que tiramos uma carta de espadas na primeira e o caso onde não tiramos uma carta de espadas na primeira. Enfim, temos
    \[P(E\mid F) = \dfrac{P(E\,F)}{P(E)}
    	= \dfrac{\frac{13\cdot 12\cdot 11}{52\cdot 51\cdot 50}}
        	{\frac{13\cdot 12\cdot 11}{52\cdot 51\cdot 50}
        	+\frac{39\cdot 13\cdot 12}{52\cdot 51\cdot 50}}
		= \dfrac{11}{50} = \text{0,22}.\]
\end{solution}

% 3.12 % % % % % % % % % % % % % % % % % % % % %
\setcounter{question}{11}
\question{Uma recém-formada planeja realizar as
primeiras três provas em atuária no próximo verão.
Ela fará a primeira prova em dezembro,
depois a segunda prova em janeiro
e, se ela passar em ambas, fará a terceira
prova em fevereiro. Se ela for reprovada
em alguma prova, então não poderá fazer
nenhuma das outras. A probabilidade de
ela passar na primeira prova é de 0,9. Se ela
passar na primeira prova, então a probabilidade
condicional de ela passar na segunda prova é de 0,8.
Finalmente, se ela passar tanto na primeira quanto
na segunda prova, então a probabilidade condicional de
ela passar na terceira prova é de 0,7.
\begin{parts}
	\part Qual é a probabilidade de ela passar
em todas as três provas?
	\part Dado que ela não tenha passado em
todas as três provas, qual é a probabilidade condicional de ela ter sido
reprovada na segunda prova?
\end{parts}
}
\begin{solution}
\begin{parts}
	\part Sejam os eventos $A$, $B$ e $C$ passar na primeira, segunda e terceira prova, respectivamente. Então, usando a fórmula de Bayes, temos
    \begin{align*}
    	P(A\,B\,C) &= P(A\,B\,C\mid A\,B) P(A\,B\mid A) P(A)\\
        	&= P(C\mid B) P(B\mid A) P(A)\\
            &= \text{0,9}\cdot \text{0,8}\cdot \text{0,7} = \text{0,504}.
    \end{align*}
    Note que $P(C\mid B^c) = P(B\mid A^c) = 0$.
	\part Note que $A\,B^c \subset B^c \subset A^c\cup B^c\cup C^c = (A\,B\,C)^c$. Assim,
    \begin{align*}
		P(A\,B^c\mid (A\,B\,C)^c)
        	&= \dfrac{P(A\,B^c\,(A\,B\,C)^c)}{P((A\,B\,C)^c)}\\
        	&= \dfrac{P(B^c\mid A) P(A)}{1 - P(A\,B\,C)}\\
            &= \dfrac{(1-\text{0,8})\cdot \text{0,9}}{1-\text{0,504}}\\
            &\approx \text{0,363}.
	\end{align*}
\end{parts}
\end{solution}

% 3.15 % % % % % % % % % % % % % % % % % % % % %
\setcounter{question}{14}
\question{Uma gravidez ectópica é duas vezes mais
provável em mulheres fumantes do que
em mulheres não fumantes. Se 32\% das
mulheres na idade reprodutiva são fumantes, 
que percentual de mulheres com
gravidez ectópica são fumantes?
}
\begin{solution}
	O enunciado da pergunta é equivalente a calcular a probabilidade de uma mulher gravida ser fumante dado que ela tem gravidez ectópica. Seja o evento $A$ a mulher ser fumante e $B$ ela ter gravidez ectópica. Sabemos que ${P(B\mid A) = 2 P(B\mid A^c)}$. Estamos interessados em
    \begin{align*}
    	P(A\mid B) 
        	&= \dfrac{P(A\,B)}{P(B)}\\
        	&= \dfrac{P(B\mid A)P(A)}{P(B\mid A)P(A) + P(B\mid A^c)P(A^c)}\\
            &= \dfrac{P(A)}{P(A) + \frac{P(B\mid A^c)}{P(B\mid A)}P(A^c)}\\
            &= \dfrac{\text{0,32}}{\text{0,32} + \frac{1}{2}(1-\text{0,32})}\\
            &\approx \text{0,4848}.
    \end{align*}
\end{solution}

% 3.16 % % % % % % % % % % % % % % % % % % % % %
%\setcounter{question}{14}
\question{Noventa e oito por cento de todas as crianças sobrevivem ao parto. Entretanto, 15\% de todos os nascimentos envolvem cesarianas (C), e quando uma cesariana é feita os bebês sobrevivem 96\% dos casos. Se
uma gestante aleatoriamente selecionada não fez uma cesariana, qual é a probabilidade de que seu bebê sobreviva?
}
\begin{solution}
	Seja $S$ o evento da criança sobreviver e $C$ o evento da grávida fazer cesariana. Estamos interessados na probabilidade
    \begin{align*}
    	P(S\mid C^c) 
        &= \dfrac{P(S\,C^c)}{P(C^c)} = \dfrac{P(S)-P(S\,C)}{1-P(C)}
    		= \dfrac{P(S)-P(S\mid C)P(C)}{1-P(C)}\\
        &= \dfrac{\text{0,98}-\text{0,96}\cdot\text{0,15}}{1-\text{0,15}}
        	\approx \text{0,9835}.
    \end{align*}
\end{solution}

% 3.17 % % % % % % % % % % % % % % % % % % % % %
%\setcounter{question}{14}
\question{Em certa comunidade, 36\% das famílias
têm um cão e 22\% das famílias que possuem um cão também possuem um gato.
Além disso, 30\% das famílias têm um gato. Qual é
\begin{parts}
	\part a probabilidade de que uma família
aleatoriamente selecionada tenha
tanto um cão quanto um gato?
    \part a probabilidade condicional de que
uma família aleatoriamente selecionada tenha
um cão dado que ela também tenha um gato?
\end{parts}
}
\begin{solution}
Sejam os eventos $C$, $G$ a família ter um cão e um gato, respectivamente.
\begin{parts}
	\part $P(C\,G) = P(G\mid C)P(C)
    	= \text{0,22}\cdot\text{0,36} = \text{0,0792}.$
    \part $P(C\mid G) = \dfrac{P(C\, G)}{P(G)}
    	= \dfrac{\text{0,0792}}{\text{0,3}} = \text{0,264}.$
\end{parts}
\end{solution}

% 3.19 % % % % % % % % % % % % % % % % % % % % %
\setcounter{question}{18}
\question{Um total de 48\% das mulheres e 37\% dos
homens que fizeram um curso para largar
o cigarro seguiu sem fumar por pelo menos um ano após o final do curso. Essas
pessoas frequentaram uma festa de comemoração no final do ano. Se 62\% da
turma original era de homens
\begin{parts}
	\part que percentual daqueles que foram à
festa era de mulheres?
    \part que percentual da turma original foi à
festa?
\end{parts}
}
\begin{solution}
Sejam os eventos $A$, $H$ uma pessoa escolhida ao acaso da turma original parar de fumar e ser homem, respectivamente.
\begin{parts}
	\part ~\vspace{-9mm}
    \begin{align*}
    	P(H^c\mid A)
    	&= \dfrac{P(A\mid H^c)P(H^c)}{P(A\mid H^c)P(H^c)+P(A\mid H)P(H)}\\
    	&= \dfrac{\text{0,48}\,(1-\text{0,62})}
        	{\text{0,48}\,(1-\text{0,62})+\text{0,37}\cdot\text{0,62}}\\
        &\approx \text{0,443}.
    \end{align*}
    \part $P(A) = P(A\mid H^c)P(H^c)+P(A\mid H)P(H) = \text{0,412}$.
\end{parts}
\end{solution}

% 3.22 % % % % % % % % % % % % % % % % % % % % %
\setcounter{question}{21}
\question{Um dado vermelho, um azul e um amarelo (todos com seis lados) são rolados. Estamos interessados na probabilidade de
que o número que sair no dado azul seja
menor do que aquele que sair no dado
amarelo, e que este seja menor do que
aquele que sair no dado vermelho. Isto
é, com $B$, $Y$ e $R$ representando, respectivamente, os números que aparecem nos
dados azul, amarelo e vermelho, estamos
interessados em $P(B < Y < R)$.
\begin{parts}
	\part Qual é a probabilidade de que um mesmo número não saia em dois dados?
    \part Dado que um mesmo número não saia em dois dos dados, qual é a probabilidade condicional de que $B < Y < R$?
	\part O que é $P(B < Y < R)?$
\end{parts}
}
\begin{solution}
\begin{parts}
	\part Seja o evento $D$ não sair dados com o mesmo número,
    	então ${P(D) = \dfrac{5}{6}\,\dfrac{4}{6} = \dfrac{5}{9}}$.
    \part Por simetria, qualquer ordenação de dados é equiprovável.\\
    	Logo, ${P(B < Y < R\mid D) = \dfrac{1}{3!} = \dfrac{1}{6}}$.
    \part Como $(B<Y<R)\subset D$, então\\
    	$P(B < Y < R) = P(B < Y < R\mid D)P(D) = \dfrac{5}{54}$.
\end{parts}
\end{solution}

% 3.25 % % % % % % % % % % % % % % % % % % % % %
\setcounter{question}{24}
\question{O seguinte método foi proposto para se
estimar o número de pessoas com idade
acima de 50 anos que moram em uma
cidade com população conhecida de
100.000: ``A medida que você caminhar
pela rua, mantenha uma contagem contínua
do percentual de pessoas que você
encontrar com idade acima de 50 anos.
Faça isso por alguns dias; então multiplique
o percentual obtido por 100.000 para
obter a estimativa desejada.'' Comente esse método.\\
\textit{Dica}: Seja $p$ a proporção de pessoas na
cidade que tem idade acima de 50 anos.
Além disso, suponha que $\alpha_1$, represente a
proporção de tempo que uma pessoa com
idade abaixo de 50 anos passa nas ruas, e
que $\alpha_2$, seja o valor correspondente para
aqueles com idade acima de 50 anos. Que
tipo de grandeza estima o método sugerido?
Quando é que a estimativa é aproximadamente igual a $p$?
}
\begin{solution}
	O método sugerido estima a porcentagem de pessoas acima de 50 anos que caminham pelas ruas, o que pode ser diferente da porcentagem de pessoas acima de 50 anos na população. Seja o evento $C$ uma pessoa selecionada ao acaso estar caminhando nas ruas e o evento $D$ uma pessoa selecionada ao acaso ter mais de 50 anos, então o que estimamos é
    \[P(D\mid C) = \dfrac{P(C\mid D)P(D)}{P(C\mid D)P(D)+P(C\mid D^c)P(D^c)}
    	= \dfrac{\alpha_2\,p}{\alpha_2\,p + \alpha_1\,(1-p)}.\]
	Podemos verificar que a estimativa é razoável quando $\alpha_1\approx \alpha_2$.
\end{solution}

% 3.31 % % % % % % % % % % % % % % % % % % % % %
\setcounter{question}{30}
\question{Dona Aquina acabou de fazer uma biópsia para verificar a existência de um tumor cancerígeno. Evitando estragar um
evento de família no final de semana, ela
não quer ouvir nenhuma má notícia nos
próximos dias. Mas se ela disser ao médico para telefonar-lhe somente se as notícias forem boas, então, se o médico não ligar, Dona Aquina pode concluir que as
notícias não são boas. Assim, sendo uma estudante de probabilidade, Dona Aquina instrui o médico a jogar uma moeda, se der cara, o doutor deverá telefonar-lhe
se as novidades forem boas e não telefonar-lhe se elas forem más. Se a moeda der
coroa, o médico não deverá telefonar-lhe.
Dessa maneira, mesmo se o médico não telefonar-lhe, as notícias não serão necessariamente más. Seja $\alpha$ a probabilidade de que o tumor seja cancerígeno e $\beta$ a probabilidade condicional de que o tumor seja cancerígeno dado que o médico
não faça o telefonema.
\begin{parts}
	\part Qual probabilidade é maior, $\alpha$ ou $\beta$?
    \part Determine $\beta$ em termos de $\alpha$ e demonstre a sua resposta da letra (a).
\end{parts}
}
\begin{solution}
Seja o evento $C$ do tumor ser cancerígeno e $T$ do médico telefonar, então
\begin{parts}
	\part Devemos ter $\beta \ge \alpha$, pois ter câncer é condição suficiente para o médico não ligar.
    \part Mais rigorosamente, temos que
    \begin{align*}
    	\beta &= P(C\mid T^c)
    			= \dfrac{P(T^c\mid C)P(C)}{P(T^c\mid C)P(C)+P(T^c\mid C^c)P(C^c)}\\
        	&= \dfrac{1\cdot \alpha}{1\cdot \alpha + \frac{1}{2}(1-\alpha)}
        		= \dfrac{2\alpha}{1+\alpha}\\
            &\ge \dfrac{2\alpha}{2} = \alpha.
    \end{align*}
\end{parts}
\end{solution}

% 3.32 % % % % % % % % % % % % % % % % % % % % %
%\setcounter{question}{30}
\question{Uma família tem $j$ crianças com
probabilidade $p_j$, onde $p_1 = \text{0,1},
p_2 = \text{0,25}, p_3 =\text{0,35}, p_4 = \text{0,3}$.
Uma criança desta família é
escolhida aleatoriamente. Dado que ela é
a primogênita, determine a probabilidade
condicional de que a família tenha:
\begin{parts}
	\part apenas 1 criança;
    \part 4 crianças.
\end{parts}
Repita as letras (a) e (b) quando a criança
selecionada aleatoriamente for a caçula.
}
\begin{solution}
Seja o evento $A_j$ de uma família escolhida ao acaso ter $j$ crianças e $B$ o evento de sortear a criança mais velha em uma família.
\begin{parts}
	\part A probabilidade que estamos interessados é dada por
    \begin{align*}
    	P(A_1\mid B)
    		= \dfrac{P(B\mid A_1)P(A_1)}{\sum_{j=1}^4 P(B\mid A_j)P(A_j)}
            = \dfrac{1\cdot p_1}{\sum_{j=1}^4 \frac{1}{j}\cdot p_j} = \text{0,24}.
    \end{align*}
    Note que para o caso da criança mais nova o problema é exatamente o mesmo.
    \part Analogamente, temos
        \begin{align*}
    	P(A_4\mid B)
    		= \dfrac{P(B\mid A_4)P(A_4)}{\sum_{j=1}^4 P(B\mid A_j)P(A_j)}
            = \dfrac{\frac{1}{4}\cdot p_4}{\sum_{j=1}^4 \frac{1}{j}\cdot p_j}
            = \text{0,18}.
    \end{align*}
\end{parts}
\end{solution}

% 3.38 % % % % % % % % % % % % % % % % % % % % %
\setcounter{question}{37}
\question{A urna $A$ tem 5 bolas brancas e 7 bolas
pretas. A urna $B$ em 3 bolas brancas e 12
bolas pretas. Jogamos uma moeda honesta; se der cara, retiramos uma bola da
urna $A$. Se der coroa, retiramos uma bola
da urna $B$. Suponha que uma bola branca
seja selecionada. Qual é a probabilidade
de que tenha dado coroa na moeda?
}
\begin{solution}
	Sejam os eventos $E$ retirar uma bola branca e $F$ selecionar a urna $B$. A probabilidade que estamos interessados é
    \[P(F\mid E) = \dfrac{P(E\mid F)P(F)}{P(E\mid F)P(F)+P(E\mid F^c)P(F^c)}
    	= \dfrac{\frac{3}{15}\frac{1}{2}}
        	{\frac{3}{15}\frac{1}{2}+\frac{5}{12}\frac{1}{2}}
		= \dfrac{12}{37} \approx \text{0,324}.\]
\end{solution}

% 3.45 % % % % % % % % % % % % % % % % % % % % %
\setcounter{question}{44}
\question{Suponha que tenhamos 10 moedas tais
que, se a $i$-ésima moeda for jogada, a probabilidade de ela dar cara é igual a $i/10, i = 1, 2, \dots,10$. Quando uma das moedas é selecionada aleatoriamente e jogada, ela dá
cara. Qual é a probabilidade condicional
de que tenha sido a quinta moeda?
}
\begin{solution}
	Sejam os eventos $A_i$ selecionar a $i$-ésima moeda e $B$ sair cara. Estamos interessados na probabilidade
    \[P(A_5\mid B) = \dfrac{P(B\mid A_5)P(A_5)}{\sum_{i=1}^{10}P(B\mid A_i)P(A_i)}
    	= \dfrac{\frac{5}{10}\frac{1}{10}}{\sum_{i=1}^{10}\frac{i}{10}\frac{1}{10}}
        = \dfrac{5}{\frac{10\cdot11}{2}} = \frac{1}{11}.\]
\end{solution}

% 3.46 % % % % % % % % % % % % % % % % % % % % %
\question{Em um ano qualquer, um homem usará
o seu seguro de carro com probabilidade $p_m$, e uma mulher terá probabilidade
$p_f$ de usar o seu seguro de carro, onde $p_f \neq p_m$. A fração de segurados homens é
igual a $\alpha$, $0 < \alpha < 1$. Um segurado é escolhido aleatoriamente. Se $A_i$ representar o evento em que este segurado fará
uso de seu seguro em um ano, mostre que $P(A_2\mid A_1) > P(A_1)$. Dê uma explicação intuitiva do porquê da desigualdade anterior ser verdade.
}
\begin{solution}
	A explicação intuitiva é que uma vez que o segurado tenha usado o seguro no ano anterior, é mais provável que o mesmo faça parte de um grupo de pessoas que use mais frequentemente o seguro.
    \begin{proof}
    	Seja $H$ o evento referente à um segurado selecionado ao acaso ser homem. Queremos provar que $P(A_2\mid A_1) > P(A_1)$. Usando o teorema de Bayes, \textit{i.e}, $P(A) = P(A\mid H)P(H)+P(A\mid H^c)P(H^c)$ temos que
        \begin{align*}
        	P(A_2\mid A_1) > P(A_1) 
            	&\Leftrightarrow P(A_1\,A_2) > P(A_1)^2 \\
                &\Leftrightarrow p_m^2\,\alpha+p_f^2\,(1-\alpha) > 
                	(p_m\,\alpha+p_f\,(1-\alpha))^2\\
				&\Leftrightarrow \alpha\,(1-\alpha)\,(p_m-p_f)^2 > 0,
        \end{align*}
        o que é sempre verdade, uma vez que $0<\alpha<1$ e $p_m\neq p_f$.
    \end{proof}
\end{solution}

% 3.50 % % % % % % % % % % % % % % % % % % % % %
\setcounter{question}{49}
\question{Suponha que uma companhia de seguros
classifique as pessoas em uma de três
classes: risco baixo,risco médio e risco elevado.
Os registros da companhia indicam
que as probabilidades de que pessoas com
riscos baixo, médio e elevado estejam envolvidas
em acidentes ao longo do período
de um ano são de 0,05, 0,15 e 0,30. Se 20\%
da população são classificados como de
risco baixo, 50\% de risco médio e 30\% de
risco elevado, que proporção de pessoas
sofre acidentes ao longo de um ano? Se o
segurado $A$ não sofreu acidentes em 1997,
qual é a probabilidade de que ele ou ela
seja uma pessoa de risco baixo ou médio?
}
\begin{solution}
	Sejam $L$, $M$, $H$ os eventos correspondentes a uma pessoa selecionada ao acaso pertencer ao grupo de baixo, médio e alto risco, respectivamente; seja $E$ o evento de uma pessoa selecionada ao acaso sofrer acidentes ao longo do ano. Temos que
    \begin{align*}
    	P(E) &= P(E\mid L)P(L)+P(E\mid M)P(M)+P(E\mid H)P(H)\\
    		&= \text{0,2}\cdot\text{0,05}+\text{0,5}\cdot\text{0,15}+
            	\text{0,3}\cdot\text{0,3} = \text{0,175}.
    \end{align*}
    Logo, 17,5\% das pessoas sofrem acidentes ao longo do ano.\\
    Em seguida, estamos interessados na probabilidade
    	\begin{align*}
    		P(L\cup M\mid E^c) &= P(L\mid E^c)+P(M\mid E^c)\\
            	&= \dfrac{P(E^c\mid L)P(L)+P(E^c\mid M)P(M)}{P(E^c)}\\
                &= \dfrac{(1-\text{0,05})\,\text{0,2}+(1-\text{0,15})\,\text{0,5}}
                	{1-\text{0,175}} \approx \text{0,745}.
    	\end{align*}
    Logo, uma pessoa que não sofreu acidentes em um determinado ano tem probabilidade 74,5\% de pertencer aos grupos de baixo ou médio risco.
\end{solution}

% 3.63 % % % % % % % % % % % % % % % % % % % % %
\setcounter{question}{62}
\question{$A$ e $B$ estão envolvidos em um duelo. As
regras do duelo rezam que eles devem sacar suas armas e atirar um no outro simultaneamente. Se um ou ambos são atingidos, o duelo é encerrado; se ambos os tiros
erram os alvos, então repete-se o processo.
Suponha que os resultados dos tiros sejam
independentes e que cada tiro de $A$ atinja
$B$ com probabilidade $p_A$ e que cada tiro de
$B$ atinja $A$ com probabilidade $p_B$. Qual é
\begin{parts}
	\part a probabilidade de que $A$ não seja
atingido?
	\part a probabilidade de que ambos os duelistas sejam atingidos?
    \part a probabilidade de que o duelo termine após a $n$-ésima rodada de tiros?
    \part a probabilidade condicional de que o
duelo termine após a $n$-ésima rodada
de tiros dado que $A$ não tenha sido
atingido.
	\part a probabilidade condicional de que o
duelo termine após a $n$-ésima rodada
de tiros dado que ambos os duelistas
tenham sido atingidos?
\end{parts}
}
\begin{solution} Seja $E_n$ o evento correspondente à ambos duelistas estarem vivos na $n$-ésima rodada e sejam $A_n$ e $B_n$ os eventos de na $n$-ésima rodada o participante $A$ e $B$ morrer, respectivamente.
	\begin{parts}
		\part Primeiramente, vamos calcular a probabilidade de $B$ morrer e $A$ sobreviver na $n$-ésima rodada, dado que sobreviveram até a $(n-1)$-ésima rodada:
        	\[P(A_{n}^c\,B_{n}\mid E_{n-1}) = p_A(1-p_B).\]
            A probabilidade de sobreviverem até a $n$-ésima rodada é simplesmente
            \[P(E_n) = (1-p_A)^n(1-p_B)^n.\]
            Enfim, a probabilidade que estamos interessados é
            \begin{align*}
            	P\left( \bigcup_{n=1}^\infty A_n^c\,B_n \right)
            		&= \sum_{n=1}^\infty P(A_n^c\,B_n)\\
                	&= \sum_{n=1}^\infty P(A_n^c\,B_n\mid E_{n-1}) P(E_{n-1})\\
                	&= \sum_{n=1}^\infty p_A(1-p_A)^{n-1}(1-p_B)^n\\
                    &= p_A(1-p_B) \sum_{n=0}^\infty [(1-p_A)(1-p_B)]^n\\
                    &= \dfrac{p_A(1-p_B)}{1-(1-p_A)(1-p_B)},
            \end{align*}
        onde usamos o fato de que os eventos $(A_n^c\,B_n)_n$ são mutuamente exclusivos e usamos a fórmula da soma de progressão geométrica.
        \part Analogamente, temos
            \begin{align*}
            	P\left( \bigcup_{n=1}^\infty A_n\,B_n \right)
            		&= \sum_{n=1}^\infty P(A_n\,B_n)\\
                	&= \sum_{n=1}^\infty P(A_n\,B_n\mid E_{n-1}) P(E_{n-1})\\
                	&= \sum_{n=1}^\infty p_A\,p_B\,(1-p_A)^{n-1}(1-p_B)^{n-1}\\
                    &= p_A\,p_B \sum_{n=0}^\infty [(1-p_A)(1-p_B)]^n\\
                    &= \dfrac{p_A\,p_B}{1-(1-p_A)(1-p_B)},
            \end{align*}
        \part Isso é equivalente à eles sobreviverem até a $n$-ésima rodada, ou seja,
        	\[P(E_n) = (1-p_A)^n(1-p_B)^n.\]
        \part Note que a probabilidade de $A$ não ser atingido dado que o duelo termine após a $n$-ésima rodada não muda, pois nesse caso o duelo a partir da rodada $n+1$ é idêntico à um duelo descondicionado que acabou de começar. Logo, esses eventos são independentes. Da mesma forma, a probabilidade de que o duelo termine após a $n$-ésima rodada dado que $A$ não tenha sido atingido é simplesmente $P(E_n)$.
	\part Análogo ao item anterior.
	\end{parts}
\end{solution}

% 3.70 % % % % % % % % % % % % % % % % % % % % %
\setcounter{question}{69}
\question{Há uma chance de 50-50 de que a rainha
seja portadora do gene da hemofilia. Se
ela é portadora, então cada um dos príncipes tem uma chance de 50-50 de ter
hemofilia. Se a rainha tiver tido três príncipes sadios, qual é a probabilidade de
ela ser portadora? Se houver um quarto
príncipe, qual é a probabilidade de que
ele venha a desenvolver a hemofilia?
}
\begin{solution}
	Sejam os eventos $H$ a rainha ser portadora do gene da hemofilia e $E$ ela ter tido 3 príncipes sadios, então
    \[P(H\mid E) = \dfrac{P(E\mid H) P(H)}{P(E\mid H)P(H)+P(E\mid H^c)P(H^c)}
    	= \dfrac{(\frac{1}{2})^3\cdot\frac{1}{2}}
        	{(\frac{1}{2})^3\cdot\frac{1}{2}+1\cdot\frac{1}{2}}
		= \dfrac{1}{9}.\]
	Assim, a probabilidade de ela ter um quarto príncipe com hemofilia é $\frac{1}{2}\cdot\frac{1}{9} = \frac{1}{18}$.
\end{solution}

% 3.75 % % % % % % % % % % % % % % % % % % % % %
\setcounter{question}{74}
\question{Em certa aldeia, é tradicional que o filho
mais velho e sua esposa cuidem dos pais
dele em sua velhice. Nos últimos anos, no
entanto, as mulheres desta aldeia, não
querendo assumir responsabilidades, têm
preferido não se casar com os filhos mais
velhos de uma família.
\begin{parts}
	\part Se cada família da aldeia tem duas
crianças, qual é a proporção de filhos
mais velhos?
    \part Se cada família da aldeia tem três
crianças, qual é a proporção de filhos
mais velhos?
\end{parts}
Suponha que cada criança tenha a mesma
probabilidade de ser menino ou menina.
}
\begin{solution}
\begin{parts}
	\part Sejam os eventos $H$ e $V$ uma criança selecionada ao acaso ser menino e ser o mais velho, respectivamente. Como toda família tem duas crianças, esses eventos são equivalentes à selecionar uma família ao acaso e então selecionar uma criança da família ao acaso. Definimos o evento $E$ como o da família selecionada ao acaso ter pelo menos um menino. Assim, como $H\,V \subset H\subset E$, então
    \[P(H\,V) = P(H\,V\mid E) P(E) 
    	= \dfrac{1}{2}\left(1-\dfrac{1}{2^2}\right) = \dfrac{3}{8}.\]
    A probabilidade que nos interessa é
    \[P(H\,V\mid H) = \dfrac{P(H\,V)}{P(H)} = \dfrac{\frac{3}{8}}{\frac{1}{2}}
    	= \dfrac{3}{4}.\]
    \part Analogamente ao item anterior,
    	\[P(H\,V) = \dfrac{1}{3}\left(1-\dfrac{1}{2^3}\right) = \dfrac{7}{24}.\]
    Logo,
    	\[P(H\,V\mid H) = \dfrac{\frac{7}{24}}{\frac{1}{2}} = \dfrac{7}{12}.\]
\end{parts}
\end{solution}

% 3.78 % % % % % % % % % % % % % % % % % % % % %
\setcounter{question}{77}
\question{\textit{O enunciado dessa questão está errado na versão brasileira. Em inglês é:}\\
$A$ and $B$ play a series of games. Each game is independently won by $A$ with probability $p$ and by $B$ with probability $1-p$. They stop when the total
number of wins of one of the players is two greater
than that of the other player. The player with the
greater number of total wins is declared the winner
of the series.
\begin{parts}
	\part Find the probability that a total of 4 games are played.
    \part Find the probability that $A$ is the winner of the series.
\end{parts}
}
\begin{solution}
\begin{parts}
	\part As sequências de vitórias que terminam com quatro jogos são $ABBB$, $BABB$, $BAAA$ e $ABAA$. Logo, a probabilidade desse evento acontecer é dada por
    \[{2p(1-p)^3+2p^3(1-p)}.\]
	\part Seja $p_i$ a probabilidade do jogador $A$ ser o ganhador dado que a diferença entre os pontos de $A$ e $B$ é igual à $i$. Podemos escrever o seguinte sistema de equações:
    \begin{align*}
    	p_2 &= 1,\\
        p_1 &= p\,p_2 + (1-p)\,p_0,\\
        p_0 &= p\,p_1 + (1-p)\,p_{-1},\\
        p_{-1} &= p\,p_0 + (1-p)\,p_{-2},\\
        p_{-2} &= 0.
    \end{align*}
    Resolvendo para $p_0$, obtemos que $p_0 = \dfrac{p^2}{1-2p(1-p)}$.
\end{parts}
\end{solution}

% 3.79 % % % % % % % % % % % % % % % % % % % % %
%\setcounter{question}{69}
\question{Em jogadas sucessivas de um par de dados
honestos, qual é a probabilidade de que
saiam 2 setes antes de 6 números pares?
}
\begin{solution}
	A probabilidade da soma de dois dados ser sete é $\frac{6}{6^2}=\frac{1}{6}$. Por outro lado, a probabilidade da soma de dois dados ser par é $\frac{1}{4}+\frac{1}{4} = \frac{1}{2}$ (par e par ou ímpar e ímpar). Podemos ignorar todos os resultados que estejam fora desses casos. Assim, seja $p$ a probabilidade de sair um número par, dado que saiu sete ou número par, então \[p = \dfrac{\frac{1}{2}}{\frac{1}{6}+\frac{1}{2}} = \dfrac{3}{4}.\]
    Note que, dessa forma, esse problema é equivalente ao Exemplo 4j: O problema dos pontos, com $n=6$ e $m=2$. Assim, a probabilidade de sair 2 setes antes de sair 6 números pares é o complemento de sair 6 números pares antes de 2 setes, ou seja,
    \[1-\sum_{k=6}^{7} \binom{7}{k} \textstyle(\frac{3}{4})^k(\frac{1}{4})^{7-k}
    	\approx \text{0,555}.\]
\end{solution}

% 3.81 % % % % % % % % % % % % % % % % % % % % %
\setcounter{question}{80}
\question{Uma investidora tem participação em
uma ação cujo valor atual é igual a 25.
Ela decidiu que deve vender sua ação se
ela chegar a 10 ou 40. Se cada mudança
de preço de 1unidade para cima ou
para baixo ocorrer com probabilidades
de 0,55 e 0,45, respectivamente, e se as
variações sucessivas são independentes,
qual é a probabilidade de que a investidora tenha lucro?
}
\begin{solution}
	Esse problema é equivalente ao Exemplo 4l: O Problema da ruína do jogador, com $N=40-10=30$ e $i=25-10=15$, $p=\text{0,55}$. Logo, a probabilidade que a empresa tenha lucro é dado por
    \[\dfrac{1-(\text{0,45}/\text{0,55})^{15}}{1-(\text{0,45}/\text{0,55})^{30}}
    	\approx \text{0,953}.\]
\end{solution}

\end{questions}
