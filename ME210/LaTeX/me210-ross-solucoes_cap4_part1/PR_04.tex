
\begin{questions}

% 4.4  % % % % % % % % % % % % % % % % % % % % %
\setcounter{question}{3}
\question{Cinco homens e 5 mulheres são classificados de acordo com suas notas em uma prova. Suponha que não existam notas
iguais e que todas as $10!$ classificações possíveis sejam igualmente prováveis. Faça $X$ representar a melhor classificação
obtida por uma mulher (por exemplo, $X=1$ se a pessoa mais bem classificada for uma mulher). Determine $P\{X = i\},i = 1,2,3,\dots,8,9,10$.
}
\begin{solution}
	Se $i > 6$, então $P\{X = i\} = 0$, pois temos 5 homens participando. Se $i < 6$, então teremos 4 mulheres ocupando qualquer posição entre as posições abaixo a $i$-ésima, isto é, 4 mulheres entre $10-i$ posições. Logo,
	\[P\{X = i\} = \dfrac{\binom{10-i}{4}}{\binom{10}{5}}, \quad
    	i\in\{1,2\dots,6\}.\]
\end{solution}

% 4.5  % % % % % % % % % % % % % % % % % % % % %
%\setcounter{question}{3}
\question{Suponha que $X$ represente a diferença
entre o número de caras e coroas obtido
quando uma moeda é jogada $n$ vezes.
Quais são os possíveis valores de $X$?
}
\begin{solution}
	Se $n$ for ímpar, $X \in \{-n,-n+2,\dots,-1,1,\dots,n-2,n\}$.
    Se $n$ for par, $X \in \{-n,-n+2,\dots,-2,0,2,\dots,n-2,n\}$.
\end{solution}

% 4.13 % % % % % % % % % % % % % % % % % % % % %
\setcounter{question}{12}
\question{Um vendedor agendou duas visitas para
vender enciclopédias. Sua primeira visita resultará em venda com probabilidade 0,3, e sua segunda visita resultará em
venda com probabilidade de 0,6, sendo
ambas as probabilidades de venda independentes. Qualquer venda realizada tem
a mesma probabilidade de ser do modelo
luxo,que custa R\$ 1000,00, ou do modelo
padrão, que custa R\$ 500,00. Determine
a função de probabilidade de $X$, o valor
total das vendas em reais.
}
\begin{solution}
	Para encontrar a probabilidade de $X$ ser igual a uma certa quantia, basta listar todas as formas de ganhar essa quantia e em seguida somar essas probabilidades.
	\begin{align*}
        p_X(0) &= (\text{0,7})(\text{0,4}) =\text{0,28},\\
    	p_X(500) &= (\text{0,7})(\text{0,6}\cdot\text{0,5})
            +(\text{0,3}\cdot\text{0,5})(\text{0,4})
        	=\text{0,27},\\
    	p_X(1000) &=(\text{0,3}\cdot\text{0,5})(\text{0,6}\cdot\text{0,5})
        	+(\text{0,7})(\text{0,6}\cdot\text{0,5})
            +(\text{0,3}\cdot\text{0,5})(\text{0,4})
        	=\text{0,315},\\
    	p_X(1500) &=2(\text{0,3}\cdot\text{0,5})(\text{0,6}\cdot\text{0,5})
        	=\text{0,09},\\
		p_X(2000) &=(\text{0,3}\cdot\text{0,5})(\text{0,6}\cdot\text{0,5})
        	=\text{0,045}.\\
	\end{align*}
    \vspace{-10mm}
\end{solution}

% 4.17 % % % % % % % % % % % % % % % % % % % % %
\setcounter{question}{16}
\question{Suponha que a função distribuição de $X$
seja dada por $F(b)$ (vide livro).
\begin{parts}
	\part Determine $P\{X= i\},i = 1,2,3$.
    \part Determine $P\{\frac{1}{2} < X <\frac{3}{2}\}$.
\end{parts}
}
\begin{solution}
\begin{parts}
	\part Sabemos que \[P\{X= i\} = F(i) - \lim_{b\to i^-}F(b).\]
    Logo,
    \begin{align*}
    	P\{X=1\} &= \frac{1}{2} - \lim_{b\to 1^-}\frac{b}{4} = \frac{1}{4},\\
        P\{X=2\} &= \frac{11}{12} - \lim_{b\to 2^-}
        	\left(\frac{1}{2}+\frac{b-1}{4} \right) = \frac{1}{6},\\
        P\{X=3\} &= 1 - \frac{11}{12} = \frac{1}{12}.\\
    \end{align*}

    \part
    \begin{align*}
    	P\{\tfrac{1}{2} < X <\tfrac{3}{2}\}
        	&= P\{X <\tfrac{3}{2}\}-P\{X \le\tfrac{1}{2}\}\\
        	&= \lim_{b\to \frac{3}{2}^-}F(b) - F(\tfrac{1}{2})\\
            &= \frac{1}{2} + \frac{\frac{3}{2}-1}{4} - \frac{\frac{1}{2}}{4}\\
            &= \frac{1}{2}.
    \end{align*}
\end{parts}
\end{solution}

% 4.21 % % % % % % % % % % % % % % % % % % % % %
\setcounter{question}{20}
\question{Quatro ônibus levando 148 estudantes da
mesma escola chegam a um estádio de futebol. Os ônibus levam, respectivamente,
40, 33, 25 e 50 estudantes. Um dos estudantes é selecionado aleatoriamente. Suponha
que $X$ represente o número de estudantes
que estavam no ônibus que levava o estudante selecionado. Um dos 4 motoristas
dos ônibus também é selecionado aleatoriamente. Seja $Y$ o número de estudantes
no ônibus do motorista selecionado
\begin{parts}
	\part Qual valor esperado você pensa ser
maior, $E[X]$ ou $E[Y]$?
    \part Calcule $E[X]$ e $E[Y]$.
\end{parts}
}
\begin{solution}
\begin{parts}
	\part $E[X]$ é maior que $E[Y]$, pois a probabilidade que o estudante selecionado venha de um ônibus com mais estudantes é maior, enquanto que para os motoristas é equiprovável.

    \part
    \begin{align*}
    	E[Y] &= \frac{1}{4}\cdot40+\frac{1}{4}\cdot33
    			+\frac{1}{4}\cdot25+\frac{1}{4}\cdot50 = 37,\\
		E[X] &= \frac{40}{148}\cdot40+\frac{33}{148}\cdot33
    			+\frac{25}{148}\cdot25+\frac{50}{148}\cdot50 \approx 39.
    \end{align*}
\end{parts}
\end{solution}

% 4.22 % % % % % % % % % % % % % % % % % % % % %
%\setcounter{question}{20}
\question{Suponha que dois times joguem uma série
de partidas que termina quando um deles
tiver ganhado $i$ partidas. Suponha que cada
partida jogada seja, independentemente,
vencida pelo time $A$ com probabilidade $p$.
Determine o número esperado de partidas
jogadas quando (a) $i = 2$ e (b) $i = 3$.
Também, mostre em ambos os casos que este
número é maximizado quando $p = \frac{1}{2}$.
}
\begin{solution}
Seja $N$ a variável aleatória que representa o número de jogos. Quando $N=n$ e $A$ venceu a partida, então temos $n-1$ jogos nos quais $A$ venceu $i-1$ vezes e $A$ vence o último jogo, ou então se o outro time vencer a partida, então temos $i-1$ vitórias desse time nos $n-1$ jogos e ele vence o $n$-ésimo jogo. Logo, se $n\in\{i,i+1,\dots,2i-1\}$,
\begin{align*}
	P\{N=n\}
    	&= \binom{n-1}{i-1}p^{i-1}(1-p)^{n-i}\cdot p
    		+ \binom{n-1}{i-1}p^{n-i}(1-p)^{i-1}\cdot (1-p)\\
        &= \binom{n-1}{i-1} \left(p^i(1-p)^{n-i}+p^{n-i}(1-p)^{i} \right).
\end{align*}

\begin{parts}
	\part Se $i=2$,
    \begin{align*}
    	E[N]
        	&= \sum_{n=2}^3 n \binom{n-1}{1} 
        		\left(p(1-p)^{n-1}+p^{n-1}(1-p) \right)\\
        	&= 2\,(1+p-p^2).
    \end{align*}

    \part Se $i=3$,
    \begin{align*}
    	E[N]
        	&= \sum_{n=3}^5 n \binom{n-1}{2}
            	\left(p^2(1-p)^{n-2}+p^{n-2}(1-p)^{2}\right)\\
        	&= 3\, (1 + p + p^2 - 4 p^3 + 2 p^4).
    \end{align*}
\end{parts}
    \hrulefill \\
    Em ambos casos, $\frac{\mathrm{d}}{\mathrm{d}p}E[N]$ é maior que 0 se $p\in(0,\frac{1}{2})$ e é menor que 0 se $p\in(\frac{1}{2},1)$. Logo, $p=\frac{1}{2}$ é o ponto de máximo global.
\end{solution}

% 4.22 % % % % % % % % % % % % % % % % % % % % %
%\setcounter{question}{20}
\question{Você tem R\$1000,00, e certa mercadoria
é vendida atualmente por R\$2,00 o quilo.
Suponha que uma semana depois a mercadoria passe a ser vendida por R\$1,00
ou R\$4,00 o quilo, com essas duas possibilidades sendo igualmente prováveis.
\begin{parts}
	\part Se o seu objetivo é maximizar a quantidade esperada de dinheiro que você possuirá no final da semana, que estratégia você deve empregar?
    \part Se o seu objetivo é maximizar a quantidade esperada de mercadoria que
você possuirá no final da semana, que
estratégia você deve empregar?
\end{parts}
}
\begin{solution}
\begin{parts}
	\part Suponha que compremos $b \in (0,500)$ quilos da mercadoria. Seja a variável aleatória $X$ o dinheiro que teremos ao final da semana se vendermos toda mercadoria comprada. Então,
    \begin{align*}
    	E[X] &= (1000-2b) + b\frac{1}{2} + 4b\frac{1}{2}\\
        	&= 1000+\frac{b}{2}.
    \end{align*}
    Logo, devemos comprar o máximo possível da mercadoria, isto é, para maximizar o valor esperado do dinheiro ao final da semana devemos comprar $b = 500$ quilos da mercadoria.

    \part Seja a variável aleatória $Y$ os quilos de mercadoria que teremos ao final da semana se comprarmos toda mercadoria com o dinheiro que sobrar. Então,
    \begin{align*}
    	E[Y] &= b + \frac{1000-2b}{1}\frac{1}{2} + \frac{1000-2b}{4}\frac{1}{2}\\
        	&= 625-\frac{b}{4}.
    \end{align*}
    Logo, devemos comprar o mínimo possível da mercadoria no início da semana, isto é, para maximizar a quantidade de mercadoria ao final da semana devemos comprar $b = 0$ quilos da mercadoria.    
\end{parts}
\end{solution}

% 4.27 % % % % % % % % % % % % % % % % % % % % %
\setcounter{question}{26}
\question{Uma companhia de seguros vende uma
apólice dizendo que uma quantidade
$A$ de dinheiro deve ser paga se algum
evento $E$ ocorrer em um ano. Se a companhia
estima que $E$ tem probabilidade $p$
de ocorrer em um ano, que preço deve o
cliente pagar pela apólice se o lucro 
esperado pela companhia é de 10\% de $A$?
}
\begin{solution}
	Seja $X$ a variável aleatória que representa o lucro da empresa para uma dada apólice e $D$ o preço da apólice, então
    \begin{align*}
    	E[X] = (D-A)\cdot p + D\cdot(1-p).
    \end{align*}
    Como queremos $E[X] = \frac{A}{10}$, então isolando $D$ obtemos que $D = \frac{A}{10}+A\,p$.
\end{solution}

% 4.28 % % % % % % % % % % % % % % % % % % % % %
%\setcounter{question}{26}
\question{Uma amostra de 3 itens é selecionada
aleatoriamente de uma caixa contendo 20
itens, dos quais 4 são defeituosos.
Determine o número esperado de itens
defeituosos na amostra.
}
\begin{solution}
	Seja $X_i$ a variável aleatória que vale 1 se o $i$-ésimo item retirado for defeituoso e vale 0 caso contrário. Queremos saber o valor de
    \begin{align*}
    	E[X_1+X_2+X_3] &= E[X_1]+E[X_2]+E[X_3]\\
        	&= 3\,E[X_1] = 3\frac{4}{20} = \frac{3}{5}.
    \end{align*}
\end{solution}

% 4.29 % % % % % % % % % % % % % % % % % % % % %
%\setcounter{question}{26}
\question{Existem duas causas possíveis para a quebra de certa máquina. Verificar a primeira
possibilidade custa $C_1$ reais, e, se aquela tiver sido de fato a causa da quebra, o problema pode ser reparado ao custo de $R_1$
reais. Similarmente, existem os custos $C_2$
e $R_2$ associados a segunda possibilidade.
suponha que $p$ e $1-p$ representem, respectivamente,
as probabilidades de que a
quebra seja causada pela primeira e pela
segunda possibilidades. Em quais condições
de $p$, $C_i$ e $R_i$, $i = 1,2$, devemos verificar inicialmente a primeira causa possível de defeito e depois a segunda, em vez de
inverter a ordem de verificação, de forma
a minimizarmos o custo envolvido na manutenção da máquina?\\
\textit{Nota}: Se a primeira verificação for negativa, devemos ainda assim verificar a segunda possibilidade.
}
\begin{solution}
	Sejam as variáveis aleatórias $X$ e $Y$ os custo de manutenção da máquina não invertendo e invertendo a ordem das verificações, respectivamente. Então,
    \begin{align*}
    	E[X] &= (C_1+R_1)\,p + (C_1+C_2+R_2)\,(1-p),\\
        E[Y] &= (C_2+R_2)\,(1-p) + (C_1+C_2+R_1)\,p.
    \end{align*}
    Vale a pena não inverter quando $E[Y]>E[X]$, ou seja,
    \begin{align*}
    	E[Y]-E[X] > 0 &\Leftrightarrow C_2\,p - C_1\,(1-p) > 0\\
        	&\Leftrightarrow \frac{C_1}{C_2} < \frac{p}{1-p}.
    \end{align*}
\end{solution}

% 4.32 % % % % % % % % % % % % % % % % % % % % %
\setcounter{question}{31}
\question{Cem pessoas terão seu sangue examinado para determinar se possuem ou não
determinada doença. Entretanto, em vez
de testar cada indivíduo separadamente,
decidiu-se primeiro colocar as pessoas em
grupos de 10. As amostras de sangue das
10 pessoas de cada grupo serão analisadas
em conjunto. Se o teste der negativo, apenas um teste será suficiente para as 10 pessoas. Por outro lado, se o teste der positivo, cada uma das demais pessoas também
será examinada e, no total, 11 testes serão
feitos no grupo em questão. Suponha que
a probabilidade de se ter a doença seja de
0,1 para qualquer pessoa, de forma independente, e calcule o número esperado de
testes necessários para cada grupo (observe que supomos que o teste conjunto
dará positivo se pelo menos uma pessoa
no conjunto tiver a doença).
}
\begin{solution}
	Seja $p=\text{0,1}$ de uma pessoa ter a doença e $X_i$ a variável aleatória que representa o número de testes feitos no $i$-ésimo grupo. Então, o número esperado de testes em cada grupo é dado por
    \begin{align*}
    	 E[X_i]
            &= 1\,(1-p)^{10}+11\,(1-(1-p)^{10})\\
            &\approx \text{7,513} < 10. 
    \end{align*}
    De fato, essa estratégia vale a pena quando $p < 1-10^{-\frac{1}{10}} \approx \text{0,2}$. Por outro lado, para $p=\text{0,1}$ valeria mais a pena dividir as pessoas em grupos de 4, ao invés de 10. Dessa forma, $E[X_i]\approx\text{2,376}$ e $\frac{\text{2,376}}{4} < \frac{\text{7,513}}{10}$.
\end{solution}

% 4.40 % % % % % % % % % % % % % % % % % % % % %
\setcounter{question}{39}
\question{Em um teste de múltipla escolha com 3
respostas possíveis para cada uma das 5
questões, qual é a probabilidade de que
um estudante acerte 4 questões ou mais
apenas chutando?
}
\begin{solution}
	Seja $X$ a variável aleatória correspondente ao número de acertos. Sabemos que $X$ segue uma distribuição binomial de parâmetros $p=\tfrac{1}{3}$ e $n=5$. Queremos saber
    \begin{align*}
    	P\{X\ge4\} &= P\{X=4\}+P\{X=5\} \\
        	&= \binom{5}{4}(\tfrac{1}{3})^4(\tfrac{2}{3})
            	+ \binom{5}{5}(\tfrac{1}{3})^5 \\
			&\approx \text{0,045}. 
    \end{align*}
\end{solution}

% 4.43 % % % % % % % % % % % % % % % % % % % % %
\setcounter{question}{42}
\question{Um canal de comunicações transmite
os algarismos 0 e 1. Entretanto, devido
a interferência estática, um algarismo
transmitido tem probabilidade 0,2 de ser
incorretamente recebido. Suponha que
queiramos transmitir uma mensagem
importante formada por um único algarismo
binário. Para reduzir as chances de erro,
transmitimos 00000 em vez de 0 e 11111
em vez de 1. Se o receptor da mensagem
usa um decodificador de ``maioria'' qual é
a probabilidade de que a mensagem não
esteja correta quando decodificada?
}
\begin{solution}
	Seja $X$ a variável aleatória correspondente ao número de algarismos transmitidos errados. Note que $X$ segue uma distribuição binomial de parâmetros $p=\text{0,2}$ e $n=5$. A mensagem é decodificada incorretamente quando $X\ge 3$. Assim,
    \begin{align*}
    	P\{X\ge 3\} &= P\{X=3\}+P\{X=4\}+P\{X=5\} \\
        	&= \binom{5}{3}(\tfrac{1}{5})^3(\tfrac{4}{5})^2
            	+ \binom{5}{4}(\tfrac{1}{5})^4(\tfrac{4}{5})
                + \binom{5}{5}(\tfrac{1}{5})^5\\
			&\approx \text{0,058}. 
    \end{align*}
\end{solution}

% 4.45 % % % % % % % % % % % % % % % % % % % % %
\setcounter{question}{44}
\question{Um estudante se prepara para fazer uma
importante prova oral e está preocupado
com a possibilidade de estar em um dia
``bom'' ou em um dia ``ruim''. Ele pensa
que se estiver em um dia bom, então cada
um de seus examinadores irá aprová-lo,
independentemente uns dos outros, com
probabilidade 0,8. Por outro lado, se
estiver em um dia ruim, essa probabilidade
cairá para 0,4. Suponha que o estudante
seja aprovado caso a maioria dos examinadores
o aprove. Se o estudante pensa
que tem duas vezes mais chance de estar
em um dia ruim do que em um dia bom,
é melhor que ele faça a prova com 3 ou 5
examinadores?
}
\begin{solution}
	Seja $B$ o evento do estudante estar em um dia bom. Sabemos que $P(B^c) = 2\,P(B)$, mas como $P(B)+P(B^c) = 1$, então $P(B)=\frac{1}{3}$ e $P(B^c)=\frac{2}{3}$. Seja $X_i$ a variável aleatória correspondente ao número de professores que aprovaram ele, onde $i$ é o número de professores que o avaliam. As probabilidades do estudante ser aprovado com $i=3$ ou $i=5$ são
    \begin{align*}
    	P\{X_3\ge 2\} 
        	&= P\{X_3\ge 2\mid B\}P(B)+P\{X_3\ge 2\mid B^c\}P(B^c)\\
        	&= \left(\binom{3}{2}(\text{0,8})^2(\text{0,2})
            	+ \binom{3}{3}(\text{0,8})^3\right)\cdot\frac{1}{3}
                + \left(\binom{3}{2}(\text{0,4})^2(\text{0,6})
            	+ \binom{3}{3}(\text{0,4})^3\right)\cdot\frac{2}{3}\\
			&\approx \text{0,533}.\\[3mm]
		P\{X_5\ge 3\} 
         	&= P\{X_5\ge 3\mid B\}P(B)+P\{X_5\ge 3\mid B^c\}P(B^c)\\
        	&= \left(\binom{5}{3}(\text{0,8})^3(\text{0,2})^2
            	+ \binom{5}{4}(\text{0,8})^4(\text{0,2})
                + \binom{5}{5}(\text{0,8})^5\right)\cdot\frac{1}{3}\\
			&\quad + \left(\binom{5}{3}(\text{0,4})^3(\text{0,6})^2
            	+ \binom{5}{4}(\text{0,4})^4(\text{0,6})
                + \binom{5}{5}(\text{0,4})^5\right)\cdot\frac{2}{3}\\
			&\approx \text{0,526}.
    \end{align*}
\end{solution}

% 4.48 % % % % % % % % % % % % % % % % % % % % %
\setcounter{question}{47}
\question{Sabe-se que os disquetes produzidos por
certa companhia têm probabilidade de defeito
igual a 0,01, independentemente uns
do outros. A companhia vende os disquetes
em embalagens com 10 e oferece uma
garantia de devolução se mais que 1 disquete
em uma embalagem com 10 disquetes apresentar defeito.
Se alguém compra 3 embalagens, qual é a probabilidade de que
ele ou ela devolva exatamente 1 delas?
}
\begin{solution}
	Seja $X$ a variável aleatória correspondente ao número de disquetes defeituosos em uma embalagem, então $X$ segue uma distribuição binomial de parâmetros $p=\text{0,01}$ e $n=10$. A empresa troca a embalagem se $X>1$. Assim, a probabilidade da empresa trocar uma dada embalagem é
    \begin{align*}
    	P\{X > 1\} &= 1 - P\{X\le 1\} \\
        	&= 1 - \binom{10}{0}(\text{0,99})^{10}
            	- \binom{10}{1}(\text{0,01})(\text{0,99})^9 \\
			&\approx \text{0,0043}.
    \end{align*}
    Dessa forma, a probabilidade que exatamente uma embalagem seja trocada é
    \begin{align*}
		\binom{3}{1}(P\{X > 1\})(1-P\{X > 1\})^2 \approx \text{0,0127}.
    \end{align*}
\end{solution}

% 4.49 % % % % % % % % % % % % % % % % % % % % %
%\setcounter{question}{42}
\question{A moeda 1 dá cara com probabilidade
0,4; a moeda 2 tem probabilidade 0,7 de
dar cara. Uma dessas moedas é escolhida
aleatoriamente e jogada 10 vezes.
\begin{parts}
	\part Qual é a probabilidade de que a
moeda dê cara em exatamente 7 das
10 jogadas?
	\part Dado que a primeira dessas 10 jogadas dê cara, qual é a probabilidade
condicional de que exatamente 7 das
10 jogadas deem cara?
\end{parts}
}
\begin{solution}
	Seja $A$ o evento de selecionar a moeda 1. Seja $X_i$ a variável aleatória correspondente ao número de caras quando jogamos 10 vezes a moeda $i$, $i\in\{0,1\}$. Note que $X_i$ segue uma distribuição binomial de parâmetros $p=p_i$ ($p_1=\text{0,4}$ e $p_2=\text{0,7}$) e $n=10$. Definimos também a v.a. $X$, que representa o número de caras de uma das duas moedas selecionadas ao acaso.
\begin{parts}
	\part A probabilidade da moeda dar cara em 7 das 10 jogadas é dado por
    \begin{align*}
    	P\{X=7\}
        	&= P\{X=7\mid A\}P(A)+P\{X=7\mid A^c\}P(A^c) \\
            &= P\{X_1=7\}P(A)+P\{X_2=7\}P(A^c) \\
        	&= \binom{10}{7}(\text{0,4})^7(\text{0,6})^3\cdot\frac{1}{2}
            	+\binom{10}{7}(\text{0,7})^7(\text{0,3})^3\cdot\frac{1}{2}\\
			&\approx \text{0,155}.
    \end{align*}
    
    \part Seja o evento $C$ a primeira jogada dar cara, então estamos interessados na probabilidade
    \begin{align*}
    	P\{X=7\mid C\}
        	&= \frac{P(\{X=7\} \cap C)}{P(C)} \\
            &= \frac{P(\{X=7\} \cap C\mid A) P(A)
                + P(\{X=7\} \cap C\mid A^c) P(A^c)}
                {P(C\mid A)P(A)+P(C\mid A^c)P(A^c)} \\
			&= \frac{ \binom{9}{6}(\text{0,4})^6(\text{0,6})^3
            	\cdot(\text{0,4})\cdot\frac{1}{2}
            	+\binom{9}{6}(\text{0,7})^6(\text{0,3})^3
                \cdot(\text{0,7})\cdot\frac{1}{2}}
                {(\text{0,4})\cdot\frac{1}{2}+(\text{0,7})\cdot\frac{1}{2}} \\
			&\approx \text{0,197}.
    \end{align*}
\end{parts}
\end{solution}

\end{questions}
