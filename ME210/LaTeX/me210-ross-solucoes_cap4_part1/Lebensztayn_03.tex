\begin{questions}
% 1   % % % % % % % % % % % % % % % % % % % % %
\question{
Seja $X$ uma variável aleatória discreta com função de probabilidade dada por
\[p(x) = c\,x,\quad x=1,2,\dots,6.\]
Encontre
\begin{parts}
	\part o valor de $c$.
    
    \part a probabilidade de $X$ ser um número ímpar.
\end{parts}
}
\begin{solution}
\begin{parts}
	\part Sabemos que
    \begin{align*}
    	\sum_{x=1}^6 p(x) = 1 \Rightarrow \sum_{x=1}^6 c\,x = 1
        	\Rightarrow c = \dfrac{1}{\sum_{x=1}^6 x} = \dfrac{1}{21}.
    \end{align*}
    
    \part A probabilidade de $X$ ser ímpar é equivalente à probabilidade de $X\in\{1,3,5\}$, ou seja,
    \[p(1)+p(3)+p(5) = \frac{1+3+5}{21} = \frac{3}{7}.\]
    
\end{parts}
\end{solution}

% 2   % % % % % % % % % % % % % % % % % % % % %
\question{
Seja $X$ uma variável aleatória discreta com função de probabilidade dada por
\[p(x) = \frac{c}{4^x},\quad x=0,1,\dots\]
Obtenha:
\begin{parts}
	\part o valor de $c$.
    
    \part a probabilidade de $X$ ser um número par.
\end{parts}
}
\begin{solution}
\begin{parts}
	\part Sabemos que
    \begin{align*}
    	\sum_{x=0}^\infty p(x) = 1
        	&\Rightarrow \sum_{x=0}^\infty \frac{c}{4^x}= 1\\
        	&\Rightarrow c = \dfrac{1}
            	{\sum_{x=0}^\infty \frac{1}{4^x}}\\
            &\Rightarrow c = \dfrac{1}{\frac{1}{1-1/4}}
            	= \frac{3}{4}.
    \end{align*}
    
    \part A probabilidade de $X$ ser par é equivalente à probabilidade de $X\in\{0,2,4,\dots\}$, ou seja,
    \begin{align*}
    	\sum_{x=0}^\infty p(2x)
        	&= \sum_{x=0}^\infty \frac{3/4}{4^{2x}}\\
            &= \frac{3}{4}\sum_{x=0}^\infty \frac{1}{16^{x}}\\
            &= \frac{3/4}{1-1/16}\\
            &= \frac{4}{5}.
    \end{align*}
\end{parts}
\end{solution}

% 3   % % % % % % % % % % % % % % % % % % % % %
\question{
Seja $X$ uma variável aleatória discreta com função de distribuição dada por
\[
	F(x) = \begin{cases}
 		0, & \text{se}~ x < 0, \\
      	1/2, & \text{se}~ 0\le x < 1, \\
        3/5, & \text{se}~ 1\le x < 2, \\
        4/5, & \text{se}~ 2\le x < 3, \\
        9/10, & \text{se}~ 3\le x < 4, \\
        1, & \text{se}~ x \ge 4.
	\end{cases}
\]
\begin{parts}
	\part Determine a função de probabilidade de $X$.
    
    \part Calcule $P(X = 0\mid X \text{ é par})$.
\end{parts}
}
\begin{solution}
\begin{parts}
	\part Sabemos que
    \begin{align*}
    	p(a) = P\{X=a\} = F(a) - \lim_{x\to a^-} F(a).
    \end{align*}
    Usando a expressão acima obtemos que
    \begin{align*}
    	p(0) &= 1/2 - 0 = 1/2,\\
        p(1) &= 3/5 - 1/2 = 1/10,\\
        p(2) &= 4/5 - 3/5 = 1/5,\\
        p(3) &= 9/10 - 4/5 = 1/10,\\
        p(4) &= 1 - 9/10 = 1/10.
    \end{align*}
    
    \part
    \begin{align*}
    	P(X = 0\mid X \text{ é par}) 
        	&= \frac{P(\{X=0\}, \{X\text{ é par})\}}
            	{P\{X\text{ é par}\}} \\
            &= \frac{P\{X=0\}}{P\{X\text{ é par}\}} \\
            &= \frac{p(0)}{p(0)+p(2)+p(4)} \\
            &= \frac{1/2}{1/2+1/5+1/10} \\
            &= \frac{5}{8}.
    \end{align*}
\end{parts}
\end{solution}

% 4   % % % % % % % % % % % % % % % % % % % % %
\question{
Quinze pessoas portadoras de determinada doença são selecionadas para se submeter a um tratamento. Sabe-se que este tratamento é eficaz na cura da doença em 80\% dos casos. Suponha que os indivíduos submetidos ao tratamento curam-se (ou não) independentemente uns dos outros e considere $X$ o número de curados dentre os 15 pacientes submetidos ao tratamento.
\begin{parts}
	\part Qual a distribuição de $X$?
    \part Qual a probabilidade de que os 15 pacientes sejam curados?
    \part Qual a probabilidade de que pelo menos dois não sejam curados?
\end{parts}
}
\begin{solution}
\begin{parts}
	
	\part A variável aleatória $X$ segue uma distribuição binomial de parâmetros $p=\text{0,8}$ e $n=15$.
    
    \part \[P\{X=15\} = \binom{15}{15}(\text{0,8})^{15} \approx \text{0,035}.\]
    
    \part Isso é equivalente ao complemento de 14 ou 15 pessoas serem curadas, ou seja,
    	\[1-(P\{X=14\}+P\{X=15\})
        	= 1-\binom{15}{15}(\text{0,8})^{15}
            -\binom{15}{14}(\text{0,8})^{14}(\text{0,2})
            \approx \text{0,833}.\]
    
\end{parts}
\end{solution}

% 5   % % % % % % % % % % % % % % % % % % % % %
\question{
Um estudante preenche por adivinhação um exame de múltipla escolha com 5 respostas possíveis (das quais uma correta) para cada uma de 10 questões.
\begin{parts}
	\part Qual a distribuição do número de respostas certas?
    \part Qual a probabilidade de que o estudante obtenha 9 ou mais respostas certas?
    \part Qual a probabilidade de que acerte pelo menos duas questões?
\end{parts}
}
\begin{solution}
\begin{parts}
	
	\part Seja $X$ a variável aleatória que corresponde ao número de respostas certas, então $X$ segue uma distribuição binomial de parâmetros $p=\frac{1}{5}$ e $n=10$.
    
    \part \[P\{X=9\}+P\{X=10\}
    	= \binom{10}{9}(\tfrac{1}{5})^{9}(\tfrac{4}{5})+\binom{10}{10}(\tfrac{1}{5})^{10}
        \approx \text{0,0000042}.\]
    
    \part Isso é equivalente ao complemento de acertar uma ou nenhuma questão, ou seja,
    	\[1-(P\{X=0\}+P\{X=1\})
        	= 1-\binom{10}{0}(\tfrac{4}{5})^{10}
            -\binom{10}{1}(\tfrac{1}{5})(\tfrac{4}{5})^{9}
            \approx \text{0,624}.\]
    
\end{parts}
\end{solution}

% 7   % % % % % % % % % % % % % % % % % % % % %
\setcounter{question}{6}
\question{
Em 1693, Samuel Pepys escreveu uma carta para Isaac Newton propondo-lhe um problema de probabilidade, relacionado a uma aposta que planejava fazer. Pepys perguntou o que é mais provável: obter pelo menos um 6 quando 6 dados são lançados, obter pelo menos dois 6 quando 12 dados são lançados, ou obter pelo menos três 6 quando 18 dados são lançados. Newton escreveu três cartas a Pepys e finalmente o convenceu de que o primeiro evento é mais provável. Calcule as três probabilidades.
}
\begin{solution}
	A probabilidade de obter pelo menos um 6 em seis dados honestos é o complemento de não obter nenhum 6, ou seja,
    \[1-\binom{6}{0}(\tfrac{5}{6})^6 \approx \text{0,6651}. \]
    A probabilidade de obter pelo menos dois 6 em doze dados honestos é o complemento de obter no máximo um 6, ou seja,
    \[1-\left(\binom{12}{0}(\tfrac{5}{6})^{12}
    	+\binom{12}{1}(\tfrac{1}{6})(\tfrac{5}{6})^{11}\right)
        \approx \text{0,6187}. \]
    A probabilidade de obter pelo menos três 6 em dezoito dados honestos é o complemento de obter no máximo dois 6, ou seja,
    \[1-\left(\binom{18}{0}(\tfrac{5}{6})^{18}
    	+\binom{18}{1}(\tfrac{1}{6})(\tfrac{5}{6})^{17}
        +\binom{18}{2}(\tfrac{1}{6})^2(\tfrac{5}{6})^{16}\right)
        \approx \text{0,5973}. \]
\end{solution}

\end{questions}