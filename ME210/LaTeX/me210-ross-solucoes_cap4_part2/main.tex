
\documentclass[answers, 12pt]{exam}
%\usepackage{amsmath}
\usepackage{amsthm}
%\usepackage{amsfonts}
\usepackage{amssymb}
\usepackage{mathrsfs}
\usepackage{graphicx}
\usepackage{mathtools}
\usepackage{caption}
\usepackage[brazil]{babel}
\usepackage[utf8]{inputenc}

\usepackage{url}

\usepackage{multicol}

\usepackage{tikz}
\usepackage{pgfplots}
\pgfplotsset{compat=1.15}

\newcommand{\Var}{\mathrm{Var}}
\newcommand{\euler}{\mathrm{e}}
\newcommand\diff{\mathrm{d}}

\renewcommand{\qedsymbol}{$\blacksquare$}
\renewcommand{\thequestion}{\arabic{section}.\arabic{question}}
\renewcommand{\solutiontitle}{\noindent\textbf{Solução:}\enspace}

\newtheorem{theorem}{Teorema}

\def\C{{\mathbb C}}
\def\N{{\mathbb N}}
\def\R{{\mathbb R}}
\def\Z{{\mathbb Z}}
\def\Q{{\mathbb Q}}
\def\E{{\mathbb E}}
\def\X{{\mathbb X}}
\def\ind{\mathds{1}}
\def\cal{\mathcal}
\def\T{\top}

\footer{}{\thepage}{}

\title{	ME210 - Probabilidade I - 2S 2017\\
		{\large \textit{Docente}: Marina Vachkovskaia}\\[2mm]
		{\large Soluções para problemas selecionados do livro\\[-2mm]
        \textit{Probabilidade: Um curso moderno com aplicações}
        8.ed. de Sheldon Ross}\\
}
\author{Plínio Santini Dester (\url{p103806@dac.unicamp.br})}

\begin{document}

%% Content goes here
\maketitle

Em caso de dúvidas, sugestões ou correções (inclusive erros de digitação), não hesite em mandar um e-mail.

\setcounter{section}{3}
\section{Problemas}

\begin{questions}

% 4.2  % % % % % % % % % % % % % % % % % % % % %
\setcounter{question}{1}
\question{
{\bf (Cuidados com o Teorema de Green)} Considere as duas integrais de linha clássicas:
  \[
  W = \oint_C \vec F \cdot \d\vec r \quad \text{e} \quad \Phi = \oint_C \vec F \cdot \hat N \d s,
  \]
  que calculam, respectivamente, trabalho e fluxo definidos por um campo vetorial $\vec F$ e uma curva fechada $C$. Suponha, neste exercício, que $C$ seja um círculo de raio $a > 0$ orientado positivamente.
  \begin{enumerate}[label=(\alph*)]
    \item Considere o campo vetorial dado por
    \[
    \vec F(x,y) = \frac{1}{x^2 + y^2}\big(-y \hat \i + x \hat \j\big).
    \]
    Este campo é conservativo? Calcule a integral de linha usando o Teorema de Green e a definição. Comente os resultados.
    \item Considere o campo vetorial dado por
    \[
    \vec F(x,y) = \frac{1}{x^2 + y^2}\big(x \hat \i + y \hat \j\big).
    \]
    Este campo é sem fontes? Calcule a integral de linha usando o Teorema de Green e a definição. Comente os resultados.
  \end{enumerate}
}
\begin{solution}
  \begin{enumerate}[label=(\alph*)]
    \item Este campo é conservativo onde $\vec F$ é definida, pois $\nabla \times \vec F = 0$. Dessa forma, se usarmos o Teorema de Green encontramos que o trabalho é nulo. Porém, ao fazermos a integral pela definição com a parametrização $\vec r(t) = (a\cos\theta,a\sin\theta)$, $\theta\in[0,2\pi]$. Temos que
    \begin{align*}
        W &= \oint_C \vec F \cdot \d\vec r
            = \int_0^{2\pi} \vec F(\vec r(t))\cdot \dot{\vec r}(t)\,\d t \\
            &= \int_0^{2\pi} \frac{1}{a^2}(-a\sin\theta, a\cos\theta)\cdot (-a\sin\theta,a\cos\theta)\,\d \theta \\
            &= 2\pi.
    \end{align*}
    O resultado foi diferente, pois neste caso uma das hipóteses do Teorema de Green não é satisfeita, que é a das derivadas parciais de primeira ordem serem contínuas em uma região aberta que contenha a região de integração. No ponto (0,0) o limite não existe.
    
    \item Este campo é sem fontes onde $\vec F$ é definida, pois $\nabla \cdot \vec F = 0$. Dessa forma, se usarmos o Teorema de Green (fluxo) encontramos que o fluxo é nulo.
    Porém, ao fazermos a integral pela definição
    \begin{align*}
        \Phi &= \oint_C \vec F \cdot \hat N\,\d s \\
            &= \int_0^{2\pi} \frac{1}{a^2}(a\cos\theta, a\sin\theta)\cdot(\cos\theta,\sin\theta)\, a\, \d \theta \\
            &= 2\pi.
    \end{align*}
    O resultado foi diferente, pois neste caso uma das hipóteses do Teorema de Green (fluxo) não é satisfeita, que é a das derivadas parciais de primeira ordem serem contínuas em uma região aberta que contenha a região de integração. No ponto (0,0) o limite não existe.
  \end{enumerate}
\end{solution}

% 4.5  % % % % % % % % % % % % % % % % % % % % %
\setcounter{question}{4}
\question{
Considere uma carga $Q_0$ distribuída em um volume esférico de raio $\rho_0$ centrado na origem. Esta distribuição de carga produz um campo elétrico $\vec E$ dado por
    \[
    \vec E(\rho,\theta,\phi) = \left\{ \begin{array}{rcl}
    \frac{Q_0}{4\pi \epsilon_0 \rho^2} \hat e_\rho & \text{se} & \rho \geq \rho_0, \vspace{0.2cm}\\
    \frac{Q_0\rho^2}{4\pi \epsilon_0 \rho_0^4} \hat e_\rho & \text{se} & \rho < \rho_0.
    \end{array} \right.
    \]
    Calcule o divergente $\nabla \cdot \vec E$ para os casos $0 < \rho < \rho_0$ e $\rho \geq \rho_0$ e determine a densidade de carga em cada caso.
}
\begin{solution}
    Vamos usar o divergente em coordenadas esféricas, ou seja,
    \[\nabla\cdot \vec E = \frac{1}{\rho^2} \frac{\partial}{\partial \rho} (\rho^2 E_\rho). \]
    Dessa forma, obtemos que a densidade de carga é dada por
    \[
        \epsilon_0 (\nabla\cdot \vec E) =
            \begin{cases}
                0, &\text{se } \rho \ge \rho_0, \\
                \frac{Q_0\rho}{\pi \rho_0^4}, &\text{se } \rho < \rho_0.
            \end{cases}
    \]
\end{solution}

% % 3.7  % % % % % % % % % % % % % % % % % % % % %
% \setcounter{question}{6}
% \question{
% {\bf (Integral Gaussiana)} Um resultado fundamental na área de estatística e probabilidade é dado pela igualdade
%   \[
%   \frac{1}{\sqrt{2\pi}}\int_{-\infty}^\infty \euler^{-x^2/2} \d x = 1.
%   \]
%   Neste exercício, provaremos esta igualdade.
%     \begin{enumerate}[label=(\alph*)]
%     \item Considere a integral imprópria
%     \[
%         I = \iint_{\R^2} \euler^{-x^2 - y^2} \d A.
%     \]
%     Use que $\R^2$ pode ser visto como um disco com raio infinito para mostrar que $I = \pi$.
%     \item Redefina a região da integral agora para o quadrado $R = [-a,a]\times[-a,a]$, com $a \to \infty$ e mostre que
%     \[
%         I = \iint_{\R^2} \euler^{-x^2 - y^2} \d A = \int_{-\infty}^\infty e^{-x^2} \d x  \int_{-\infty}^\infty \euler^{-y^2} \d y = \pi.
%     \]
%     \item Use o item anterior para provar a igualdade no início do exercício.
%   \end{enumerate}
% }
% \begin{solution}
%     \begin{enumerate}[label=(\alph*)]
%     \item Em coordenadas polares temos
%     \[
%         I = \int_0^{\infty} \int_0^{2\pi} \euler^{-r^2} r\, \d \theta \, \d r
%             = 2\pi \left. \frac{-\euler^{-r^2}}{2} \right|_0^\infty
%             = \pi.
%     \]
%     \item ~
%     \begin{align*}
%         I
%             &= \iint_{\R^2} \euler^{-x^2 - y^2} \d A \\
%             &= \int_{-\infty}^{\infty}\int_{-\infty}^{\infty} \euler^{-x^2}\euler^{- y^2} \d x \d y \\
%             &= \int_{-\infty}^\infty \euler^{-x^2} \d x  \int_{-\infty}^\infty \euler^{-y^2} \d y \\
%             &= \left( \int_{-\infty}^\infty \euler^{-x^2} \d x \right)^2 \\
%             &= \pi.
%     \end{align*}
%     \item Do item anterior, temos que
%     \[
%         \int_{-\infty}^\infty \euler^{-x^2} \d x = \sqrt{\pi}.
%     \]
%     Fazendo a mudança de variável $x = u/\sqrt{2}$ mostramos o resultado desejado.
%   \end{enumerate}
% \end{solution}

% % 3.9  % % % % % % % % % % % % % % % % % % % % %
% \setcounter{question}{8}
% \question{
% {\bf (Área de uma Superfície)} Se uma superfície suave $S$ for definida por $z = f(x,y)$, sendo $(x,y) \in D$, a sua área é dada por
%     \[
%     A(S) = \iint_D \sqrt{ 1 + \left( \frac{\partial z}{\partial x} \right)^2 + \left( \frac{\partial z}{\partial y} \right)^2 } \d A.
%     \]
%     Calcule a área do parabolóide $z = x^2 + y^2$ delimitado pelo plano $z = 9$.
% }
% \begin{solution}
%     \begin{align*}
%         A(S) &= \iint_D \sqrt{1+(2x)^2+(2y)^2}\,\d A \\
%             &= \int_{0}^{3} \int_{0}^{2\pi} \sqrt{1+4r^2} \,r \,\d \theta \,\d r\\
%             &= \frac{\pi}{6} \left( 37\sqrt{37} - 1 \right).
%     \end{align*}
    
% \end{solution}

\end{questions}

\newpage

\setcounter{section}{3}
\section{Exercícios Teóricos}
\begin{questions}

% 2   % % % % % % % % % % % % % % % % % % % % %
\setcounter{question}{1}
\question{
Se $X$ tem função distribuição $F_X$, qual é a
função distribuição de $\euler^X$?
}
\begin{solution}
	\begin{align*}
		F_{\euler^X}(x)
        	= P(\euler^X \le x)
            = P(X \le \ln(x))
            = F_X(\ln(x)).
	\end{align*}
\end{solution}

% 3   % % % % % % % % % % % % % % % % % % % % %
%\setcounter{question}{1}
\question{
Se $X$ tem função distribuição $F_X$, qual é a
função distribuição da variável aleatória
$\alpha X + \beta$, onde $\alpha$ e $\beta$ são constantes, $\alpha\neq 0$.
}
\begin{solution}
	\begin{align*}
		F_{\alpha X + \beta}(x)
        	= P(\alpha X + \beta \le x)
            = P\left(X \le \frac{x-\beta}{\alpha}\right)
            = F_X\left(\frac{x-\beta}{\alpha}\right).
	\end{align*}
\end{solution}

% 4   % % % % % % % % % % % % % % % % % % % % %
%\setcounter{question}{1}
\question{
Para uma variável aleatória inteira não
negativa $N$, mostre que \[E[N] = \sum_{i=1}^\infty P(N\ge i).\]
\textit{Dica:} $\sum_{i=1}^\infty P(N\ge i) = \sum_{i=1}^\infty \sum_{k=i}^\infty P(N = k)$. Agora troque a ordem da soma.
}
\begin{solution}
\begin{proof}
	\begin{align*}
		\sum_{i=1}^\infty P(N\ge i)
        	&= \sum_{i=1}^\infty \sum_{k=i}^\infty P(N = k)\\
            &= \sum_{k=1}^\infty \sum_{i=1}^k P(N = k)\\
            &= \sum_{k=1}^\infty k P(N = k)\\
            &= E[N].
	\end{align*}
\end{proof}
\end{solution}

% 6   % % % % % % % % % % % % % % % % % % % % %
\setcounter{question}{5}
\question{
Seja $X$ tal que \[P(X=1) = p = 1-P(X=-1),\] determine $c\neq 1$ tal que $E[c^X]=1$.
}
\begin{solution} Sabemos que $c\neq 0$, pois $E[0^X] = 0$. Então, $E[c^X] = c^1\cdot p + c^{-1}\cdot(1-p)$. Queremos que $E[c^X] = 1$, multiplicando ambos lados dessa equação por $c$ e rearranjando-a obtemos uma equação do segundo grau em $c$, \[p\,c^2-c+(1-p) = 0,\] cuja solução é \[c = \frac{1\pm \sqrt{1-4p(1-p)}}{2p} = \frac{1\pm|1-2p|}{2p}.\]
Dessa forma, as soluções são $c=1$, que não queremos e $c = \frac{1-p}{p}$.
\end{solution}

% 8   % % % % % % % % % % % % % % % % % % % % %
\setcounter{question}{7}
\question{
Determine $\Var(X)$ se \[P(X=a) = p = 1-P(X=b).\]
\vspace{-10mm}
}
\begin{solution}
	\begin{align*}
		E[X] &= a\cdot p+b\cdot(1-p),\\[1ex]
        E[X^2] &= a^2\cdot p+b^2\cdot(1-p),\\[1ex]
        \Var(X) &= E[X^2]-E[X]^2\\
        	&= a^2p(1-p)+b^2(1-p)(1-(1-p))-2\,a\,b\,p(1-p)\\
            &= (a^2-2\,a\,b+b^2)\,p(1-p)\\
            &= (a-b)^2\,p(1-p).
	\end{align*}
\end{solution}

% 16  % % % % % % % % % % % % % % % % % % % % %
\setcounter{question}{15}
\question{Seja $X$ uma variável aleatória de Poisson
com parâmetro $\lambda$. Mostre que ${P(X = i)}$
cresce monotonicamente e então decresce
monotonicamente à medida que $i$ cresce,
atingindo o seu máximo quando $i$ é o
maior inteiro não excedendo $\lambda$.\\
\textit{Dica:} Considere $P(X = i)/P(X = i-1)$.
}
\begin{solution}
	\begin{align*}
		\frac{P(X=i)}{P(X=i-1)}
        	= \frac{\frac{\lambda^i}{i!}\euler^{-\lambda}}
            	{\frac{\lambda^{i-1}}{(i-1)!}\euler^{-\lambda}}
            = \frac{\lambda}{i}
	\end{align*}
    A sequência $\{\frac{\lambda}{i}\}$ é monótona decrescente em relação à $i$. Logo, enquanto $\frac{\lambda}{i} \ge 1$, $P(X=i) \ge P(X=i-1)$ e a probabilidade cresce com $i$. Mas, quando $\frac{\lambda}{i} < 1$, $P(X=i) < P(X=i-1)$ e a probabilidade descresce com $i$. Dessa forma, temos um máximo quando $i = \lfloor\lambda\rfloor$, onde $\lfloor\lambda\rfloor$ é o maior inteiro menor ou igual à $\lambda$.
\end{solution}

% 18  % % % % % % % % % % % % % % % % % % % % %
\setcounter{question}{17}
\question{Seja $X$ uma variável aleatória de Poisson
com parâmetro $\lambda$. Que valor de $\lambda$ maximiza $P(X = k)$, $k \ge 0$?
}
\begin{solution}
	Seja $\lambda>0$. Sabemos que $P(X=k)=\frac{\lambda^k}{k!}\euler^{-\lambda}$. Para maximizar essa quantidade, que é função de $\lambda$, vamos achar o ponto crítico, isto é, quando a derivada dessa função em relação à $\lambda$ se anula. A derivada é dada por
    \begin{align*}
    	\dfrac{\mathrm{d}}{\mathrm{d}\lambda}
        		\left(\frac{\lambda^k}{k!}\euler^{-\lambda}\right)
        	= \left( 1 - \frac{\lambda}{k} \right)
            	\frac{\lambda^{k-1}}{(k-1)!}\euler^{-\lambda}.
    \end{align*}
    É fácil ver que essa função se anula quando $\lambda = k$. Para verificarmos que é um ponto de máximo, basta notar que essa função é maior que 0 quando $\lambda < k$ e é menor que 0 quando $\lambda > k$.
\end{solution}

\end{questions}
%\newpage

\vspace{10mm} {\LARGE \textbf{Desafio!}}
\begin{enumerate}
\item No dia do exame de Probabilidade I, o PED ficou responsável por distribuir as provas. Porém, ele se esqueceu que eram $m$ tipos de provas diferentes e distribuiu-as ao acaso. Sabe-se que foram impressas $n$ provas de cada tipo e todas as provas foram entregues aos $n m$ alunos. Por simplicidade, assuma que os alunos foram arranjados em uma sequência linear. Calcule o número esperado de
	\begin{enumerate}
	%\item pares de alunos adjacentes que receberam a mesma prova;
	\item alunos que possuam algum vizinho com o mesmo tipo de prova;
    \item blocos de alunos com a mesma prova. Por exemplo, se a sequência linear de tipos de prova for 1112212, então temos os blocos 111, 22, 1 e 2, totalizando 4 blocos.
	\end{enumerate}
\end{enumerate}




\end{document}