\begin{questions}

% 2   % % % % % % % % % % % % % % % % % % % % %
\setcounter{question}{1}
\question{
Se $X$ tem função distribuição $F_X$, qual é a
função distribuição de $\euler^X$?
}
\begin{solution}
	\begin{align*}
		F_{\euler^X}(x)
        	= P(\euler^X \le x)
            = P(X \le \ln(x))
            = F_X(\ln(x)).
	\end{align*}
\end{solution}

% 3   % % % % % % % % % % % % % % % % % % % % %
%\setcounter{question}{1}
\question{
Se $X$ tem função distribuição $F_X$, qual é a
função distribuição da variável aleatória
$\alpha X + \beta$, onde $\alpha$ e $\beta$ são constantes, $\alpha\neq 0$.
}
\begin{solution}
	\begin{align*}
		F_{\alpha X + \beta}(x)
        	= P(\alpha X + \beta \le x)
            = P\left(X \le \frac{x-\beta}{\alpha}\right)
            = F_X\left(\frac{x-\beta}{\alpha}\right).
	\end{align*}
\end{solution}

% 4   % % % % % % % % % % % % % % % % % % % % %
%\setcounter{question}{1}
\question{
Para uma variável aleatória inteira não
negativa $N$, mostre que \[E[N] = \sum_{i=1}^\infty P(N\ge i).\]
\textit{Dica:} $\sum_{i=1}^\infty P(N\ge i) = \sum_{i=1}^\infty \sum_{k=i}^\infty P(N = k)$. Agora troque a ordem da soma.
}
\begin{solution}
\begin{proof}
	\begin{align*}
		\sum_{i=1}^\infty P(N\ge i)
        	&= \sum_{i=1}^\infty \sum_{k=i}^\infty P(N = k)\\
            &= \sum_{k=1}^\infty \sum_{i=1}^k P(N = k)\\
            &= \sum_{k=1}^\infty k P(N = k)\\
            &= E[N].
	\end{align*}
\end{proof}
\end{solution}

% 6   % % % % % % % % % % % % % % % % % % % % %
\setcounter{question}{5}
\question{
Seja $X$ tal que \[P(X=1) = p = 1-P(X=-1),\] determine $c\neq 1$ tal que $E[c^X]=1$.
}
\begin{solution} Sabemos que $c\neq 0$, pois $E[0^X] = 0$. Então, $E[c^X] = c^1\cdot p + c^{-1}\cdot(1-p)$. Queremos que $E[c^X] = 1$, multiplicando ambos lados dessa equação por $c$ e rearranjando-a obtemos uma equação do segundo grau em $c$, \[p\,c^2-c+(1-p) = 0,\] cuja solução é \[c = \frac{1\pm \sqrt{1-4p(1-p)}}{2p} = \frac{1\pm|1-2p|}{2p}.\]
Dessa forma, as soluções são $c=1$, que não queremos e $c = \frac{1-p}{p}$.
\end{solution}

% 8   % % % % % % % % % % % % % % % % % % % % %
\setcounter{question}{7}
\question{
Determine $\Var(X)$ se \[P(X=a) = p = 1-P(X=b).\]
\vspace{-10mm}
}
\begin{solution}
	\begin{align*}
		E[X] &= a\cdot p+b\cdot(1-p),\\[1ex]
        E[X^2] &= a^2\cdot p+b^2\cdot(1-p),\\[1ex]
        \Var(X) &= E[X^2]-E[X]^2\\
        	&= a^2p(1-p)+b^2(1-p)(1-(1-p))-2\,a\,b\,p(1-p)\\
            &= (a^2-2\,a\,b+b^2)\,p(1-p)\\
            &= (a-b)^2\,p(1-p).
	\end{align*}
\end{solution}

% 16  % % % % % % % % % % % % % % % % % % % % %
\setcounter{question}{15}
\question{Seja $X$ uma variável aleatória de Poisson
com parâmetro $\lambda$. Mostre que ${P(X = i)}$
cresce monotonicamente e então decresce
monotonicamente à medida que $i$ cresce,
atingindo o seu máximo quando $i$ é o
maior inteiro não excedendo $\lambda$.\\
\textit{Dica:} Considere $P(X = i)/P(X = i-1)$.
}
\begin{solution}
	\begin{align*}
		\frac{P(X=i)}{P(X=i-1)}
        	= \frac{\frac{\lambda^i}{i!}\euler^{-\lambda}}
            	{\frac{\lambda^{i-1}}{(i-1)!}\euler^{-\lambda}}
            = \frac{\lambda}{i}
	\end{align*}
    A sequência $\{\frac{\lambda}{i}\}$ é monótona decrescente em relação à $i$. Logo, enquanto $\frac{\lambda}{i} \ge 1$, $P(X=i) \ge P(X=i-1)$ e a probabilidade cresce com $i$. Mas, quando $\frac{\lambda}{i} < 1$, $P(X=i) < P(X=i-1)$ e a probabilidade descresce com $i$. Dessa forma, temos um máximo quando $i = \lfloor\lambda\rfloor$, onde $\lfloor\lambda\rfloor$ é o maior inteiro menor ou igual à $\lambda$.
\end{solution}

% 18  % % % % % % % % % % % % % % % % % % % % %
\setcounter{question}{17}
\question{Seja $X$ uma variável aleatória de Poisson
com parâmetro $\lambda$. Que valor de $\lambda$ maximiza $P(X = k)$, $k \ge 0$?
}
\begin{solution}
	Seja $\lambda>0$. Sabemos que $P(X=k)=\frac{\lambda^k}{k!}\euler^{-\lambda}$. Para maximizar essa quantidade, que é função de $\lambda$, vamos achar o ponto crítico, isto é, quando a derivada dessa função em relação à $\lambda$ se anula. A derivada é dada por
    \begin{align*}
    	\dfrac{\mathrm{d}}{\mathrm{d}\lambda}
        		\left(\frac{\lambda^k}{k!}\euler^{-\lambda}\right)
        	= \left( 1 - \frac{\lambda}{k} \right)
            	\frac{\lambda^{k-1}}{(k-1)!}\euler^{-\lambda}.
    \end{align*}
    É fácil ver que essa função se anula quando $\lambda = k$. Para verificarmos que é um ponto de máximo, basta notar que essa função é maior que 0 quando $\lambda < k$ e é menor que 0 quando $\lambda > k$.
\end{solution}

\end{questions}