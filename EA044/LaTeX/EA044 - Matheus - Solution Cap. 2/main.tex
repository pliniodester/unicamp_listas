
\documentclass[answers, 12pt]{exam}
%\usepackage{amsmath}
\usepackage{amsthm}
%\usepackage{amsfonts}
\usepackage{amssymb}
\usepackage{mathrsfs}
\usepackage{graphicx}
\usepackage{mathtools}
\usepackage{caption}
\usepackage[brazil]{babel}
\usepackage[utf8]{inputenc}

\usepackage{url}

\usepackage{multicol}

\usepackage{tikz}
\usepackage{pgfplots}
\pgfplotsset{compat=1.15}

\newcommand{\Var}{\mathrm{Var}}
\newcommand{\euler}{\mathrm{e}}
\newcommand\diff{\mathrm{d}}

\renewcommand{\qedsymbol}{$\blacksquare$}
\renewcommand{\thequestion}{\arabic{section}.\arabic{question}}
\renewcommand{\solutiontitle}{\noindent\textbf{Solução:}\enspace}

\newtheorem{theorem}{Teorema}

\def\C{{\mathbb C}}
\def\N{{\mathbb N}}
\def\R{{\mathbb R}}
\def\Z{{\mathbb Z}}
\def\Q{{\mathbb Q}}
\def\E{{\mathbb E}}
\def\X{{\mathbb X}}
\def\ind{\mathds{1}}
\def\cal{\mathcal}
\def\T{\top}

\footer{}{\thepage}{}

\title{	%EA044 - Planejamento e Análise\\ de Sistemas de Produção - 2S 2018\\
		% {\large \textit{Docente}: Matheus Souza}\\[2mm]
		{\Large Soluções para problemas selecionados da apostila\\[-0mm]
        \textit{Otimização Matemática e Pesquisa Operacional}\\[-2mm]
        de André R. Fioravanti e Matheus Souza}\\[2mm]
        -- Capítulo 2 --
}
\author{Plínio Santini Dester (\url{p103806@dac.unicamp.br})}

\begin{document}

%% Content goes here
\maketitle

*Em caso de dúvidas, sugestões ou correções, não hesite em mandar um e-mail.

\setcounter{section}{1}
\section{Problemas}

\begin{questions}

% 2.3  % % % % % % % % % % % % % % % % % % % % %
\setcounter{question}{2}
\question{
{\bf (Lançamento Oblíquo)} Um projétil é disparado com um ângulo de elevação $\alpha$ medido a partir da horizontal e velocidade inicial $v_0$. Assumindo que a resistência do ar seja desprezível e que a única força externa sobre o móvel seja a força peso, determine o vetor de posição do móvel $\vec r(t)$ para $t \in [0,t_f]$, sendo $t_f$ o tempo que o móvel leva até atingir o solo novamente. Para que valor de $\alpha$ obtemos o maior alcance horizontal? Em que ponto da trajetória a curvatura é máxima? Interprete geometricamente.
}

\begin{solution}
    Sabemos que a posição do móvel
    \begin{align*}
        \vec r(t) = \left( v_0 \cos(\alpha)\,t, v_0 \sin(\alpha)\,t - g\,t^2/2 \right), \quad t \in [0,t_f].
    \end{align*}
    Ademais, $t_f>0$ deve satisfazer
    \begin{align*}
        v_0 \sin(\alpha)\,t_f - g\,t_f^2/2 = 0
            ~\Rightarrow t_f (v_0 \sin(\alpha) - g\,t_f/2) = 0
            ~\Rightarrow t_f = \frac{2 v_0}{g} \sin(\alpha).
    \end{align*}
    Dessa forma, o alcance horizontal máximo
    \begin{align*}
        D_H = v_0 \cos(\alpha)\,t_f =  \frac{v_0^2}{g}\,2 \sin(\alpha)\cos(\alpha)
            = \frac{v_0^2}{g} \sin(2\alpha) \le \frac{v_0^2}{g},
    \end{align*}
    e satisfaz a igualdade para $\alpha = \pi/4$.
    
    A curvatura é dada por
    \begin{align*}
        \kappa(t) 
            = \frac{|\dot{\vec r}(t) \times \ddot{\vec r}(t)|}{|\dot{\vec r}(t)|^3} 
            = \frac{|(v_0 \cos(\alpha),v_0 \sin(\alpha) - g\,t) \times (0,-g)|}{|(v_0 \cos(\alpha),v_0 \sin(\alpha) - g\,t)|^3}
            = \frac{g v_0 \cos(\alpha)}{(v_0^2 - 2 g v_0 \sin(\alpha)\,t+ g^2 t^2 )^{3/2}},
    \end{align*}
    que atinge um máximo em $t^* = \frac{v_0}{g} \sin(\alpha)$. Nesse instante, $\kappa(t^*) = \frac{g}{v_0^2}\sec^2(\alpha)$.
    
    O ponto de máxima curvatura
        $
        \vec r(t^*) = \frac{v_0^2}{2g} \left(\sin(2\alpha), \sin^2(\alpha) \right)
        $
    corresponde ao máximo da trajetória, ou seja, é o ponto no qual toda força peso atua como normal, portanto faz sentido ser o ponto de máxima curvatura.
    %
    Ademais, a trajetória é parabólica e esse ponto trata-se do vértice da parábola, que é o ponto de maior curvatura.
\end{solution}

\end{questions}

\newpage

% \setcounter{section}{4}
% \section{Exercícios Teóricos}
% \begin{questions}

% 5.27 % % % % % % % % % % % % % % % % % % % % %
\setcounter{question}{26}
\question{
Se $X$ é uniformemente distribuída em $(a, b)$, qual variável aleatória que varia linearmente com $X$ é uniformemente distribuída em $(0, 1)$?
}
\begin{solution}
Seja $Y = (X-a)/(b-a)$, então
\begin{align*}
	F_Y(y) &= P(Y\le y) = P((X-a)/(b-a) \le y)\\
    	&= P(X \le (b-a)\,y+a) = F_X((b-a)\,y+a).
\end{align*}
Derivando os dois lados da equação em relação à $y$ obtemos que
\begin{align*}
	f_Y(y) =  (b-a)f_X((b-a)\,y+a) =
    \begin{cases}
    	1, &\text{se }y\in(0,1);\\
        0, &\text{caso contrário.}
    \end{cases}
\end{align*}
Logo, $Y$ é uniformemente distribuída em $(0,1)$.\\[1mm]
\textit{Observação:} outra opção é fazer $Y = (b-X)/(b-a)$.
\end{solution}

% 5.29 % % % % % % % % % % % % % % % % % % % % %
\setcounter{question}{28}
\question{
Seja $X$ uma variável aleatória contínua
com função distribuição cumulativa $F$.
Defina a variável aleatória $Y$ como $Y = F(X)$.
Mostre que $Y$ é uniformemente distribuída em $(0, 1)$.
}
\begin{solution}
	Por simplicidade, vamos supor que $F: \mathbb{R} \to [0,1]$ seja estritamente crescente. Logo, $F$ é inversível e para $y \in (0,1)$,
	\begin{align*}
		F_Y(y) = P(Y\le y) = P(F(X)\le y) = P(X \le F^{-1}(y)) = F(F^{-1}(y)) = y.
	\end{align*}
    Dessa forma, quando $y \in \mathbb{R}$,
    \begin{align*}
    	F_Y(y) =
        \begin{cases}
    	0, &\text{se }y \le 0;\\
        y, &\text{se }0 < y < 1;\\
        1, &\text{se }y \ge 1;\\
    	\end{cases}
    \end{align*}
    o que caracteriza uma distribuição uniforme em $(0,1)$.\\[1mm]
    \textit{Observação:} Isso também acontece quando $F$ não é inversível.
\end{solution}

% 5.30 % % % % % % % % % % % % % % % % % % % % %
%\setcounter{question}{28}
\question{
Suponha que $X$ tenha função densidade
de probabilidade $f_X$. Determine a função
densidade de probabilidade da variável
aleatória $Y$ definida como $Y = aX + b$.
}
\begin{solution}
	Seja $a>0$,
	\begin{align*}
		F_Y(y) = P(Y\le y) = P(aX+b\le y) = P(X\le (y-b)/a) = F_X((y-b)/a).
	\end{align*}
    Derivando ambos lados da equação em relação à $y$ leva à
    \begin{align*}
    	f_Y(y) = \frac{f_X((y-b)/a)}{a}.
    \end{align*}
    Por outro lado, se $a<0$, então
    \begin{align*}
    	F_Y(y) &= P(X\ge (y-b)/a) = 1-P(X<(y-b)/a) = 1-F_X([(y-b)/a]^-)\\
        	&= 1-F_X((y-b)/a) \quad\text{(a variável aleatória é contínua).}
    \end{align*}
    Novamente, derivando ambos lados da equação em relação à $y$ leva à
    \begin{align*}
    	f_Y(y) = \frac{f_X((y-b)/a)}{-a}.
    \end{align*}
    Portanto, quando $a\neq 0$,
    \begin{align*}
    	f_Y(y) = \frac{f_X((y-b)/a)}{|a|}.
    \end{align*}
\end{solution}

\end{questions}
%\newpage

% \vspace{10mm} {\LARGE \textbf{Desafio!}}
% \begin{enumerate}
% \item Um investidor comprou uma ação muito instável. A cada mês, o valor dessa ação segue uma distribuição uniforme no intervalo $(0,1000)$ e é independente dos meses anteriores. O investidor pode vender a ação quando quiser, porém a cada mês que passa o dinheiro, para o investidor, vale $d$ vezes o mês anterior ($0<d<1$). Qual estratégia ele deve seguir para maximizar o retorno esperado na venda dessa ação? Se $d=4/5$, qual deve ter sido o valor máximo pago na compra da ação para que o investidor tenha um valor esperado de lucro positivo?
% \end{enumerate}

\end{document}