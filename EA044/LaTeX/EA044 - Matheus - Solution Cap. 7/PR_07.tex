
\begin{questions}

% 7.1  % % % % % % % % % % % % % % % % % % % % %
% \setcounter{question}{1}
\question{
A Figura 7.14 mostra a localização dos principais terminais ferroviários nos Estados Unidos e as ferrovias existentes. O objetivo da autoridade de transporte ferroviário daquele país é decidir quais das ferrovias devem ser expandidas para ter um aumento de capacidade. Em particular, este programa de expansão deve revitalizar a ligação direta entre Los Angeles (LA) e Chicago (CH). Excluindo-se estes dois terminais, todos os outros terminais podem ser conectados entre si direta ou indiretamente de modo que o comprimento total das ferrovias revitalizadas seja minimizado. Quais trechos devem ser revitalizados? Qual é a distância total que deve ser expandida?
}

\begin{solution}
    Transforme os nós LA e CH em um único nó e aplique o algoritmo de Prim (por exemplo) normalmente. No fim, adicione a ligação entre LA e CH.
\end{solution}

% 7.3 % % % % % % % % % % % % % % % % % % % % %
\setcounter{question}{2}
\question{
Suponha que $G = (V,E)$ seja um grafo não-direcionado com custos associados às arestas, que podem ser positivos ou negativos. Os algoritmos de Prim e de Kruskal ainda produzem a árvore geradora mínima para estes grafos? Justifique.
}

\begin{solution}
    Sim, pois se somarmos uma constante a todos os nós, ambos algoritmos não mudam seus resultados.
\end{solution}

% 7.4  % % % % % % % % % % % % % % % % % % % % %
% \setcounter{question}{8}
\question{
Considere o problema de {\em árvore geradora máxima}, ou seja, o problema de encontrar a árvore geradora que maximiza a soma dos custos (que agora podem ser vistos como utilidades) das suas arestas. Os algoritmos de Prim e de Kruskal, modificados para escolher a aresta mais pesada que cruza cada corte, ainda funcionam para este caso? 
}

\begin{solution}
    Sim, pois se multiplicarmos todos os nós por -1, os algoritmos originais farão exatamente o proposto.
\end{solution}

% 7.6 % % % % % % % % % % % % % % % % % % % % %
\setcounter{question}{5}
\question{
Suponha que você esteja dirigindo de Campinas para Porto Alegre. O tanque de combustível do seu carro, quando cheio, tem autonomia de $m$ quilômetros e você tem à sua disposição um mapa que fornece as distâncias entre postos de combustíveis ao longo da estrada. Suponha que $d_1 < d_2 < \cdots < d_n$ sejam as localizações dos postos de combustíveis ao logo da sua rota, sendo $d_i$ a distância do posto a partir de Campinas. Assuma que a distância entre dois postos consecutivos não excede $m$. O seu objetivo é chegar a Porto alegre com o menor número de paradas possível. Construa um algoritmo guloso para resolver este problema e mostre que ele fornece a solução ótima.
}

\begin{solution}
    O algoritmo guloso é abastecer no último posto antes de percorrer $m$ quilômetros desde o último abastecimento.\\
    Esse algoritmo é ótimo, pois para a primeira iteração, não seguir o algoritmo guloso coloca numa situação igual ou estritamente pior. A partir do novo ponto, caímos numa situação análoga e, assim, por indução o algoritmo guloso é ótimo.
\end{solution}

% 7.12 % % % % % % % % % % % % % % % % % % % % %
\setcounter{question}{11}
\question{
Um técnico de manutenção tem à sua disposição quatro equipamentos de teste, os quais são usados em $25\%$, $30\%$, $55\%$ e $15\%$ das chamadas. Estes itens pesam $20$, $30$, $40$ e $20$ quilogramas, respectivamente. Este funcionário pode carregar até $60$ quilogramas. Determine a combinação de equipamentos de maior utilidade que pode ser carregada pelo técnico.
}
\begin{solution}
    Esse problema é análogo ao da mochila. Seja $T(i,p)$ a solução ótima usando até o $i$-ésimo item e permitindo carregar $p$ de peso, então propomos a seguinte resolução por programação dinâmica:
    \begin{align*}
        T(i,p) = \max\{T(i-1,p), T(i-1,p-p_i)+u_i\},
    \end{align*}
    onde $p_i$ é o peso do item $i$ e $u_i$ é a utilidade do item $i$, $i\in\{1,2,3,4\}$.
    \begin{center}
        \begin{tabular}{c||c|c|c|c|c|c}
        $i\backslash\,p$  & $0$ & $20$ & $30$ & $40$ & $50$ & $60$ \\
        \hline \hline
        $0$ & 0 & 0 & 0 & 0 & 0 & 0\\
        \hline
        $1$ & 0 & 25 & 25 & 25 & 25 & 25\\
        \hline
        $2$ & 0 & 25 & 30 & 30 & 55 & 55\\
        \hline
        $3$ & 0 & 25 & 30 & 55 & 55 & 80\\
        \hline
        $4$ & 0 & 25 & 30 & 55 & 55 & 80\\
        \end{tabular}
    \end{center}
    Logo, a melhor solução é usar o primeiro e o terceiro item.
\end{solution}

% 7.13 % % % % % % % % % % % % % % % % % % % % %
% \setcounter{question}{11}
\question{
Considere um vetor de $n$ números inteiros e não-negativos, $a_1, \cdots, a_n$. O seu objetivo é determinar o maior valor de uma combinação de somas e produtos destes elementos, sem parentizações ou reordenações, ou seja, apenas inserindo sinais de $+$ ou $\times$ entre os elementos do vetor. Por exemplo, dado o vetor
\[
1~~2~~3~~1,
\]
o valor máximo que pode ser obtido com somas e produtos é
\[
1+2\times3+1 = 8.
\]
Resolva este problema com programação dinâmica.
}
\begin{solution}
    Note que se não houver 1's entre números adjacentes, então é sempre melhor multiplicá-los; se houver 1's no começo ou no fim do vetor é sempre melhor somá-los. Usando isso, podemos supor que nosso problema é do tipo
    \[ b_1~~k_2~~b_2~~k_3~~b_3~~\dots~~k_m~~b_m, \]
    onde $b_i$ é um número inteiro diferente de $1$ e pode ser interpretado como a multiplicação de todos os números adjacentes diferentes de 1 e $k_i$ é o número de 1's adjacentes entre números maiores que $1$. Nossa solução por programação dinâmica é dada por $T_m$, e pode ser encontrada através da seguinte equação para $i\in\{2,\dots,m\}$
    \begin{align*}
        T_i = \max\big\{&T_{i-1}+k_i+b_i,~ T_{i-2}+k_{i-1}+b_{i-1}b_i,~\dots,\\
            & ~T_2+k_3+b_3\cdots b_i,~ b_1+k_2+b_2\cdots b_i,~ b_1 \cdots b_i\big\}.
    \end{align*}
    Vamos exemplificar com o vetor: $\quad3~~\underbrace{1~~1}_{k_2=2}~~3~~\underbrace{1~~\dots~~1}_{k_3=9}~~2$,
    \begin{align*}
        T_2 &= \max\{3 + 2 + 3,~ \underline{3\cdot 3}\} = 9,\\
        T_3 &= \max\{\underline{9 + 9 + 2},~ 3 + 2 + 3\cdot 2,~ 3\cdot 3\cdot 2\} = 20.
    \end{align*}
    Logo, a melhor escolha é $\quad3 \cdot \underbrace{1\cdots 1}_{k_2=2}\cdot 3+\underbrace{1+\dots+1}_{k_3=9}+2$.
\end{solution}

\end{questions}
